\section{2.1}


\subsection{} %A


	\paragraph{} %1
		\begin{enumerate}
			\item %(1)
			      $\begin{vmatrix}
					      4  & -3 \\
					      -7 & 6
				      \end{vmatrix} = 24 - 21 = 3$.

			\item %(2)
			      $\begin{vmatrix}
					      \cos\alpha & -\sin\alpha \\
					      \sin\alpha & \cos\alpha
				      \end{vmatrix} = \cos^2\alpha + \sin^2\alpha = 1$.

			\item %(3)
			      $\begin{vmatrix}
					      1 & -2 & 3  \\
					      4 & 5  & -6 \\
					      7 & 0  & 9
				      \end{vmatrix} = 5 \times 9 + 2 \times 6 \times 7 - 3 \times 5 \times 7 + 4 \times 2 \times 9 = 96$.

			\item %(4)
			      $\begin{vmatrix}
					      x  & 1  & -1 \\
					      -1 & x  & 1  \\
					      1  & -1 & x
				      \end{vmatrix} = x^3 + 3x$.

			\item %(5)
			      $\begin{vmatrix}
					      0   & 0   & a_1 \\
					      0   & a_2 & 0   \\
					      a_3 & 0   & 0
				      \end{vmatrix} = -a_1a_2a_3$.
		\end{enumerate}


	\paragraph{} %2
		\begin{enumerate}
			\item %(1)
			      由 $\begin{pmatrix}
					      3  & -2 & 1  \\
					      -7 & 5  & 10
				      \end{pmatrix} \to \begin{pmatrix}
					      1 &   & 5 \\
					        & 1 & 7
				      \end{pmatrix}$ 故 $\begin{cases}
					      x=5 \\
					      y=7
				      \end{cases}$.

			\item %(2)
			      由 $\begin{pmatrix}
					      1 & 2 & 1 \\
					      3 & 7 & 0
				      \end{pmatrix} \to \begin{pmatrix}
					      1 &   & 7  \\
					        & 1 & -3
				      \end{pmatrix}$ 故 $\begin{cases}
					      x=7 \\
					      y=-3
				      \end{cases}$.

			\item %(3)
			      由 $\begin{pmatrix}
					      2 & -3 & -3 & 0  \\
					      1 & 4  & 6  & -1 \\
					      3 & -1 & 1  & 2
				      \end{pmatrix} \to \begin{pmatrix}
					      1 &   &   & -\frac{9}{4}  \\
					        & 1 &   & -\frac{41}{8} \\
					        &   & 1 & \frac{29}{8}
				      \end{pmatrix}$ 故 $\begin{cases}
					      x=-\frac{9}{4}  \\
					      y=-\frac{41}{8} \\
					      z=\frac{29}{8}
				      \end{cases}$.

			\item %(4)
			      由 $\begin{pmatrix}
					      0 & 1 & -1 & 0  \\
					      1 & 2 & 5  & -1 \\
					      2 & 0 & 1  & 3
				      \end{pmatrix} \to \begin{pmatrix}
					      1 &   &   & \frac{22}{13} \\
					        & 1 &   & -\frac{5}{13} \\
					        &   & 1 & -\frac{5}{13}
				      \end{pmatrix}$ 故 $\begin{cases}
					      x=\frac{22}{13} \\
					      y=-\frac{5}{13} \\
					      z=-\frac{5}{13}
				      \end{cases}$.
		\end{enumerate}


	\paragraph{} %3
		\begin{enumerate}
			\item %(1)
			      $\tau(4357261) = 1+4+1+1+6 = 12$. 偶排列.

			\item %(2)
			      $\tau(217986354) = 1+1+3+4+4+4+5 = 18$. 偶排列.
		\end{enumerate}


	\paragraph{} %4
		因为对于元素 $x_1,x_2,\dots,x_n$ 中任何两个不同的 $x_i$ 与 $x_j$,在 $x_1,x_2,\dots,x_n$ 与 $x_n,x_{n-1},\dots,x_1$ 中必有且只有一个构成逆序,所以这两个排列的逆序数之和应等于从 $n$ 个元素中任取两个元素的组合数 $C_n^2 = \frac{n(n-1)}{2}$
		故 $x_n \cdots x_2x_1$ 的逆序数为 $\frac{n(n-1)}{2} - k$.


	\paragraph{} %5
		$-a_{13}a_{21}a_{34}a_{42}$, $a_{14}a_{21}a_{33}a_{42}$.


	\paragraph{} %6
		$-a_{11}a_{23}a_{32}a_{44}$, $-a_{12}a_{23}a_{34}a_{41}$, $-a_{14}a_{23}a_{31}a_{42}$.


	\paragraph{} %7
		\begin{enumerate}
			\item %(1)
			      $\begin{vmatrix}
					      1 & 1  &   &   \\
					      2 & -1 &   &   \\
					        &    & 3 & 0 \\
					        &    & 4 & 4
				      \end{vmatrix} = \begin{vmatrix}
					      1 & 1  \\
					      2 & -1
				      \end{vmatrix} \cdot \begin{vmatrix}
					      3 & 0 \\
					      4 & 4
				      \end{vmatrix} = (-1-2) \times 12 = -36$.

			\item %(2)
			      $\begin{vmatrix}
					      0      & n      & 0      & \cdots & 0      \\
					      0      & 0      & n-1    & \cdots & 0      \\
					      \vdots & \vdots & \vdots & \ddots & \vdots \\
					      0      & 0      & \cdots & 0      & 2      \\
					      1      & 0      & \cdots & 0      & 0
				      \end{vmatrix} = (-1)^{n-1} \begin{vmatrix}
					      n & n-1 & \cdots &  & 2
				      \end{vmatrix} = (-1)^{n-1}n!$.

			\item %(3)
			      $\begin{vmatrix}
					      0 & 0 & 1  & 2 \\
					      0 & 0 & -1 & 2 \\
					      3 & 0 & 0  & 0 \\
					      1 & 2 & 0  & 0
				      \end{vmatrix} = (-1)^{2 \times 2} \begin{vmatrix}
					      1  & 2 \\
					      -1 & 2
				      \end{vmatrix} \cdot \begin{vmatrix}
					      3 & 0 \\
					      1 & 2
				      \end{vmatrix} = 24$.
		\end{enumerate}


\subsection{} %B


	\paragraph{} %1
		$f(x) = \begin{vmatrix}
				x & x & 1  & 2 & 3 \\
				1 & x & 0  & 2 & 4 \\
				2 & 5 & x  & 1 & 2 \\
				1 & 3 & -4 & x & 0 \\
				2 & 6 & 4  & 1 & x
			\end{vmatrix} = \begin{vmatrix}
				x-1 & 0 & 1  & 0 & -1 \\
				1   & x & 0  & 2 & 4  \\
				2   & 5 & x  & 1 & 2  \\
				1   & 3 & -4 & x & 0  \\
				2   & 6 & 4  & 1 & x
			\end{vmatrix}$
		则要出现 $x^4$ 项, 只能取主对角线元素
		$(x-1)x^4 \Rightarrow x^4$ 的系数为 $-1$.


	\paragraph{} %2
		由 $D_n = \begin{vmatrix}
				1      & 1      & \cdots & 1      \\
				1      & 1      & \cdots & 1      \\
				\vdots & \vdots & \ddots & \vdots \\
				1      & 1      & \cdots & 1
			\end{vmatrix} = 0$, 反行列式定义知 $D$ 为偶排列个数减奇排列个数
		所以奇偶排列各半.


	\paragraph{} %3
		$\begin{vmatrix}
				a_1 & a_2 & a_3 & a_4 & a_5 \\
				b_1 & b_2 & b_3 & b_4 & b_5 \\
				c_1 & c_2 &     &     &     \\
				d_1 & d_2 &     &     &     \\
				e_1 & e_2 &     &     &
			\end{vmatrix} = 0$ 的一般项可表示为 $(-1)^{N(j_1,j_2,j_3,j_4,j_5)} a_{1j_1}a_{2j_2}a_{3j_3}a_{4j_4}a_{5j_5}$
		记 $a_{1i} = a_i$, $a_{2i} = b_i$, $a_{3i} = c_i$ ($i < 3$), $a_{4i} = d_i$ ($i < 3$), $a_{5i} = e_i$ ($i < 3$), $a_{mn} = 0$ ($m > 2$, $n > 2$)
		则一般项的列下标 $j_3,j_4,j_5$ 只能在 $1,2,3,4,5$ 中取3个不同值,
		故 $j_3,j_4,j_5$ 必在 $3,4,5$ 中取一个数, 从而至少有一项都包含至少一个 $0$ 因子, 故任意一项必为 $0$. 从而该行列式为 $0$.

\section{2.2}


\subsection{} %A


	\paragraph{} %1
		\begin{enumerate}
			\item %(1)
			      $\begin{vmatrix}
					      1  & 2  & 3  & 4 \\
					      3  & -1 & 6  & 7 \\
					      41 & -7 & -9 & 0 \\
					      -9 & 21 & 32 & 1
				      \end{vmatrix} = \begin{vmatrix}
					      1  & 2  & 3  & 10 \\
					      3  & -1 & 6  & 15 \\
					      41 & -7 & -9 & 25 \\
					      -9 & 21 & 32 & 45
				      \end{vmatrix} = 5\begin{vmatrix}
					      1  & 2  & 3  & 2 \\
					      3  & -1 & 6  & 3 \\
					      41 & -7 & -9 & 5 \\
					      -9 & 21 & 32 & 9
				      \end{vmatrix}$
			      故行列式是5的整数倍.

			\item %(2)
			      由 $|\alpha_1, \alpha_2, \beta_2, \alpha_3| = n \Rightarrow |\alpha_1, \alpha_2, \alpha_3, \beta_2| = -n$,
			      故 $|\alpha_1, \alpha_2, \alpha_3, (\beta_1 - \beta_2)| = m - (-n) = m + n$,
			      则 $|\alpha_3, \alpha_2, \alpha_1, (\beta_1 - \beta_2)| = -(m + n)$.
		\end{enumerate}


	\paragraph{} %2
		\begin{enumerate}
			\item %(1)
			      $\begin{vmatrix}
					      0 & -1 & -1 \\
					      1 & 0  & -1 \\
					      1 & 1  & 0
				      \end{vmatrix} = \begin{vmatrix}
					      1  & 0 & 0  & 0  \\
					      3  & 0 & -1 & -1 \\
					      1  & 1 & 0  & -1 \\
					      -1 & 1 & 1  & 0
				      \end{vmatrix} = \begin{vmatrix}
					      1  & 0 & 0  & 1 \\
					      3  & 0 & -1 & 1 \\
					      1  & 1 & 0  & 1 \\
					      -1 & 1 & 1  & 1
				      \end{vmatrix} = 0$.

			\item %(2)
			      $\begin{vmatrix}
					      4  & 1 & 2 & 4 \\
					      1  & 2 & 0 & 2 \\
					      10 & 5 & 2 & 0 \\
					      0  & 1 & 1 & 7
				      \end{vmatrix} = \begin{vmatrix}
					      4  & 1 & 1  & -3  \\
					      1  & 2 & -2 & -12 \\
					      10 & 5 & -3 & -35 \\
					      0  & 1 & 0  & 0
				      \end{vmatrix} = \begin{vmatrix}
					      4  & 1  & -3  \\
					      1  & -2 & -12 \\
					      10 & -3 & -35
				      \end{vmatrix} = 0$.

			\item %(3)
			      $\begin{vmatrix}
					      a^2 & (a+1)^2 & (a+2)^2 & (a+3)^2 \\
					      b^2 & (b+1)^2 & (b+2)^2 & (b+3)^2 \\
					      c^2 & (c+1)^2 & (c+2)^2 & (c+3)^2 \\
					      d^2 & (d+1)^2 & (d+2)^2 & (d+3)^2
				      \end{vmatrix} = \begin{vmatrix}
					      a^2 & 2a+1 & 4a+4 & 6a+9 \\
					      b^2 & 2b+1 & 4b+4 & 6b+9 \\
					      c^2 & 2c+1 & 4c+4 & 6c+9 \\
					      d^2 & 2d+1 & 4d+4 & 6d+9
				      \end{vmatrix}$
			      $= \begin{vmatrix}
					      a^2 & 2a+1 & 2 & 6 \\
					      b^2 & 2b+1 & 2 & 6 \\
					      c^2 & 2c+1 & 2 & 6 \\
					      d^2 & 2d+1 & 2 & 6
				      \end{vmatrix} = 0$.

			\item %(4)
			      $\begin{vmatrix}
					      a   & b   & a+b \\
					      b   & a+b & a   \\
					      a+b & a   & b
				      \end{vmatrix} = \begin{vmatrix}
					      2(a+b) & b   & a+b \\
					      2(a+b) & a+b & a   \\
					      2(a+b) & a   & b
				      \end{vmatrix} = 2(a+b)\begin{vmatrix}
					      1 & b   & a+b \\
					      1 & a+b & a   \\
					      1 & a   & b
				      \end{vmatrix}$
			      $= 2(a+b)\begin{vmatrix}
					      a   & -b \\
					      a-b & -a
				      \end{vmatrix} = 2(a+b)(ab-a^2-b^2) = -2(a^3+b^3)$.
		\end{enumerate}


	\paragraph{} %3
		$\begin{vmatrix}
				a_1+b_1 & a_1+b_2 & \cdots & a_1+b_n \\
				a_2+b_1 & a_2+b_2 & \cdots & a_2+b_n \\
				\vdots  & \vdots  & \ddots & \vdots  \\
				a_n+b_1 & a_n+b_n & \cdots & a_n+b_n
			\end{vmatrix} = \begin{vmatrix}
				a_1+b_1 & a_1+b_2 & \cdots & a_1+b_n \\
				a_2-a_1 & a_2-a_1 & \cdots & a_2-a_1 \\
				\vdots  & \vdots  & \ddots & \vdots  \\
				a_n-a_1 & a_n-a_1 & \cdots & a_n-a_1
			\end{vmatrix}$
		故 $n=1$ 时,值为 $a_1+b_1$,
		$n=2$ 时,$\begin{vmatrix}
				a_1+b_1 & a_1+b_2 \\
				a_2-a_1 & a_2-a_1
			\end{vmatrix} = (a_1-a_2)(b_2-b_1)$
		当 $n \geq 3$ 时,第2行与第n行成比例,值为0.


	\paragraph{} %4
		对 $A$ 作 $r_i + kr_k$ 变换,相当于 $A$ 加上一个第 $i$ 行为 $kr_k$ 的行列式,
		该行列式的第 $i$ 行与第 $k$ 行成比例,值为0.
		同理作 $c_i + kc_k$ 变换也相当于 $A$ 加上一个值为0的行列式
		故 $A \xrightarrow[r_i+kr_k]{c_i+kc_k} B$ 后,$|B| = |A|$.


	\paragraph{} %5
		$AA^T = \begin{pmatrix}
				a  & b  & c  & d  \\
				-b & a  & d  & -c \\
				-c & -d & a  & b  \\
				-d & c  & -b & a
			\end{pmatrix}
			\begin{pmatrix}
				a & -b & -c & -d \\
				b & a  & -d & c  \\
				c & d  & a  & -b \\
				d & -c & b  & a
			\end{pmatrix} = \begin{pmatrix}
				a^2+b^2+c^2+d^2 & 0               & 0        & 0        \\
				0               & a^2+b^2+c^2+d^2 & 0        & 0        \\
				0               & 0               & \sum a^2 & 0        \\
				0               & 0               & 0        & \sum a^2
			\end{pmatrix}$
		故 $|AA^T| = (a^2+b^2+c^2+d^2)^4$.
		又 $\begin{vmatrix}
				a  & b  & c  & d  \\
				-b & a  & d  & -c \\
				-c & -d & a  & b  \\
				-d & c  & -b & a
			\end{vmatrix} = -\begin{vmatrix}
				a & b  & c  & d  \\
				b & -a & -d & c  \\
				c & d  & -a & -b \\
				d & -c & b  & -a
			\end{vmatrix} = -\frac{1}{abcd}\begin{vmatrix}
				a^2 & ba  & ca  & da  \\
				b^2 & -ab & -db & cb  \\
				c^2 & dc  & -ac & bc  \\
				d^2 & -cd & bd  & -ad
			\end{vmatrix}$
		$= -\frac{1}{abcd}\begin{vmatrix}
				a^2             & ba  & ca  & da \\
				b^2             & -ab & -db & cb \\
				c^2             & dc  & -ac & bc \\
				a^2+b^2+c^2+d^2 & 0   & 0   & 0
			\end{vmatrix} = (-1)^{1+4} \cdot (-1) \cdot \frac{a^2+b^2+c^2+d^2}{abcd} \begin{vmatrix}
				ba  & ca  & da \\
				-ab & -db & cb \\
				dc  & -ac & bc
			\end{vmatrix}$
		$= \frac{a^2+b^2+c^2+d^2}{d} \begin{vmatrix}
				b  & c  & d  \\
				-a & -d & c  \\
				d  & -a & -b
			\end{vmatrix} = (a^2+b^2+c^2+d^2)^2$.


\subsection{} %B


	\paragraph{} %1
		\begin{enumerate}
			\item %(1)
			      $\begin{vmatrix}
					      1      & 1      & 1      & \cdots & 1      \\
					      -1     & 2      & 0      & \cdots & 0      \\
					      -1     & 0      & 3      & \cdots & 0      \\
					      \vdots & \vdots & \vdots & \ddots & \vdots \\
					      -1     & 0      & 0      & \cdots & n
				      \end{vmatrix} = \begin{vmatrix}
					      \sum_{i=1}^{n} \frac{1}{i} & 1      & 1      & 1      & \cdots & 1      \\
					      0                          & 2      & 0      & 0      & \cdots & 0      \\
					      0                          & 0      & 3      & 0      & \cdots & 0      \\
					      \vdots                     & \vdots & \vdots & \vdots & \ddots & \vdots \\
					      0                          & 0      & 0      & 0      & \cdots & n
				      \end{vmatrix}$
			      $= n! \left(1 + \frac{1}{2} + \frac{1}{3} + \cdots + \frac{1}{n}\right)$.

			\item %(2)
			      $\begin{vmatrix}
					      a_1    & 1      & 1      & \cdots & 1      \\
					      1      & a_2    & 0      & \cdots & 0      \\
					      1      & 0      & a_3    & \cdots & 0      \\
					      \vdots & \vdots & \vdots & \ddots & \vdots \\
					      1      & 0      & 0      & \cdots & a_n
				      \end{vmatrix} = \begin{vmatrix}
					      a_1 - \frac{1}{a_2} - \frac{1}{a_3} - \cdots - \frac{1}{a_n} & 1      & 1      & \cdots & 1      \\
					      0                                                            & a_2    & 0      & \cdots & 0      \\
					      0                                                            & 0      & a_3    & \cdots & 0      \\
					      \vdots                                                       & \vdots & \vdots & \ddots & \vdots \\
					      0                                                            & 0      & 0      & \cdots & a_n
				      \end{vmatrix}$
			      $= a_2 a_3 \cdots a_n \left(a_1 - \sum_{i=2}^{n} \frac{1}{a_i}\right)$.
		\end{enumerate}


	\paragraph{} %2
		证明: $\begin{vmatrix}
				n      & n-1    & \cdots & 3      & 2      & 1      \\
				n      & n-1    & \cdots & 3      & 2      & 2      \\
				n      & n-1    & \cdots & 3      & 3      & 3      \\
				\vdots & \vdots & \ddots & \vdots & \vdots & \vdots \\
				n      & n-1    & \cdots & n-1    & n-1    & n-1    \\
				n      & n      & \cdots & n      & n      & n
			\end{vmatrix} = \begin{vmatrix}
				-1     &        &        &        &    \\
				-1     & -1     &        &        &    \\
				-1     & -1     & -1     &        &    \\
				\vdots & \vdots & \vdots & \ddots &    \\
				-1     & -1     & -1     & \cdots & -1 \\
				n      & n      & n      & \cdots & n
			\end{vmatrix} = (-1)^{\frac{n(n-1)}{2}} \cdot (-1)^{n-1} \cdot n$
		$= (-1)^{\frac{(n-1)(n+2)}{2}} \cdot n$.


	\paragraph{} %3
		证明: 若有两行成比例, 则将比例常数提出后, 有两行相同,
		则交换这两行, 有 $|A| = -|A|$,
		得 $|A| = 0$.


	\paragraph{} %4
		\begin{enumerate}
			\item %(1)
			      证明: 在等式左端的 $k+t$ 阶行列式中, 取定前 $k$ 行, 由这 $k$ 行元素构成的 $k$ 阶子式中, 只有取前 $k$ 列时该子式不为 0, 根据拉普拉斯定理,
			      左边 $= \begin{vmatrix}
					      a_{11} & \cdots & a_{1k} \\
					      \vdots &        & \vdots \\
					      a_{k1} & \cdots & a_{kk}
				      \end{vmatrix} \cdot \begin{vmatrix}
					      b_{11} & \cdots & b_{1t} \\
					      \vdots &        & \vdots \\
					      b_{t1} & \cdots & b_{tt}
				      \end{vmatrix} \cdot (-1)^{(1+2+\cdots+k)+(1+2+\cdots+k)} = $右边.

			\item %(2)
			      证明: 同(1), 由拉普拉斯定理
			      左边 $= \begin{vmatrix}
					      a_{11} & \cdots & a_{1k} \\
					      \vdots &        & \vdots \\
					      a_{k1} & \cdots & a_{kk}
				      \end{vmatrix} \cdot \begin{vmatrix}
					      b_{11} & \cdots & b_{1t} \\
					      \vdots &        & \vdots \\
					      b_{t1} & \cdots & b_{tt}
				      \end{vmatrix} \cdot (-1)^{[(t+1)+(t+2)+\cdots+k]+[1+2+\cdots+k]}$
			      $= (-1)^{tk+2(1+2+\cdots+k)} \begin{vmatrix}
					      a_{11} & \cdots & a_{1k} \\
					      \vdots &        & \vdots \\
					      a_{k1} & \cdots & a_{kk}
				      \end{vmatrix} \cdot \begin{vmatrix}
					      b_{11} & \cdots & b_{1t} \\
					      \vdots &        & \vdots \\
					      b_{t1} & \cdots & b_{tt}
				      \end{vmatrix} = $右边.
		\end{enumerate}


	\paragraph{} %5
		$f(x) = \begin{vmatrix}
				2  & 1     & 2 & 3     \\
				2  & 5-x^2 & 2 & 3     \\
				10 & 5     & 2 & 1     \\
				10 & 5     & 2 & 2-x^2
			\end{vmatrix} = \begin{vmatrix}
				2     & 1       & 2 & 3 \\
				4-x^2 &         &   &   \\
				-8    & -14     &   &   \\
				-8    & -13-x^2 &   &
			\end{vmatrix} = \begin{vmatrix}
				2     & 1  & 2  & 3     \\
				4-x^2 &    & -8 & -14   \\
				      & -8 &    & 1-x^2
			\end{vmatrix}$
		$= -16(1-x^2)(4-x^2)$.
		则零点为 $x_{1,2} = \pm 1$, $x_{3,4} = \pm 2$.

\section{2.3}


\subsection{} %A


	\paragraph{} %1
		\begin{enumerate}
			\item %(1)
			      $A_{11} = \begin{vmatrix}
					      1 & 0 \\
					      2 & 5
				      \end{vmatrix} = 5$, $A_{12} = -\begin{vmatrix}
					      -3 & 0 \\
					      1  & 5
				      \end{vmatrix} = 15$, $A_{13} = \begin{vmatrix}
					      -3 & 1 \\
					      1  & 2
				      \end{vmatrix} = -7$,
			      $A_{21} = -\begin{vmatrix}
					      0 & 0 \\
					      2 & 5
				      \end{vmatrix} = 0$, $A_{22} = \begin{vmatrix}
					      2 & 0 \\
					      1 & 5
				      \end{vmatrix} = 10$, $A_{23} = -\begin{vmatrix}
					      2 & 0 \\
					      1 & 2
				      \end{vmatrix} = -4$,
			      $A_{31} = \begin{vmatrix}
					      0 & 0 \\
					      1 & 0
				      \end{vmatrix} = 0$, $A_{32} = -\begin{vmatrix}
					      2  & 0 \\
					      -3 & 0
				      \end{vmatrix} = 0$, $A_{33} = \begin{vmatrix}
					      2  & 0 \\
					      -3 & 1
				      \end{vmatrix} = 2$.

			\item %(2)
			      $A_{11} = \begin{vmatrix}
					      1 & 1 \\
					      4 & 3
				      \end{vmatrix} = -1$, $A_{12} = 1$, $A_{13} = 0$,
			      $A_{21} = 2$, $A_{22} = -2$, $A_{23} = 0$,
			      $A_{31} = 0$, $A_{32} = 0$, $A_{33} = 0$.
		\end{enumerate}


	\paragraph{} %2
		\begin{enumerate}
			\item %(1)
			      $\begin{vmatrix}
					      2 & -1 & 3  & 1  & 0 \\
					      1 & 2  & -1 & 4  & 3 \\
					      0 & -1 & -3 & 2  & 3 \\
					      4 & 5  & 0  & 3  & 1 \\
					      1 & -1 & 2  & -2 & 3
				      \end{vmatrix} = \begin{vmatrix}
					      2   & -1  & 3  & 1   & 0 \\
					      -11 & -13 & -1 & -5  & 0 \\
					      -12 & -16 & -3 & -7  & 0 \\
					      4   & 5   & 0  & 3   & 1 \\
					      -11 & -16 & 2  & -11 & 0
				      \end{vmatrix} = -\begin{vmatrix}
					      2   & -1  & 3  & 1   \\
					      -11 & -13 & -1 & -5  \\
					      -12 & -16 & -3 & -7  \\
					      -11 & -16 & 2  & -11
				      \end{vmatrix}$
			      $= -\begin{vmatrix}
					      2   & 0   & 3  & 1   \\
					      -11 & -18 & -1 & -5  \\
					      -12 & -23 & -3 & -7  \\
					      -11 & -27 & 2  & -11
				      \end{vmatrix} = -\begin{vmatrix}
					      0  & 0   & 0  & 1   \\
					      -1 & -18 & 14 & -5  \\
					      2  & -23 & 18 & -7  \\
					      11 & -27 & 35 & -11
				      \end{vmatrix} = \begin{vmatrix}
					      -1 & -18 & 14 \\
					      2  & -23 & 18 \\
					      11 & -27 & 35
				      \end{vmatrix}$
			      $= \begin{vmatrix}
					      -1 & -18  & 14  \\
					      0  & -59  & 46  \\
					      0  & -225 & 189
				      \end{vmatrix} = -\begin{vmatrix}
					      -59  & 46  \\
					      -225 & 189
				      \end{vmatrix} = 11151 - 10350 = 801$.

			\item %(2)
			      $\begin{vmatrix}
					      1 & 2 & 3 & 4 & 5 \\
					      2 & 3 & 4 & 5 & 6 \\
					      3 & 4 & 5 & 6 & 7 \\
					      4 & 5 & 6 & 7 & 8 \\
					      5 & 6 & 7 & 8 & 9
				      \end{vmatrix} = \begin{vmatrix}
					      1 & 2 & 3 & 4 & 5 \\
					      1 & 1 & 1 & 1 & 1 \\
					      1 & 1 & 1 & 1 & 1 \\
					      1 & 1 & 1 & 1 & 1 \\
					      1 & 1 & 1 & 1 & 1
				      \end{vmatrix} = 0$.

			\item %(3)
			      $\begin{vmatrix}
					      \cos\alpha & 1           &             &             &             \\
					      1          & 2\cos\alpha & 1           &             &             \\
					                 & 1           & 2\cos\alpha & 1           &             \\
					                 &             & 1           & 2\cos\alpha & 1           \\
					                 &             &             & 1           & 2\cos\alpha
				      \end{vmatrix} = \begin{vmatrix}
					      1          & \cos\alpha &            &            &   \\
					      \cos\alpha & 1          &            &            &   \\
					                 &            & 1          & \cos\alpha &   \\
					                 &            & \cos\alpha & 1          &   \\
					                 &            &            &            & 1
				      \end{vmatrix}$
			      $= \begin{vmatrix}
					      1          & \cos\alpha &                 &                 &             \\
					      \cos\alpha & 1          &                 &                 &             \\
					                 &            & 1-2\cos^2\alpha &                 &             \\
					                 &            &                 & 1-2\cos^2\alpha &             \\
					                 &            &                 &                 & 2\cos\alpha
				      \end{vmatrix} = \begin{vmatrix}
					      1-2\cos^2\alpha & 1           \\
					      -\cos\alpha     & 2\cos\alpha
				      \end{vmatrix}$
			      $= 1-2\cos^2\alpha - 2\cos^2\alpha + (1-2\cos^2\alpha)4\cos^2\alpha$
			      $= 1-2\sin^2 2\alpha = \cos 4\alpha$.

			\item %(4)
			      设 $D_n = \begin{vmatrix}
					      a+b & ab  &        &        &     \\
					      1   & a+b & ab     &        &     \\
					          & 1   & a+b    & \ddots &     \\
					          &     & \ddots & \ddots & ab  \\
					          &     &        & 1      & a+b
				      \end{vmatrix}_n$
			      则 $D_n = (a+b) \cdot D_{n-1} + ab \cdot \begin{vmatrix}
					      1 & ab  &        &        \\
					        & a+b & ab     &        \\
					        &     & \ddots & \ddots \\
					        &     &        & a+b
				      \end{vmatrix}_{n-1}$
			      $= (a+b)D_{n-1} + abD_{n-2}$, 又 $D_1 = a+b$, $D_2 = a^2+ab+b^2$
			      $\Rightarrow D_n - aD_{n-1} = b(D_{n-1} - aD_{n-2})$
			      $= b^2(D_{n-2} - aD_{n-3})$
			      $= b^{n-2}(D_2 - aD_1) = b^{n-2}(a^2+ab+b^2 - a^2 - ab) = b^n$
			      则 $\frac{D_n}{a^n} - \frac{D_{n-1}}{a^{n-1}} = \left(\frac{b}{a}\right)^n$
			      $\frac{D_n}{a^n} - \frac{D_1}{a} = \left(\frac{b}{a}\right)^2 + \left(\frac{b}{a}\right)^3 + \cdots + \left(\frac{b}{a}\right)^n = \left(\frac{b}{a}\right)^2 \frac{1-\left(\frac{b}{a}\right)^{n-1}}{1-\frac{b}{a}}$
			      $\Rightarrow D_n = a^n \left[ \frac{a+b}{a} + \frac{b^2}{a} \cdot \frac{1-\left(\frac{b}{a}\right)^{n-1}}{a-b} \right]$
			      $= \frac{(a^2-b^2)a^{n-1}}{a-b} + \frac{a^{n-1}-b^{n-1}}{a-b}b^2$
			      $= \frac{a^{n+1}-b^{n+1}}{a-b}$.
		\end{enumerate}


	\paragraph{} %3
		\begin{enumerate}
			\item %(1)
			      $\begin{pmatrix}
					      2 & 1 \\
					      4 & 3
				      \end{pmatrix}^{-1} = \frac{1}{|2\ 1|} \begin{pmatrix}
					      2 & 1 \\
					      4 & 3
				      \end{pmatrix}^{*} = \frac{1}{2} \begin{pmatrix}
					      3  & -1 \\
					      -4 & 2
				      \end{pmatrix}$.

			\item %(2)
			      $\begin{pmatrix}
					      1 & 1  & -1 \\
					      0 & 2  & 2  \\
					      1 & -1 & 0
				      \end{pmatrix}^{-1} = \frac{1}{\begin{vmatrix}
						      1 & 1  & -1 \\
						      0 & 2  & 2  \\
						      1 & -1 & 0
					      \end{vmatrix}} \begin{pmatrix}
					      1 & 1  & -1 \\
					      0 & 2  & 2  \\
					      1 & -1 & 0
				      \end{pmatrix}^{*} = \frac{1}{6} \begin{pmatrix}
					      2  & 1 & 4  \\
					      2  & 1 & -2 \\
					      -2 & 2 & 2
				      \end{pmatrix}$.
		\end{enumerate}


	\paragraph{} %4
		\begin{enumerate}
			\item %(1)
			      证明: 按第一行展开, 则 $x$ 只能出现在第一行元素中
			      故 $P(x)$ 的最高次数项为 $\begin{vmatrix}
					      1      & a_1     & a_1^2     & \cdots & a_1^{n-2}     \\
					      1      & a_2     & a_2^2     & \cdots & a_2^{n-2}     \\
					      \vdots & \vdots  & \vdots    & \ddots & \vdots        \\
					      1      & a_{n-1} & a_{n-1}^2 & \cdots & a_{n-1}^{n-2}
				      \end{vmatrix} (-1)^{n+1} x^{n-1}$
			      故 $P(x)$ 的次数不超过 $n-1$.

			\item %(2)
			      $\because P(x)$ 的次数不超过 $n-1$.
			      注意到 $x$ 为 $a_1, a_2, \dots, a_{n-1}$ 时 $P(x)$ 为 0
			      故 $P(x)$ 的根为 $\{a_1, a_2, \dots, a_{n-1}\}$ 中的不重复元素
			      1) 若 $a_i \neq a_j$ ($i \neq j$), 则 $P(x)$ 的根为 $a_1, a_2, \dots, a_{n-1}$
			      2) 若 $a_i = a_j$, 则 $P(x) \equiv 0$, 此时 $P(x)$ 的根为 $x \in \mathbb{R}$
			      由题设条件, $a_1, a_2, \dots, a_n$ 互异
			      故 $P(x)$ 的根为 $a_1, a_2, \dots, a_{n-1}$.

			\item %(3)
			      由于 $\begin{vmatrix}
					      1      & a_1     & a_1^2     & \cdots & a_1^{n-2}     \\
					      1      & a_2     & a_2^2     & \cdots & a_2^{n-2}     \\
					      \vdots & \vdots  & \vdots    & \ddots & \vdots        \\
					      1      & a_{n-1} & a_{n-1}^2 & \cdots & a_{n-1}^{n-2}
				      \end{vmatrix} = \prod_{1 \leq j < i \leq n-1} (a_i - a_j)$
			      则 $x^{n-1}$ 的系数为 $(-1)^{n+1} \prod_{1 \leq j < i \leq n-1} (a_i - a_j)$.
		\end{enumerate}


\subsection{} %B


	\paragraph{} %1
		\begin{enumerate}
			\item %(1)
			      在 $D_{2n} = \begin{vmatrix}
					      a &        &   &   & b \\
					        & \ddots &   &   &   \\
					        &        & a & b &   \\
					        &        & b & a &   \\
					      b &        &   &   & a
				      \end{vmatrix}_{2n}$ 中, 取定 $n$ 和 $n+1$ 行,
			      由这两行元素组成的所有 $2$ 阶子式中, 只有取第 $n$ 和 $n+1$ 列时子式不为 $0$.
			      故由拉普拉斯展开定理, 得
			      \[D_{2n} = \begin{vmatrix}
					      a & b \\
					      b & a
				      \end{vmatrix} \cdot (-1)^{n+(n+1)+n+(n+1)} \begin{vmatrix}
					      a &        &   &   & b \\
					        & \ddots &   &   &   \\
					        &        & a & b &   \\
					        &        & b & a &   \\
					      b &        &   &   & a
				      \end{vmatrix}_{2n-2}\]
			      \[= (a^2-b^2)D_{2n-2}\]
			      又 $D_2 = \begin{vmatrix}
					      a & b \\
					      b & a
				      \end{vmatrix} = a^2-b^2$.
			      故 $D_{2n} = (a^2-b^2)^{n-1}D_2 = (a^2-b^2)^n$.

			\item %(2)
			      将 $D_n = \begin{vmatrix}
					      a_1    & -1     & 0      & \cdots & 0  \\
					      a_2    & x      & -1     & \cdots & 0  \\
					      a_3    & 0      & x      & \ddots & 0  \\
					      \vdots & \vdots & \ddots & \ddots & -1 \\
					      a_n    & 0      & \cdots & 0      & x
				      \end{vmatrix}$ 按最后一行展开
			      \[D_n = x(-1)^{n+n} \begin{vmatrix}
					      a_1     & -1     & 0      & \cdots & 0  \\
					      a_2     & x      & -1     & \cdots & 0  \\
					      a_3     & 0      & x      & \ddots & 0  \\
					      \vdots  & \vdots & \ddots & \ddots & -1 \\
					      a_{n-1} & 0      & \cdots & 0      & x
				      \end{vmatrix} + (-1)^{n+1}a_n \begin{vmatrix}
					      -1 & -1 &        &    \\
					      x  & -1 &        &    \\
					         & x  & \ddots &    \\
					         &    & \ddots & -1
				      \end{vmatrix}\]
			      \[= xD_{n-1} + a_n\]
			      \[= x(xD_{n-2} + a_{n-1}) + a_n\]
			      \[= x^2D_{n-2} + a_{n-1}x + a_n\]
			      \[\cdots\]
			      \[= a_1x^{n-1} + a_2x^{n-2} + \cdots + a_{n-2}x^2 + a_{n-1}x + a_n\]
			      \[= \sum_{i=1}^{n} a_i x^{n-i}\]
		\end{enumerate}


	\paragraph{} %2
		由 $A^*BA = 2BA - 8E$
		得 $(2E - A^*)BA = 8E$
		即 $B = 8(2E - A^*)^{-1}A^{-1}$
		$= 8(A(2E - A^*))^{-1}$
		$= 8(2A - |A|E)^{-1}$
		$= 8(2A + 2E)^{-1}$
		$= 4\begin{pmatrix}
				1  & 2 & -2 \\
				-2 & 4 &    \\
				   &   & 2
			\end{pmatrix}^{-1}$
		$= \begin{pmatrix}
				2  & 4 & -6 \\
				-4 & 8 &    \\
				   &   & 2
			\end{pmatrix}$.


	\paragraph{} %3
		$|2A^*B^{-1}| = 2^n|A||A^{-1}B^{-1}| = \frac{2^n|A|^n}{|A||B|} = -\frac{4^n}{b}$.


	\paragraph{} %4
		\begin{enumerate}
			\item %(1)
			      $|A^*| = ||A|A^{-1}| = |A|^n|A^{-1}| = |A|^{n-1}$.

			\item %(2)
			      $(A^*)^* = |A^*|(A^*)^{-1} = |A|^{n-1} \cdot \frac{A}{|A|} = |A|^{n-2}A$.

			\item %(3)
			      $(kA)^* = |kA|(kA)^{-1} = k^{n-1}|A|A^{-1} = k^{n-1}A^*$.
		\end{enumerate}


	\paragraph{} %5
		由 $\begin{pmatrix}
				2      & 2      & 2      & \cdots & 2      & 1      \\
				0      & 1      & 1      & \cdots & 1      & 1      \\
				0      & 0      & 1      & \cdots & 1      & 1      \\
				\vdots & \vdots & \vdots & \ddots & \vdots & \vdots \\
				0      & 0      & 0      & \cdots & 1      & 1
			\end{pmatrix} \to \begin{pmatrix}
				1 &   &        &   & 1      & -1     \\
				  & 1 &        &   & 1      & -1     \\
				  &   & \ddots &   & \vdots & \vdots \\
				  &   &        & 1 & 1      & -1     \\
				  &   &        &   & 1      & 1
			\end{pmatrix}$
		故 $A^* = 2A^{-1} = \begin{pmatrix}
				1 & -2 &    &        &    \\
				  & 2  & -2 &        &    \\
				  &    & 2  & -2     &    \\
				  &    &    & \ddots & -2 \\
				0 &    &    &        & 2
			\end{pmatrix}$, 故 $\sum_{i,j=1}^{n} A_{ij} = 1$.


\subsection{} %C


	\paragraph{} %1
		\begin{enumerate}
			\item %(1)
			      \[ \begin{vmatrix}
					      1+x_{1}y_{1} & 1+x_{1}y_{2} & \cdots & 1+x_{1}y_{n} \\
					      1+x_{2}y_{1} & 1+x_{2}y_{2} & \cdots & 1+x_{2}y_{n} \\
					      \vdots       & \vdots       & \ddots & \vdots       \\
					      1+x_{n}y_{1} & 1+x_{n}y_{2} & \cdots & 1+x_{n}y_{n}
				      \end{vmatrix} = \begin{vmatrix}
					      1+x_{1}y_{1}       & 1+x_{1}y_{2}       & \cdots & 1+x_{1}y_{n}       \\
					      y_{1}(x_{2}-x_{1}) & y_{2}(x_{2}-x_{1}) & \cdots & y_{n}(x_{2}-x_{1}) \\
					      \vdots             & \vdots             & \ddots & \vdots             \\
					      y_{1}(x_{n}-x_{1}) & y_{2}(x_{n}-x_{1}) & \cdots & y_{n}(x_{n}-x_{1})
				      \end{vmatrix} \]

			      则当 \( n=1 \) 时, 值为 \( 1+x_{1}y_{1} \)

			      当 \( n=2 \) 时, \( \begin{vmatrix}
				      1+x_{1}y_{1} & 1+x_{1}y_{2} \\
				      1+x_{2}y_{1} & 1+x_{2}y_{2}
			      \end{vmatrix} = (x_{2}-x_{1})(y_{2}-y_{1}) \)

			      当 \( n \geq 3 \) 时, 第2行与第n行成比例, 值为0.
			\item %(2)
			      记 \( A = \begin{vmatrix}
				      1      & x_{1}  & \cdots & x_{1}^{k-1} & x_{1}^{k+1} & \cdots & x_{1}^{n} \\
				      1      & x_{2}  & \cdots & x_{2}^{k-1} & x_{2}^{k+1} & \cdots & x_{2}^{n} \\
				      \vdots & \vdots & \ddots & \vdots      & \vdots      & \ddots & \vdots    \\
				      1      & x_{n}  & \cdots & x_{n}^{k-1} & x_{n}^{k+1} & \cdots & x_{n}^{n}
			      \end{vmatrix} \)

			      考虑多项式 \( f(t) = \begin{vmatrix}
				      1      & x_{1}  & \cdots & x_{1}^{k-1} & x_{1}^{k} & x_{1}^{k+1} & \cdots & x_{1}^{n} \\
				      1      & x_{2}  & \cdots & x_{2}^{k-1} & x_{2}^{k} & x_{2}^{k+1} & \cdots & x_{2}^{n} \\
				      \vdots & \vdots & \ddots & \vdots      & \vdots    & \vdots      & \ddots & \vdots    \\
				      1      & x_{n}  & \cdots & x_{n}^{k-1} & x_{n}^{k} & x_{n}^{k+1} & \cdots & x_{n}^{n} \\
				      1      & t      & \cdots & t^{k-1}     & t^{k}     & t^{k+1}     & \cdots & t^{n}
			      \end{vmatrix} \)

			      则 \( f(t) = (x_{2}-x_{1})(x_{3}-x_{1})\cdots(t-x_{1})(x_{3}-x_{2})\cdots(t-x_{2})\cdots(t-x_{n}) \)

			      \( = \prod_{k=1}^{n}(t-x_{k}) \prod_{1\leq j<i\leq n}(x_{i}-x_{j}) \)

			      该行列式中第 \( n+1 \) 行第 \( k+1 \) 列元素 \( t^{k} \) 的代数余子式为
			      \[ t^{k}(-1)^{n+1+k+1}\begin{vmatrix}
					      1      & x_{1}  & \cdots & x_{1}^{k-1} & x_{1}^{k+1} & \cdots & x_{1}^{n} \\
					      1      & x_{2}  & \cdots & x_{2}^{k-1} & x_{2}^{k+1} & \cdots & x_{2}^{n} \\
					      \vdots & \vdots & \ddots & \vdots      & \vdots      & \ddots & \vdots    \\
					      1      & x_{n}  & \cdots & x_{n}^{k-1} & x_{n}^{k+1} & \cdots & x_{n}^{n}
				      \end{vmatrix} \]

			      考察 \( f(t) \) 中 \( t^{k} \) 的系数为 \( (-1)^{n-k}\cdot\left(\sum_{1\leq i_{1}<i_{2}<\cdots<i_{n-k}\leq n}\prod_{k=1}^{n-k}x_{i_{k}}\right)\cdot\prod_{1\leq j<i\leq n}(x_{i}-x_{j}) \)

			      故 \( A = \prod_{1\leq j<i\leq n}(x_{i}-x_{j})\left(\sum_{1\leq i_{1}<i_{2}<\cdots<i_{n-k}\leq n}\prod_{k=1}^{n-k}x_{i_{k}}\right). \)
			\item %(3)
			      在该行列式中取定第 \( n \) 和第 \( n+1 \) 行, 由这两行元素组成的所有2阶子式中, 只有取第 \( n \) 和 \( n+1 \) 列时的子式不为0.

			      由拉普拉斯展开定理, 得
			      \[ \begin{vmatrix}
					      a_{1}  & \cdots  & b_{2n} \\
					      \vdots & \ddots  & \vdots \\
					      a_{n}  & b_{n+1} & \cdots \\
					      b_{n}  & a_{n+1} & \cdots \\
					      \vdots & \ddots  & \vdots \\
					      b_{1}  & \cdots  & a_{2n}
				      \end{vmatrix} = \begin{vmatrix}
					      a_{n} & b_{n+1} \\
					      b_{n} & a_{n+1}
				      \end{vmatrix} \times (-1)^{n+(n+1)+n+(n+1)} \begin{vmatrix}
					      a_{1}   & \cdots  & b_{2n} \\
					      \vdots  & \ddots  & \vdots \\
					      a_{n-1} & b_{n+2} & \cdots \\
					      b_{n-1} & a_{n+2} & \cdots \\
					      \vdots  & \ddots  & \vdots \\
					      b_{1}   & \cdots  & a_{2n}
				      \end{vmatrix} \]

			      \( = \begin{vmatrix}
				      a_{n} & b_{n+1} \\
				      b_{n} & a_{n+1}
			      \end{vmatrix} \cdot \begin{vmatrix}
				      a_{n-1} & b_{n+2} \\
				      b_{n-1} & a_{n+2}
			      \end{vmatrix} \cdots \begin{vmatrix}
				      a_{1} & b_{2n} \\
				      b_{1} & a_{2n}
			      \end{vmatrix} \)

			      \( = \prod_{i=1}^{n}\left(a_{i}a_{2n+1-i} - b_{2n+1-i}b_{i}\right) \)
		\end{enumerate}


	\paragraph{} %2
		证明: 行列式的每一个元素都是两项和 \( a_{ij} + x \) (\( i,j=1,2,\dots,n \)), 这样行列式的每一列都可以看作由两个子列所组成, 第1子列元素为 \( a_{ij} \), 第2子列元素为 \( x \).

		则该行列式可以拆成 \( 2^{n} \) 个 \( n \) 阶行列式之和, 其中包含2列及以上的元素皆为 \( x \) 的 \( n \) 阶行列式值为0, 于是
		\[ \begin{vmatrix}
				a_{11}+x & a_{12}+x & \cdots & a_{1n}+x \\
				a_{21}+x & a_{22}+x & \cdots & a_{2n}+x \\
				\vdots   & \vdots   & \ddots & \vdots   \\
				a_{n1}+x & a_{n2}+x & \cdots & a_{nn}+x
			\end{vmatrix} = \begin{vmatrix}
				a_{11} & a_{12} & \cdots & a_{1n} \\
				a_{21} & a_{22} & \cdots & a_{2n} \\
				\vdots & \vdots & \ddots & \vdots \\
				a_{n1} & a_{n2} & \cdots & a_{nn}
			\end{vmatrix} + \begin{vmatrix}
				x      & a_{12} & \cdots & a_{1n} \\
				x      & a_{22} & \cdots & a_{2n} \\
				\vdots & \vdots & \ddots & \vdots \\
				x      & a_{n2} & \cdots & a_{nn}
			\end{vmatrix} + \cdots + \begin{vmatrix}
				a_{11} & \cdots & a_{1,n-1} & x      \\
				a_{21} & \cdots & a_{2,n-1} & x      \\
				\vdots & \ddots & \vdots    & \vdots \\
				a_{n1} & \cdots & a_{n,n-1} & x
			\end{vmatrix} \]

		\( = \begin{vmatrix}
			a_{11} & a_{12} & \cdots & a_{1n} \\
			a_{21} & a_{22} & \cdots & a_{2n} \\
			\vdots & \vdots & \ddots & \vdots \\
			a_{n1} & a_{n2} & \cdots & a_{nn}
		\end{vmatrix} + x\sum_{i=1}^{n}A_{i1} + x\sum_{i=1}^{n}A_{i2} + \cdots + x\sum_{i=1}^{n}A_{in} \)

		\( = \begin{vmatrix}
			a_{11} & a_{12} & \cdots & a_{1n} \\
			a_{21} & a_{22} & \cdots & a_{2n} \\
			\vdots & \vdots & \ddots & \vdots \\
			a_{n1} & a_{n2} & \cdots & a_{nn}
		\end{vmatrix} + x\sum_{j=1}^{n}\sum_{i=1}^{n}A_{ij} \)

		\( = \begin{vmatrix}
			a_{11} & a_{12} & \cdots & a_{1n} \\
			a_{21} & a_{22} & \cdots & a_{2n} \\
			\vdots & \vdots & \ddots & \vdots \\
			a_{n1} & a_{n2} & \cdots & a_{nn}
		\end{vmatrix} + x\sum_{i=1}^{n}\sum_{j=1}^{n}A_{ij} \)

\section{2.4}

\subsection{} %A


	\paragraph{} %1
		\begin{enumerate}
			\item %(1)
			      \( x_{1} = \frac{
				      \begin{vmatrix}
					      1 & -1 & 1  \\
					      0 & 1  & 3  \\
					      0 & 1  & 12
				      \end{vmatrix}
			      }{
				      \begin{vmatrix}
					      1 & -1 & 1  \\
					      0 & 1  & 3  \\
					      2 & 1  & 12
				      \end{vmatrix}
			      } = \frac{9}{1} = 9 \)

			      \( x_{2} = \frac{
				      \begin{vmatrix}
					      1 & 1 & 1  \\
					      0 & 0 & 3  \\
					      2 & 0 & 12
				      \end{vmatrix}
			      }{
				      \begin{vmatrix}
					      1 & -1 & 1  \\
					      0 & 1  & 3  \\
					      2 & 1  & 12
				      \end{vmatrix}
			      } = 6 \quad x_{3} = \frac{
				      \begin{vmatrix}
					      1 & -1 & 1 \\
					      0 & 1  & 0 \\
					      2 & 1  & 0
				      \end{vmatrix}
			      }{
				      \begin{vmatrix}
					      1 & -1 & 1  \\
					      0 & 1  & 3  \\
					      2 & 1  & 12
				      \end{vmatrix}
			      } = -2 \)

			      故 \( x = \begin{pmatrix}
				      9 \\
				      6 \\
				      -2
			      \end{pmatrix} \)
			\item %(2)
			      \( D = \begin{vmatrix}
				      1 & 2  & -1 & 4  \\
				      3 & 1  & 1  & 11 \\
				      2 & -3 & -1 & 4  \\
				      1 & 1  & 1  & 1
			      \end{vmatrix} = -80 \quad D_{1} = \begin{vmatrix}
				      1 & 2  & -1 & 4  \\
				      0 & 1  & 1  & 11 \\
				      0 & -3 & -1 & 4  \\
				      0 & 1  & 1  & 1
			      \end{vmatrix} = 50 \quad D_{3} = \begin{vmatrix}
				      1 & 2  & 1 & 4  \\
				      3 & 1  & 0 & 11 \\
				      2 & -3 & 1 & 4  \\
				      1 & 1  & 0 & 1
			      \end{vmatrix} = +50 \)

			      \( D_{2} = \begin{vmatrix}
				      1 & 1 & -1 & 4  \\
				      3 & 0 & 1  & 11 \\
				      2 & 1 & -1 & 4  \\
				      1 & 0 & 1  & 1
			      \end{vmatrix} = -10 \quad D_{4} = \begin{vmatrix}
				      1 & 2  & -1 & 1 \\
				      3 & 1  & 1  & 0 \\
				      2 & -3 & -1 & 0 \\
				      1 & 1  & 1  & 0
			      \end{vmatrix} = 10 \)

			      则 \( x_{1} = \frac{D_{1}}{D} = \frac{5}{8} \quad x_{2} = \frac{D_{2}}{D} = \frac{1}{8} \quad x_{3} = \frac{D_{3}}{D} = -\frac{5}{8} \quad x_{4} = \frac{D_{4}}{D} = -\frac{1}{8} \)
			\item %(3)
			      \( D = \begin{vmatrix}
				      2 & 1  & -5 & 1  \\
				      1 & -3 & 0  & -6 \\
				      0 & 2  & -1 & 2  \\
				      1 & 4  & -7 & 0
			      \end{vmatrix} = 27 \quad D_{1} = \begin{vmatrix}
				      1  & 1  & -5 & 1  \\
				      2  & -3 & 0  & -6 \\
				      -1 & 2  & -1 & 2  \\
				      0  & 4  & -7 & 0
			      \end{vmatrix} = 27 \)

			      \( D_{2} = \begin{vmatrix}
				      2 & 1  & -5 & 1  \\
				      1 & 0  & 0  & -6 \\
				      0 & 0  & -1 & 2  \\
				      1 & -1 & -7 & 0
			      \end{vmatrix} = 15 \quad D_{3} = \begin{vmatrix}
				      2 & 1  & 1  & 1  \\
				      1 & -3 & 2  & -6 \\
				      0 & 2  & 0  & 2  \\
				      1 & 4  & -1 & 0
			      \end{vmatrix} = 6 \quad D_{4} = \begin{vmatrix}
				      2 & 1  & -5 & 2  \\
				      1 & -3 & 0  & 0  \\
				      0 & 2  & -1 & 0  \\
				      1 & 4  & -7 & -1
			      \end{vmatrix} = -12 \)

			      故 \( x_{1} = \frac{D_{1}}{D} = 1 \quad x_{2} = \frac{D_{2}}{D} = \frac{5}{9} \quad x_{3} = \frac{D_{3}}{D} = \frac{2}{9} \quad x_{4} = \frac{D_{4}}{D} = -\frac{4}{9} \)
		\end{enumerate}


	\paragraph{} %2
		依题意, \( \begin{vmatrix}
			a & 1  & 1 \\
			1 & a  & 1 \\
			1 & 2a & 1
		\end{vmatrix} = 0 \) 即 \( a(1-a) = 0 \) 时有非零解.

		\( \therefore \) 当 \( a = 0 \) 或 \( a = 1 \) 时有非零解.


\subsection{} %B


	\paragraph{} %1
		证明: 设 \( P(x) = a_{0} + a_{1}x + \dots + a_{n-1}x^{n-1} \)

		则由 \( P(x_{i}) = y_{i} \), 可知 \( a_{0}, a_{1}, \dots, a_{n-1} \) 为该方程组的解.

		由于 \( D = \begin{vmatrix}
			1      & x_{1}  & \cdots & x_{1}^{n-1} \\
			1      & x_{2}  & \cdots & x_{2}^{n-1} \\
			\vdots & \vdots & \ddots & \vdots      \\
			1      & x_{n}  & \cdots & x_{n}^{n-1}
		\end{vmatrix} = \prod_{1 \leq i < j \leq n} (x_{i}-x_{j}) \neq 0 \)

		故该方程组有唯一解.

		即存在唯一的次数小于 \( n \) 的多项式 \( P(x) \), 使 \( P(x_{i}) = y_{i} \).


	\paragraph{} %2
		由 \( \left(\begin{array}{ccc:c}
				1  & 1  & b & 4     \\
				-1 & b  & 1 & b^{2} \\
				1  & -1 & 2 & -4
			\end{array}\right) \rightarrow \left(\begin{array}{ccc:c}
				1 & 1   & b   & 4       \\
				0 & b+1 & b+1 & b^{2}+4 \\
				0 & -2  & 2-b & -8
			\end{array}\right) \)

		则当 \( b = -1 \) 时, 方程组无解.

		\( b \neq -1 \) 时 \( \rightarrow \left(\begin{array}{ccc:c}
				1 & 1 & b   & 4                   \\
				0 & 1 & 1   & \frac{b^{2}+4}{b+1} \\
				0 & 0 & 4-b & \frac{2b(b-4)}{b+1}
			\end{array}\right) \)

		故当 \( b \neq -1, 4 \) 时, 方程组有唯一解.

		\( b = 4 \) 时, 方程组有无穷解, 此时 \( \rightarrow \left(\begin{array}{ccc:c}
				1 & 0 & 3 & 4 \\
				0 & 1 & 1 & 4 \\
				0 & 0 & 0 & 0
			\end{array}\right) \)

		取自由未知量 \( x_{3} \) 为1, 得基础解系 \( \eta_{1} = \begin{pmatrix}
			-3 \\
			-1 \\
			1
		\end{pmatrix} \)

		取 \( x_{3} = 0 \), 得特解 \( \eta^{*} = \begin{pmatrix}
			0 \\
			4 \\
			0
		\end{pmatrix} \)

		故 \( \begin{pmatrix}
			x_{1} \\
			x_{2} \\
			x_{3}
		\end{pmatrix} = k\begin{pmatrix}
			-3 \\
			-1 \\
			1
		\end{pmatrix} + \begin{pmatrix}
			0 \\
			4 \\
			0
		\end{pmatrix} \), \( k \) 为任意常数.


\subsection{} %C


	\paragraph{} %1
		设圆上的点为 \( (x, y) \), 则

		设圆方程为 \( A(x^{2}+y^{2}) + Dx + Ey + F = 0 \)

		则 \( \begin{cases}
			A(x^{2}+y^{2}) + Dx + Ey + F = 0                 \\
			A(x_{1}^{2}+y_{1}^{2}) + Dx_{1} + Ey_{1} + F = 0 \\
			A(x_{2}^{2}+y_{2}^{2}) + Dx_{2} + Ey_{2} + F = 0 \\
			A(x_{3}^{2}+y_{3}^{2}) + Dx_{3} + Ey_{3} + F = 0
		\end{cases} \)

		由该方程组为非零解, 则
		\[ \begin{vmatrix}
				x^{2}+y^{2}         & x     & y     & 1 \\
				x_{1}^{2}+y_{1}^{2} & x_{1} & y_{1} & 1 \\
				x_{2}^{2}+y_{2}^{2} & x_{2} & y_{2} & 1 \\
				x_{3}^{2}+y_{3}^{2} & x_{3} & y_{3} & 1
			\end{vmatrix} = 0 \]

		此即过这三点的圆方程.


	\paragraph{} %2
		设平面方程为 \( Ax + By + Cz + D = 0 \)

		则 \( \begin{cases}
			Ax + By + Cz + D = 0 \\
			Ax_{0} + By_{0} + Cz_{0} + D = 0
		\end{cases} \)

		由于该平面与 \( \pi_{1}, \pi_{2} \) 垂直, 则
		\[ (A, B, C) \cdot (a_{1}, b_{1}, c_{1}) = 0 \quad (A, B, C) \cdot (a_{2}, b_{2}, c_{2}) = 0 \]
		即 \( \begin{cases}
			Aa_{1} + Bb_{1} + Cc_{1} = 0 \\
			Aa_{2} + Bb_{2} + Cc_{2} = 0 \\
			Ax_{0} + By_{0} + Cz_{0} + D = 0
		\end{cases} \)

		该方程组有非零解, 则 \( \begin{vmatrix}
			x     & y     & z     & 1 \\
			a_{1} & b_{1} & c_{1} & 0 \\
			a_{2} & b_{2} & c_{2} & 0 \\
			x_{0} & y_{0} & z_{0} & 1
		\end{vmatrix} = 0 \)