\section{1.1}
 \subsection{} %A


	 \paragraph{} %1
		 依题意
		 $\begin{cases}
				 a + 2b = 1 \\
				 a - b = 2
			 \end{cases}$,
		 则
		 $a = \frac{\begin{vmatrix}
					 1 & 2  \\
					 2 & -1
				 \end{vmatrix}}
			 {\begin{vmatrix}
					 1 & 2  \\
					 1 & -1
				 \end{vmatrix}} = \frac{5}{3}, \quad
			 b = \frac{\begin{vmatrix}
					 1 & 1 \\
					 1 & 2
				 \end{vmatrix}}
			 {\begin{vmatrix}
					 1 & 2  \\
					 1 & -1
				 \end{vmatrix}} = -\frac{1}{3}$.


	 \paragraph{} %2
		 \begin{enumerate}
			 \item %(1)
			       $A + 3B =
				       \begin{pmatrix}
					       6 & -1 \\
					       3 & 0  \\
					       2 & 3
				       \end{pmatrix} + 3
				       \begin{pmatrix}
					       1 & -1 \\
					       0 & 5  \\
					       2 & 3
				       \end{pmatrix} =
				       \begin{pmatrix}
					       9  & -4 \\
					       3  & 6  \\
					       17 & 12
				       \end{pmatrix}$.

			 \item %(2)
			       $A^{\mathrm{T}} =
				       \begin{pmatrix}
					       6  & 3 & 2 \\
					       -1 & 0 & 3
				       \end{pmatrix}$,
			       $B^{\mathrm{T}} =
				       \begin{pmatrix}
					       1  & 0 & 5 \\
					       -1 & 2 & 3
				       \end{pmatrix}$,
			       则
			       $A^{T} - 2B^{T} =
				       \begin{pmatrix}
					       4 & 3  & -8 \\
					       1 & -4 & -3
				       \end{pmatrix}.$
		 \end{enumerate}


	 \paragraph{} %3
		 由
		 $\begin{pmatrix}
				 2  & 3 & -1 \\
				 -1 & 4 & 0
			 \end{pmatrix}
			 \begin{pmatrix}
				 -3 & 1 \\
				 7  & 3 \\
				 -1 & 5
			 \end{pmatrix} = \begin{pmatrix}
				 16 & 6  \\
				 31 & 11
			 \end{pmatrix}$,
		 故有
		 $\begin{cases}
				 y_1 = 16t_1 + 6t_2 \\
				 y_2 = 31t_1 + 11t_2
			 \end{cases}$.


	 \paragraph{} %4
		 $\begin{cases}
				 x_1 = 0 \\
				 y_1 = y
			 \end{cases}$,
		 几何意义为在 $y$ 轴上的投影.


	 \paragraph{} %5
		 \begin{enumerate}
			 \item %(1)
			       $\begin{pmatrix}
					       3  & 2  & 1  \\
					       -1 & -2 & -3
				       \end{pmatrix}
				       \begin{pmatrix}
					       1  & -1 & 1  \\
					       2  & 1  & -2 \\
					       -1 & 3  & -1
				       \end{pmatrix} =
				       \begin{pmatrix}
					       6  & 2   & -2 \\
					       -2 & -10 & 6
				       \end{pmatrix}$.

			 \item %(2)
			       $\begin{pmatrix}
					       6 & 0 & 8 & -3
				       \end{pmatrix}
				       \begin{pmatrix}
					       0.5 \\
					       -2  \\
					       2.5 \\
					       -1
				       \end{pmatrix} =
				       6 \times 0.5 + 8 \times 2.5 + 3 = 26$

			 \item %(3)
			       $\begin{pmatrix}
					       0.5 \\
					       -2  \\
					       2.5 \\
					       1
				       \end{pmatrix}
				       \begin{pmatrix}
					       6 & 0 & 8 & -3
				       \end{pmatrix} =
				       \begin{pmatrix}
					       3   & 0 & 4   & -1.5 \\
					       -12 & 0 & -16 & 6    \\
					       15  & 0 & 20  & -7.5 \\
					       6   & 0 & -8  & 3
				       \end{pmatrix}$

			 \item %(4)
			       $\begin{pmatrix}
					       1  & -1 & 1  \\
					       2  & 1  & -2 \\
					       -1 & 3  & -1
				       \end{pmatrix}
				       \begin{pmatrix}
					       3 & -1 & 0  \\
					       0 & 4  & -1 \\
					       3 & -3 & 0
				       \end{pmatrix} =
				       \begin{pmatrix}
					       6  & -8 & 1  \\
					       0  & 8  & -1 \\
					       -6 & 16 & -3
				       \end{pmatrix}$

			 \item %(5)
			       $\begin{pmatrix}
					       x_1 & x_2 & x_3
				       \end{pmatrix}
				       \begin{pmatrix}
					       1  & -1 & 1  \\
					       -1 & 1  & 3  \\
					       1  & 3  & -1
				       \end{pmatrix}
				       \begin{pmatrix}
					       x_1 \\
					       x_2 \\
					       x_3
				       \end{pmatrix} =
				       x_1^2 + x_2^2 - x_3^2 - 2x_1x_2 + 2x_1x_3 + 6x_2x_3$.

			 \item %(6)
			       $\begin{pmatrix}
					       3  & 2  & 1  \\
					       -1 & -2 & -3
				       \end{pmatrix}
				       \begin{pmatrix}
					       x_1 \\
					       x_2 \\
					       x_3
				       \end{pmatrix} =
				       \begin{pmatrix}
					       3x_1 + 2x_2 + x_3 \\
					       -x_1 - 2x_2 - 3x_3
				       \end{pmatrix}$.
		 \end{enumerate}


	 \paragraph{} %6
		 $AB = \begin{pmatrix}
				 5  & -1 & -1 \\
				 4  & 8  & 2  \\
				 -3 & -1 & 11
			 \end{pmatrix}, \quad
			 BA = \begin{pmatrix}
				 2 & 1  & -3 \\
				 2 & 11 & -1 \\
				 8 & -1 & 11
			 \end{pmatrix}, \quad
			 AB - BA = \begin{pmatrix}
				 3   & -2 & 2 \\
				 2   & -3 & 3 \\
				 -11 & 0  & 0
			 \end{pmatrix}$,

		 此题说明 $AB \neq BA$.


	 \paragraph{} %7
		 \textbf{性质2 证明}: 设 $A = (a_{ij})_{m \times n}$, $B = (b_{ij})_{n \times p}$, $C = (c_{ij})_{p \times q}$, 则
		 \[
			 (A + B)C = (a_{ij} + b_{ij})_{m \times n} (c_{ij})_{n \times q} = \left( \sum_{k=1}^{n} (a_{ik} + b_{ik}) c_{kj} \right)_{m \times q} = AC + BC.
		 \]
		 同理可证 $C(A + B) = CA + CB$.

		 \textbf{性质3 证明}: 设 $A = (a_{ij})_{m \times n}$, $B = (b_{ij})_{n \times p}$, 则
		 \[
			 \lambda(AB) = \lambda \left( \sum_{k=1}^{n} a_{ik} b_{kj} \right)_{m \times p} = \left( \sum_{k=1}^{n} (\lambda a_{ik}) b_{kj} \right)_{m \times p} = (\lambda A)B = A(\lambda B).
		 \]
		 \textbf{性质4 证明}: 设 $ = (a_{ij})_{m \times n}$, 则
		 \[
			 E_m A_{m \times n} = \begin{pmatrix}
				 1 &        &   \\
				   & \ddots &   \\
				   &        & 1
			 \end{pmatrix}
			 \begin{pmatrix}
				 a_{11} & \cdots & a_{1n} \\
				 \vdots & \ddots & \vdots \\
				 a_{m1} & \cdots & a_{mn}
			 \end{pmatrix},
		 \]
		 其中元素 $c_{ij} = 1 \cdot a_{ij} = a_{ij}$, 故 $E_m A_{m \times n} = A_{m \times n}$, 其余同理可证.


	 \paragraph{} %8
		 \begin{enumerate}
			 \item %(1)
			       $A = \begin{pmatrix}
					       2 & -1 \\
					       2 & -1
				       \end{pmatrix}, \quad
				       B = \begin{pmatrix}
					       -1 & -1 \\
					       2  & 2
				       \end{pmatrix}$.

			 \item %(2)
			       $A = \begin{pmatrix}
					       1 & -1 \\
					       1 & -1
				       \end{pmatrix}$.

			 \item %(3)
			       $A = \begin{pmatrix}
					       -1 & 1 \\
					       -2 & 2
				       \end{pmatrix}$.
		 \end{enumerate}


	 \paragraph{} %9
		 \begin{enumerate}
			 \item %(1)
			       记 $A = \begin{pmatrix}
					       0  & 1 \\
					       -1 & 0
				       \end{pmatrix}$,

			       则
			       $A^2 = \begin{pmatrix}
					       0  & 1 \\
					       -1 & 0
				       \end{pmatrix}
				       \begin{pmatrix}
					       0  & 1 \\
					       -1 & 0
				       \end{pmatrix} = \begin{pmatrix}
					       -1 & 0  \\
					       0  & -1
				       \end{pmatrix} = -E$.

			       当 $n = 2k$ 时, $A^n = A^{2k} = (-E)^k = (-1)^k E$;

			       当 $n = 2k+1$ 时, $A^n = A^{2k+1} = (-E)^k A = (-1)^k A$.

			 \item %(2)
			       $A^2 = \begin{pmatrix}
					       \cos\alpha  & \sin\alpha \\
					       -\sin\alpha & \cos\alpha
				       \end{pmatrix}^2 = \begin{pmatrix}
					       \cos2\alpha  & \sin2\alpha \\
					       -\sin2\alpha & \cos2\alpha
				       \end{pmatrix}$.

			       假设 $n = k$ 时, $A^k = \begin{pmatrix}
					       \cos k\alpha  & \sin k\alpha \\
					       -\sin k\alpha & \cos k\alpha
				       \end{pmatrix}$,

			       则当 $n = k+1$ 时,
			       \[
				       A^{k+1} = \begin{pmatrix}
					       \cos k\alpha  & \sin k\alpha \\
					       -\sin k\alpha & \cos k\alpha
				       \end{pmatrix}
				       \begin{pmatrix}
					       \cos\alpha  & \sin\alpha \\
					       -\sin\alpha & \cos\alpha
				       \end{pmatrix} = \begin{pmatrix}
					       \cos(k+1)\alpha  & \sin(k+1)\alpha \\
					       -\sin(k+1)\alpha & \cos(k+1)\alpha
				       \end{pmatrix}.
			       \]

			 \item %(3)
			       $A = \begin{pmatrix}
					       1 \\
					       1 \\
					       0
				       \end{pmatrix}
				       \begin{pmatrix}
					       1 & 1 & 0
				       \end{pmatrix} = \begin{pmatrix}
					       1 & 1 & 0 \\
					       1 & 1 & 0 \\
					       0 & 0 & 0
				       \end{pmatrix}$,
			       又 $\alpha^{\mathrm{T}} \alpha = \begin{pmatrix}
					       1 & 1 & 0
				       \end{pmatrix}
				       \begin{pmatrix}
					       1 \\
					       1 \\
					       0
				       \end{pmatrix} = 2$,

			       则 $A^n = (\alpha \alpha^{\mathrm{T}})^n = \alpha (\alpha^{\mathrm{T}} \alpha) (\alpha^{\mathrm{T}} \alpha) \cdots (\alpha^{\mathrm{T}} \alpha) \alpha^{\mathrm{T}} = 2^{n-1}\alpha\alpha^{\mathrm{T}}=2^{n-1}A=2^{n-1}
				       \begin{pmatrix}
					       1 & 1 & 0 \\
					       1 & 1 & 0 \\
					       0 & 0 & 0 \\
				       \end{pmatrix}$.
		 \end{enumerate}

	 \paragraph{} %10
		 证明: 令 $A = (a_{ij})_{m \times n}$, 则 $a_{ij} \in \mathbf{R}$, $A^{\mathrm{T}} = A$, 由 $A^2 = 0, \Rightarrow AA^{\mathrm{T}} = 0$

		 即 $(a_{ij})_{m \times n} (a_{ji})_{n \times m} = \left( \sum_{k=1}^{n} a_{ik} a_{ki} \right)_{m \times m} = 0$.

		 记 $c_{ij} = \sum_{k=1}^{n} a_{ik} a_{ki}$,

		 考虑 $c_{ii} = \sum_{k=1}^{n} a_{ik}^2 = 0 \Rightarrow a_{ik} = 0$,

		 故 $\forall i,j$, $\text{s.t. } a_{ij} = 0$. 故 $A = 0$.


	 \paragraph{} %11
		 \[
			 f(A) = A^2 + A + E = \begin{pmatrix}
				 8  & -1 & -1 \\
				 11 & 8  & 2  \\
				 8  & -1 & 11
			 \end{pmatrix} + \begin{pmatrix}
				 3 & -1 & 1  \\
				 2 & 1  & 3  \\
				 1 & 3  & -1
			 \end{pmatrix} + \begin{pmatrix}
				 1 &   &   \\
				   & 1 &   \\
				   &   & 1
			 \end{pmatrix} = \begin{pmatrix}
				 12 & -2 & 0  \\
				 13 & 10 & 5  \\
				 9  & 2  & 11
			 \end{pmatrix},
		 \]
		 \[
			 f(B) = B^n + B + E = \begin{pmatrix}
				 3^n &     &        \\
				     & 2^n          \\
				     &     & (-1)^n
			 \end{pmatrix} + \begin{pmatrix}
				 3 &   &    \\
				   & 2      \\
				   &   & -1
			 \end{pmatrix} + \begin{pmatrix}
				 1 &   &   \\
				   & 1 &   \\
				   &   & 1
			 \end{pmatrix} = \begin{pmatrix}
				 3^{n}+4 &         &          \\
				         & 2^{n}+3 &          \\
				         &         & (-1)^{n} \\
			 \end{pmatrix}.
		 \]


	 \paragraph{} %12
		 证明: 依题意, $A^H = A$, 即 $(\overline{A})^{\mathrm{T}} = A$, 即 $(\overline{a_{ji}})_{n \times n} = (a_{ij})_{n \times n}$,

		 则 $a_{ij} = \overline{a_{ji}}$.

		 又当 $i = j$ 时, $a_{ii} = \overline{a_{ii}}$, 则 $a_{ii}$ 是实数.


 \subsection{} %B


	 \paragraph{} %1
		 与 $A$ 可交换的 $B$ 必是二阶方阵, 设 $B = \begin{pmatrix}
				 a & b \\
				 c & d
			 \end{pmatrix}$,

		 $\begin{aligned}
				 \text{则 } AB = BA & \Rightarrow \begin{pmatrix}
					                                 1 & 1 \\
					                                 0 & 1
				                                 \end{pmatrix}
				 \begin{pmatrix}
					 a & b \\
					 c & d
				 \end{pmatrix} = \begin{pmatrix}
					                 a & b \\
					                 c & d
				                 \end{pmatrix}
				 \begin{pmatrix}
					 1 & 1 \\
					 0 & 1
				 \end{pmatrix}                                                       \\
				                   & \Rightarrow \begin{pmatrix}
					                                 a+c & b+d \\
					                                 c   & d
				                                 \end{pmatrix} = \begin{pmatrix}
					                                                 a & a+b \\
					                                                 c & c+d
				                                                 \end{pmatrix}       \\
				                   & \Rightarrow \begin{cases}
					                                 a+c = a   \\
					                                 b+d = a+b \\
					                                 c+d = d
				                                 \end{cases} \Rightarrow \begin{cases}
					                                                         c = 0 \\
					                                                         a = d
				                                                         \end{cases}
			 \end{aligned}$

		 故 $B = \begin{pmatrix}
				 a & b \\
				 0 & a
			 \end{pmatrix}$, $a,b$ 为实数.


	 \paragraph{} %2
		 证明: $\because A,B$ 可换, 即 $AB=BA$, 又 $A,B$ 满足
		 $$(A+B)^2 = (A+B)(A+B) = A^2 + AB + BA + B^2 = A^2 + 2AB + B^2$$
		 设当 $n=k$ 时, $(A+B)^n = \sum_{i=0}^{n} C_n^i A^{n-i} B^i$,

		 则当 $n=k+1$ 时,
		 \begin{align*}
			 (A+B)^{n+1} & = \left( \sum_{i=0}^{n} C_n^i A^{n-i} B^i \right) (A+B)                                       \\
			             & = C_n^0 A^{n+1} + (C_k^n + C_n^1) A^n B + \cdots + (C_n^{r-1} + C_n^r) A^{n-r+1} B^r + \cdots \\
			             & = \sum_{i=0}^{n+1} C_{n+1}^i A^{n-i} B^i
		 \end{align*}
		 证毕.


	 \paragraph{} %3
		 $A = \begin{pmatrix}
				 \lambda & 1       & 0       \\
				         & \lambda & 1       \\
				         &         & \lambda
			 \end{pmatrix} = \lambda E + \begin{pmatrix}
				 0 & 1 & 0 \\
				 0 & 0 & 1 \\
				 0 & 0 & 0
			 \end{pmatrix}$,

		 记 $B = \begin{pmatrix}
				 0 & 1 & 0 \\
				 0 & 0 & 1 \\
				 0 & 0 & 0
			 \end{pmatrix}$, 则 $B^2 = \begin{pmatrix}
				 0 & 0 & 1 \\
				 0 & 0 & 0 \\
				 0 & 0 & 0
			 \end{pmatrix}, B^3 = 0$ ($n \geq 3$),

		 则 $A^n = (\lambda E + B)^n = \lambda^n E + n \lambda^{n-1} B + \frac{n(n-1)}{2} \lambda^{n-2} B^2 = \begin{pmatrix}
				 \lambda^n & n\lambda^{n-1} & \frac{n(n-1)}{2}\lambda^{n-2} \\
				           & \lambda^n      & n\lambda^{n-1}                \\
				           &                & \lambda^n
			 \end{pmatrix}$,

		 故 $P_n(A) = \begin{pmatrix}
				 \lambda^n + \lambda + 1 & n\lambda^{n-1} + 1      & \frac{n(n-1)}{2}\lambda^{n-2} \\
				                         & \lambda^n + \lambda + 1 & n\lambda^{n-1} + 1            \\
				                         &                         & \lambda^n + \lambda + 1
			 \end{pmatrix}$,
		 $P_5(A) = \begin{pmatrix}
				 \lambda^5 + \lambda + 1 & 5\lambda^4 + 1          & 10\lambda^3             \\
				                         & \lambda^5 + \lambda + 1 & 5\lambda^4 + 1          \\
				                         &                         & \lambda^5 + \lambda + 1
			 \end{pmatrix}$.


	 \paragraph{} %4
		 证明: 令 $A = \begin{pmatrix}
				 a_{11} & 0      & \cdots & 0      \\
				 a_{21} & a_{22} & \cdots & 0      \\
				 \vdots & \vdots & \ddots & \vdots \\
				 a_{n1} & a_{n2} & \cdots & a_{nn}
			 \end{pmatrix}$, $B = \begin{pmatrix}
				 b_{11} & 0      & \cdots & 0      \\
				 b_{21} & b_{22} & \cdots & 0      \\
				 \vdots & \vdots & \ddots & \vdots \\
				 b_{n1} & b_{n2} & \cdots & b_{nn}
			 \end{pmatrix}$,

		 记 $C = AB$, 则 $c_{ij} = \sum_{k=1}^{n} a_{ik} b_{kj}$,

		 由假设, 当 $i < j$ 时 $a_{ij} = b_{ij} = 0$, 故 $c_{ij}$ 中各项因子都含 $0$, 则 $c_{ij} = 0$ ($i < j$),
		 故 $C$ 为下三角矩阵.


	 \paragraph{} %5
		 证明: 令 $A = \begin{pmatrix}
				 a_{11}  & a_{12}  & 0      & \cdots & 0          \\
				 a_{21}  & a_{22}  & a_{23} & \ddots & \vdots     \\
				 \vdots  & \vdots  & \ddots & \ddots & 0          \\
				 \vdots  & \vdots  &        & \ddots & a_{(n-1)n} \\
				 a_{n 1} & a_{n 2} & \cdots & \cdots & a_{nn}     \\
			 \end{pmatrix}$, $B = \begin{pmatrix}
				 b_{11}  & b_{12}  & 0      & \cdots & 0       \\
				 b_{21}  & b_{22}  & b_{23} & \ddots & \vdots  \\
				 \vdots  & \vdots  & \ddots & \ddots & 0       \\
				 b_{n 1} & b_{n 2} & \cdots & \cdots & b_{n n} \\
				         &         &        &        &         \\
			 \end{pmatrix}$,

		 记 $C = AB$, 则 $c_{ij} = a_{i1} b_{1j} + a_{i2} b_{2j} + \cdots + a_{in} b_{nj} = \sum_{k=1}^{n} a_{ik} b_{kj}$.

		 当 $j > i+1$ 时, $a_{ik} = b_{kj} = 0$, 则 $c_{ij}$ 中各项都含因子 $0$, 所以 $c_{ij} = 0$ ($j > i+1$).

		 故 $n$ 阶下 Hessenberg 矩阵乘积仍是下 Hessenberg 矩阵.

		 同理可证 $n$ 阶上 Hessenberg 矩阵乘积仍是上 Hessenberg 矩阵.


	 \paragraph{} %6
		 \begin{enumerate}
			 \item %(1)
			       由 $(A + A^{\mathrm{T}})^{\mathrm{T}} = A^{\mathrm{T}} + (A^{\mathrm{T}})^{\mathrm{T}} = A^{\mathrm{T}} + A = A + A^{\mathrm{T}}$,
			       知 $A + A^{\mathrm{T}}$ 为对称矩阵.

			       由 $(A - A^{\mathrm{T}})^{\mathrm{T}} = A^{\mathrm{T}} + (-A^{\mathrm{T}})^{\mathrm{T}} = A^{\mathrm{T}} - A = -(A - A^{\mathrm{T}})$,
			       知 $A - A^{\mathrm{T}}$ 为反对称矩阵.

			 \item %(2)
			       由 (1) 知 $A = \frac{1}{2} \left[ (A + A^{\mathrm{T}}) + (A - A^{\mathrm{T}}) \right]$.
		 \end{enumerate}


	 \paragraph{} %7
		 记 $A = \begin{pmatrix}
				 3  & 4  \\
				 -1 & -2
			 \end{pmatrix}$, $A_1 = \begin{pmatrix}
				 -1 & -4 \\
				 1  & 1
			 \end{pmatrix}$, $A_2 = \begin{pmatrix}
				 -1 & 0 \\
				 0  & 2
			 \end{pmatrix}$, $A_3 = \begin{pmatrix}
				 -1 & 4  \\
				 -1 & -1
			 \end{pmatrix}$,

		 则 $A = \frac{1}{3} A_1 A_2 A_3$, $\frac{1}{3} A_3 A_1 = E$,

		 $\begin{aligned}
				 \text{则 }A^{11} & = \frac{1}{3} A_1 A_2 A_3 \cdot \frac{1}{3} A_1 A_2 A_3 \cdots A_3 \frac{1}{3} A_1 A_2 A_3 \\
				                 & = \frac{1}{3} A_1 (A_2)^{11} A_3                                                           \\
				                 & = \frac{1}{3} \begin{pmatrix}
					                                 -1 & -4 \\
					                                 1  & 1
				                                 \end{pmatrix}
				 \begin{pmatrix}
					 -1 &        \\
					    & 2^{11}
				 \end{pmatrix}
				 \begin{pmatrix}
					 1  & 4  \\
					 -1 & -1
				 \end{pmatrix}                                                                                               \\
				                 & = \frac{1}{3} \begin{pmatrix}
					                                 1  & -2^{13} \\
					                                 -1 & 2^{11}
				                                 \end{pmatrix}
				 \begin{pmatrix}
					 1  & 4  \\
					 -1 & -1
				 \end{pmatrix}                                                                                               \\
				                 & = \frac{1}{3} \begin{pmatrix}
					                                 1+2^{13}  & 4+2^{13}  \\
					                                 -1-2^{11} & -4-2^{11}
				                                 \end{pmatrix}
			 \end{aligned}$


 \subsection{} %C
	 \paragraph{} %1
		 证明: 记 $C = AB = (c_{ij})_{m \times m}$, $D = BA = (d_{ij})_{n \times n}$,
		 则 $c_{ij} = \sum_{k=1}^{n} a_{ik} b_{kj}$, $d_{ij} = \sum_{k=1}^{m} b_{ik} a_{kj}$,
		 则 $\mathrm{tr}(AB) = \sum_{i=1}^{m} c_{ii} = \sum_{i=1}^{m} \sum_{k=1}^{n} a_{ik} b_{ki} = \sum_{k=1}^{n} \sum_{i=1}^{m} b_{ki} a_{ik} = \sum_{k=1}^{n} d_{kk} = \mathrm{tr}(BA)$.


	 \paragraph{} %2
		 证明: 因为 $A$ 与任意 $n$ 阶方阵可交换, 取 $B = \begin{pmatrix}
				 0      & 0      & \cdots & 0      \\
				 0      & 0      & \cdots & 0      \\
				 \vdots & \vdots & \ddots & \vdots \\
				 0      & 0      & \cdots & 0
			 \end{pmatrix}$,
		 则考虑 $AB$ 与 $BA$ 的第一行第二列元素有 $a_{11} = a_{22}$.
		 同理依次取 $B$ 为 $\begin{pmatrix}
				 0      & 0      & \cdots & 0      \\
				 0      & 0      & \cdots & 0      \\
				 \vdots & \vdots & \ddots & \vdots \\
				 0      & 0      & \cdots & 0
			 \end{pmatrix}, \begin{pmatrix}
				 0      & 0      & \cdots & 0      \\
				 0      & 0      & \cdots & 0      \\
				 \vdots & \vdots & \ddots & \vdots \\
				 0      & 0      & \cdots & 0
			 \end{pmatrix}, \cdots, \begin{pmatrix}
				 0      & 0      & \cdots & 0      \\
				 0      & 0      & \cdots & 0      \\
				 \vdots & \vdots & \ddots & \vdots \\
				 0      & 0      & \cdots & 0
			 \end{pmatrix}$,
		 可以得到 $a_{11} = a_{22} = \cdots = a_{nn}$.
		 又取 $B = \begin{pmatrix}
				 0      & 0      & \cdots & 0      \\
				 0      & 0      & \cdots & 0      \\
				 \vdots & \vdots & \ddots & \vdots \\
				 0      & 0      & \cdots & 0
			 \end{pmatrix}$, 可得第一行中除 $a_{11}$ 外均为 $0$.
		 依次取 $B$ 为 $\begin{pmatrix}
				 0 \\
				 1
			 \end{pmatrix}, \begin{pmatrix}
				 0 & 0 \\
				 1 & 0
			 \end{pmatrix}, \cdots, \begin{pmatrix}
				 0      & \cdots & 0      \\
				 \vdots & \ddots & \vdots \\
				 0      & \cdots & 1
			 \end{pmatrix}$,
		 可得除主对角线外均为 $0$.
		 故 $A$ 为 $n$ 阶数量矩阵.


	 \paragraph{} %3
		 证明: 用反证法. 假设 $\exists a_{ij} \neq 0$, 则考虑 $C = A^2$,
		 $c_{ii} = \sum_{k=1}^{n} a_{ik} a_{ki} = a_{i1} a_{1i} + a_{i2} a_{2i} + \cdots + a_{ii}^2 + a_{i,i+1} a_{i+1,i} + \cdots + a_{in} a_{ni}$.
		 因为 $A$ 为上三角矩阵, 则当 $i > j$ 时 $a_{ij} = 0$,
		 故 $c_{ii} = a_{ii}^2 > 0$,
		 则不可能 $\exists k$, 使 $A^k = 0$.
		 故矛盾. 则 $A$ 的主对角元素全为 $0$.

\section{1.2}


 \subsection{} %A


	 \paragraph{} %1
		 $\begin{aligned}
				 AB & = \left(\begin{array}{cc:cc}
					              1 & -1 & 0 & 0 \\
					              3 & 2  & 0 & 1 \\ \hdashline
					              3 & 0  & 0 & 0 \\
					              0 & 3  & 0 & 0
				              \end{array}\right)
				 \left(\begin{array}{cc:cc}
					       2  & 1 & 0 & 0 \\
					       -3 & 1 & 0 & 0 \\ \hdashline
					       1  & 0 & 3 & 0 \\
					       0  & 1 & 0 & 3
				       \end{array}\right)        \\
				    & = \begin{pmatrix}
					        6  & 0 & 3 & 0 \\
					        0  & 6 & 0 & 3 \\
					        6  & 3 & 0 & 0 \\
					        -9 & 3 & 0 & 0
				        \end{pmatrix}
			 \end{aligned}$
		 $\begin{aligned}
				 BA & = \left(\begin{array}{cc:cc}
					              2  & 1 & 0 & 0 \\
					              -3 & 1 & 0 & 0 \\ \hdashline
					              1  & 0 & 3 & 0 \\
					              0  & 1 & 0 & 3
				              \end{array}\right)
				 \left(\begin{array}{cc:cc}
					       1 & -1 & 0 & 0 \\
					       3 & 2  & 0 & 1 \\ \hdashline
					       3 & 0  & 0 & 0 \\
					       0 & 3  & 0 & 0
				       \end{array}\right)          \\
				    & = \left(\begin{array}{cc:cc}
					              5  & 0  & 2  & 1 \\
					              0  & 5  & -3 & 1 \\ \hdashline
					              10 & -1 & 1  & 0 \\
					              3  & 11 & 0  & 1
				              \end{array}\right)
			 \end{aligned}$
		 $AB - BA = \begin{pmatrix}
				 1   & 0  & 1  & -1 \\
				 0   & 1  & 3  & 2  \\
				 -4  & 4  & -1 & 0  \\
				 -12 & -8 & 0  & -1
			 \end{pmatrix}$.
		 $A\beta = \begin{pmatrix}
				 \alpha_1 & \alpha_2 & \alpha_3 & \alpha_4
			 \end{pmatrix}
			 \begin{pmatrix}
				 1 \\
				 2 \\
				 0 \\
				 -1
			 \end{pmatrix} = \alpha_1 + 2\alpha_2 - \alpha_4$.


	 \paragraph{} %2
		 $A^4 = \begin{pmatrix}
				 4  & 3 &   &   \\
				 -3 & 1 &   &   \\
				    &   & 3 & 0 \\
				    &   & 3 & 3
			 \end{pmatrix}^{4} = \begin{pmatrix}
				 \begin{pmatrix}
					 4  & 3 \\
					 -3 & 1
				 \end{pmatrix}^{4} &                   \\
				                   & \begin{pmatrix}
					                     3 & 0 \\
					                     3 & 3
				                     \end{pmatrix}^{4}
			 \end{pmatrix} =
			 \begin{pmatrix}
				 -176 & -15  &     &    \\
				 15   & -161 &     &    \\
				      &      & 81  & 0  \\
				      &      & 324 & 81
			 \end{pmatrix}$,
		 $B^2 = \begin{pmatrix}
				 2 & 0 & 0 \\
				 0 & 0 & 0 \\
				 0 & 3 & 0
			 \end{pmatrix}
			 \begin{pmatrix}
				 2 & 0 & 0 \\
				 0 & 0 & 0 \\
				 0 & 3 & 0
			 \end{pmatrix} = \begin{pmatrix}
				 2^2 & 0 & 0 \\
				 0   & 0 & 0 \\
				 0   & 0 & 0
			 \end{pmatrix}$,
		 假设 $n=k$ 时, $B^k = \begin{pmatrix}
				 2^k & 0 & 0 \\
				 0   & 0 & 0 \\
				 0   & 0 & 0
			 \end{pmatrix}$, 则当 $n=k+1$ 时,
		 \[B^{k+1} = \begin{pmatrix}
				 2^k & 0 & 0 \\
				 0   & 0 & 0 \\
				 0   & 0 & 0
			 \end{pmatrix}
			 \begin{pmatrix}
				 2 & 0 & 0 \\
				 0 & 0 & 0 \\
				 0 & 3 & 0
			 \end{pmatrix} = \begin{pmatrix}
				 2^{k+1} & 0 & 0 \\
				 0       & 0 & 0 \\
				 0       & 0 & 0
			 \end{pmatrix}\]
		 故 $B^n = \begin{pmatrix}
				 2^n & 0 & 0 \\
				 0   & 0 & 0 \\
				 0   & 0 & 0
			 \end{pmatrix}$.


	 \paragraph{} %3
		 $AB^{\mathrm{T}} = \left(\begin{array}{ccc:cc}
					 2  & -1          & 0 & 0  & 0  \\
					 -1 & \frac{2}{3} & 0 & 0  & 0  \\
					 0  & 0           & 4 & 0  & 0  \\ \hdashline
					 0  & 0           & 0 & 5  & -2 \\
					 0  & 0           & 0 & -7 & 3
				 \end{array}\right)
			 \left(\begin{array}{c:cc}
					 2 & 0 & 3 \\
					 0 & 2 & 0 \\
					 2 & 5 & 0 \\ \hdashline
					 0 & 2 & 0 \\
					 0 & 0 & 1
				 \end{array}\right) = \left(\begin{array}{c:cc}
					 4  & -2          & 6  \\
					 -2 & \frac{4}{3} & -3 \\
					 4  & 20          & 0  \\ \hdashline
					 0  & 5           & -2 \\
					 0  & -1          & 3
				 \end{array}\right)$


	 \paragraph{} %4
		 证明:
		 \[MM^{\mathrm{T}} = \begin{pmatrix}
				 E_m & 0 \\
				 0   & A
			 \end{pmatrix}
			 \begin{pmatrix}
				 E_m & 0              \\
				 0   & A^{\mathrm{T}}
			 \end{pmatrix} = \begin{pmatrix}
				 E_m & 0               \\
				 0   & AA^{\mathrm{T}}
			 \end{pmatrix} = E\]
		 \[M^{\mathrm{T}} M = \begin{pmatrix}
				 E_m & 0              \\
				 0   & A^{\mathrm{T}}
			 \end{pmatrix}
			 \begin{pmatrix}
				 E_m & 0 \\
				 0   & A
			 \end{pmatrix} = \begin{pmatrix}
				 E_m & 0                \\
				 0   & A^{\mathrm{T}} A
			 \end{pmatrix} = E\]
		 故 $\begin{pmatrix}
				 E_m & 0 \\
				 0   & A
			 \end{pmatrix}$ 为正交矩阵.


 \subsection{} %B


	 \paragraph{} %1
		 \begin{enumerate}
			 \item %(1)
			       证明: (1) \( A = \begin{pmatrix}
				       O & E_{n-1} \\
				       I & O
			       \end{pmatrix} \) 则 \( k=1 \) 时成立

			       假设 \( k=m \) 时 \( A^{m} = \begin{pmatrix}
				       O     & E_{n-m} \\
				       E_{m} & O
			       \end{pmatrix} \)

			       则 \( A^{m+1} = \begin{pmatrix}
				       O     & E_{n-m} \\
				       E_{m} & O
			       \end{pmatrix} \begin{pmatrix}
				       O & E_{n-1} \\
				       I & O
			       \end{pmatrix} \)

			       记 \( A_{1} = \begin{pmatrix}
				       0      & 1      & \cdots & 0      \\
				       \vdots & \ddots & \ddots & \vdots \\
				       0      & \cdots & 0      & 1      \\
				       0      & \cdots & 0      & 0
			       \end{pmatrix}_{m \times m} \quad A_{2} = \begin{pmatrix}
				       0      & \cdots & 0      \\
				       \vdots & \ddots & \vdots \\
				       1      & \cdots & 0
			       \end{pmatrix}_{m \times (n-m)} \)

			       \( A_{3} = \begin{pmatrix}
				       0      & \cdots & 0      \\
				       0      & \cdots & 0      \\
				       \vdots & \ddots & \vdots \\
				       1      & \cdots & 0
			       \end{pmatrix}_{(n-m) \times m} \quad A_{4} = \begin{pmatrix}
				       0      & 1      & \cdots & 0      \\
				       \vdots & \ddots & \ddots & \vdots \\
				       0      & \cdots & 0      & 1      \\
				       0      & \cdots & 0      & 0
			       \end{pmatrix}_{(n-m) \times (n-m)} \)

			       则 \( A^{m+1} = \begin{pmatrix}
				       O     & E_{n-m} \\
				       E_{m} & O
			       \end{pmatrix} \begin{pmatrix}
				       A_{1} & A_{2} \\
				       A_{3} & A_{4}
			       \end{pmatrix} = \begin{pmatrix}
				       A_{3} & A_{4} \\
				       A_{1} & A_{2}
			       \end{pmatrix} \)

			       \( = \begin{pmatrix}
				       0      & 1      & \cdots & 0      & 0      & \cdots & 0      \\
				       \vdots & \ddots & \ddots & \vdots & \vdots & \ddots & \vdots \\
				       0      & \cdots & 0      & 1      & 0      & \cdots & 0      \\
				       0      & \cdots & 0      & 0      & 0      & \cdots & 0      \\
				       \hdashline
				       0      & \cdots & 0      & 0      & 0      & \cdots & 0      \\
				       \vdots & \ddots & \ddots & \vdots & \vdots & \ddots & \vdots \\
				       0      & \cdots & 0      & 0      & 0      & \cdots & 1      \\
				       0      & \cdots & 0      & 0      & 0      & \cdots & 0
			       \end{pmatrix} \)

			       \( = \begin{pmatrix}
				       O       & E_{n-m-1} \\
				       E_{m+1} & O
			       \end{pmatrix} \)

			       故由数学归纳法知 \( A^{k} = \begin{pmatrix}
				       O     & E_{n-k} \\
				       E_{k} & O
			       \end{pmatrix}, k=1,2,\dots,n-1 \)
			 \item %(2)
			       在(1)中使用数学归纳法时并没有严格限制 \( k<n \), 故当 \( k=n \) 时
			       \( A^{n} = A^{n-1} \cdot A = E \).
		 \end{enumerate}

\section{1.3}


 \subsection{} %A


	 \paragraph{} %1
		 \begin{enumerate}
			 \item %(1)
			       由于 $\left(\begin{array}{cc:cc}
						       3  & -1 & 1 &   \\
						       -2 & 1  &   & 1
					       \end{array}\right) \to \left(\begin{array}{cc:cc}
						       1  & -\frac{1}{3} & \frac{1}{3} &             \\
						       -1 & \frac{1}{2}  &             & \frac{1}{2}
					       \end{array}\right)$
			       $\to \left(\begin{array}{cc:cc}
						       1 & -\frac{1}{3} & \frac{1}{3} &             \\
						       0 & \frac{1}{6}  & \frac{1}{3} & \frac{1}{2}
					       \end{array}\right) \to \left(\begin{array}{cc:cc}
						       1 & 0 & 1 & 1 \\
						       0 & 1 & 2 & 3
					       \end{array}\right)$,
			       故 $\begin{pmatrix}
					       3  & -1 \\
					       -2 & 1
				       \end{pmatrix}^{-1} = \begin{pmatrix}
					       1 & 1 \\
					       2 & 3
				       \end{pmatrix}$.

			 \item %(2)
			       $\begin{pmatrix}
					       \cos\theta & -\sin\theta     \\
					       \sin\theta & \cos\theta      \\
					                  &             & 3
				       \end{pmatrix}^{-1} = \begin{pmatrix}
					       \begin{pmatrix}
						       \cos\theta & -\sin\theta \\
						       \sin\theta & \cos\theta
					       \end{pmatrix}^{-1} &                  \\
					                                   & \frac{1}{3}
				       \end{pmatrix} = \begin{pmatrix}
					       \begin{pmatrix}
						       \cos\theta & -\sin\theta \\
						       \sin\theta & \cos\theta
					       \end{pmatrix}^{*} &                  \\
					                                   & \frac{1}{3}
				       \end{pmatrix} = \begin{pmatrix}
					       \cos\theta  & \sin\theta &             \\
					       -\sin\theta & \cos\theta &             \\
					                   &            & \frac{1}{3}
				       \end{pmatrix} \cdot $.

			 \item %(3)
			       $\begin{pmatrix}
					       5 & 2 &   &   \\
					       2 & 1 &   &   \\
					         &   & 1 & 8 \\
					         &   & 1 & 9
				       \end{pmatrix}^{-1} = \begin{pmatrix}
					       \begin{pmatrix}
						       5 & 2 \\
						       2 & 1
					       \end{pmatrix}^{-1} &                    \\
					                          & \begin{pmatrix}
						                            1 & 8 \\
						                            1 & 9
					                            \end{pmatrix}^{-1}
				       \end{pmatrix} = \begin{pmatrix}
					       \begin{pmatrix}
						       5 & 2 \\
						       2 & 1
					       \end{pmatrix}^{*} &                   \\
					                         & \begin{pmatrix}
						                           1 & 8 \\
						                           1 & 9
					                           \end{pmatrix}^{*}
				       \end{pmatrix} = \begin{pmatrix}
					       1  & -2 &    &    \\
					       -2 & 5  &    &    \\
					          &    & 9  & -8 \\
					          &    & -1 & 1
				       \end{pmatrix}$.

			 \item %(4)
			       $\begin{pmatrix}
					       2 &   \\
					       3 & 4
				       \end{pmatrix}^{-1} = \begin{pmatrix}
					       \frac{1}{2} &             \\
					       \frac{1}{3} & \frac{1}{4}
				       \end{pmatrix}$.
		 \end{enumerate}


	 \paragraph{} %2
		 \begin{enumerate}
			 \item %(1)
			       由 $\left(\begin{array}{ccc:c}
					       1 & 2  & -1 & 1 \\
					       3 & 4  & -2 & 2 \\
					       5 & -4 & 1  & 3
				       \end{array}\right) \to \left(\begin{array}{ccc:c}
					       1 & 2   & -1 & 1  \\
					       0 & -2  & 1  & -1 \\
					       0 & -14 & 6  & -2
				       \end{array}\right) \to \left(\begin{array}{ccc:c}
					       1 & 0  & 0  & 0  \\
					       0 & -2 & 1  & -1 \\
					       0 & 0  & -1 & 5
				       \end{array}\right)$
			       $\to \left(\begin{array}{ccc:c}
					       1 & 0 & 0            & 0           \\
					       0 & 1 & -\frac{1}{2} & \frac{1}{2} \\
					       0 & 0 & 1            & -5
				       \end{array}\right) \to \left(\begin{array}{ccc:c}
					       1 & 0 & 0 & 0  \\
					       0 & 1 & 0 & -2 \\
					       0 & 0 & 1 & -5
				       \end{array}\right)$,
			       故 $x = \begin{pmatrix}
					       0  \\
					       -2 \\
					       -5
				       \end{pmatrix}$.

			 \item %(2)
			       $X = \begin{pmatrix}
					       1 & 2  & -1 \\
					       3 & 4  & -2 \\
					       5 & -4 & 1
				       \end{pmatrix}^{-1}
				       \begin{pmatrix}
					       -1 & 0 \\
					       3  & 2 \\
					       0  & 1
				       \end{pmatrix}
				       \begin{pmatrix}
					       2 & 3 \\
					       1 & 2
				       \end{pmatrix}^{-1} = \begin{pmatrix}
					       -2            & 1 & 0            \\
					       -\frac{13}{2} & 3 & -\frac{1}{2} \\
					       -16           & 7 & -1
				       \end{pmatrix}
				       \begin{pmatrix}
					       -1 & 0 \\
					       3  & 2 \\
					       0  & 1
				       \end{pmatrix}
				       \begin{pmatrix}
					       2  & -3 \\
					       -1 & 2
				       \end{pmatrix}$
			       $= \begin{pmatrix}
					       5            & 2            \\
					       \frac{31}{2} & \frac{11}{2} \\
					       37           & 13
				       \end{pmatrix}
				       \begin{pmatrix}
					       2  & -3 \\
					       -1 & 2
				       \end{pmatrix} = \begin{pmatrix}
					       8            & -11           \\
					       \frac{51}{2} & -\frac{71}{2} \\
					       61           & -85
				       \end{pmatrix}$.
		 \end{enumerate}


	 \paragraph{} %3
		 由 $|M| = \begin{vmatrix}
				 0 & A \\
				 C & B
			 \end{vmatrix} = (-1)^{nm} |A| \cdot |C|$,
		 $\because A, C$ 可逆, 则 $|A| \neq 0, |C| \neq 0$, 故 $|M| \neq 0$, 则 $M$ 可逆.
		 设 $M^{-1} = \begin{pmatrix}
				 X & Y \\
				 Z & H
			 \end{pmatrix}$,
		 则 $M \cdot M^{-1} = \begin{pmatrix}
				 0 & A \\
				 C & B
			 \end{pmatrix}
			 \begin{pmatrix}
				 X & Y \\
				 Z & H
			 \end{pmatrix} = \begin{pmatrix}
				 AZ      & AH      \\
				 CX + BZ & CY + BH
			 \end{pmatrix} = \begin{pmatrix}
				 E_1 &     \\
				     & E_2
			 \end{pmatrix}$,
		 则 $\begin{cases}
				 AZ = E_1    \\
				 AH = 0      \\
				 CX + BZ = 0 \\
				 CY + BH = 0
			 \end{cases} \Rightarrow \begin{cases}
				 X = -C^{-1}BA^{-1} \\
				 Y = C^{-1}         \\
				 Z = A^{-1}         \\
				 H = 0
			 \end{cases}$
		 故 $M^{-1} = \begin{pmatrix}
				 -C^{-1}BA^{-1} & C^{-1} \\
				 A^{-1}         & 0
			 \end{pmatrix}$.


	 \paragraph{} %4
		 因为 $|A| = 0$, 故 $A$ 不可逆.


	 \paragraph{} %5
		 由 $AB + E = A^2 + B$, 得 $(A - E)B = A^2 - E = (A - E)(A + E)$,
		 $\because |A - E| = \begin{vmatrix}
				 0 & 2 \\
				 1 & 2
			 \end{vmatrix} \neq 0$, 故 $B = A + E = \begin{pmatrix}
				 2 & 2 \\
				 1 & 4
			 \end{pmatrix}$.


 \subsection{} %B


	 \paragraph{} %1
		 由 $A^2-4A-E=0 \Rightarrow A(A-4E)=E$
		 故 $A^{-1}=A-4E$
		 又由 $A^2-4A-E=0 \Rightarrow 4A^2+A-17A-\frac{17}{4}E=-\frac{E}{4}$
		 $\Rightarrow (4A+E)(17E-4A)=E$
		 故 $(4A+E)^{-1}=17E-4A$
		 也可由 $A^2=4A+E \Rightarrow (4A+E)^{-1}=(A^2)^{-1}=(A^{-1})^2=(A-4E)^2$


	 \paragraph{} %2
		 由 $A^2=A$ 得 $A^2+A-2A-2E=2E \Rightarrow (A+E)(A-2E)=-2E$
		 故 $(A+E)^{-1}=\frac{1}{2}(2E-A)$.


	 \paragraph{} %3
		 由 $A^{-1}BA=6A+BA$ 得 $(A^{-1}-E)BA=6A$
		 故 $B=6(A^{-1}-E)^{-1} = \begin{pmatrix}
				 6 &   &             \\
				   & 2 &             \\
				   &   & \frac{3}{2}
			 \end{pmatrix}$


	 \paragraph{} %4
		 由 $E^k-A^k=E$
		 得 $(E-A)(E^{k-1}+E^{k-2}A+E^{k-3}A^2+\cdots+A^{k-1})=E$
		 $\Rightarrow (E-A)^{-1}=E+A+A^2+\cdots+A^{k-1}$


	 \paragraph{} %5
		 设 $\begin{pmatrix}
				 A & B \\
				 B & A
			 \end{pmatrix}
			 \begin{pmatrix}
				 X & Y \\
				 Z & H
			 \end{pmatrix} = \begin{pmatrix}
				 E & O \\
				 O & E
			 \end{pmatrix}$
		 则 $\begin{cases}
				 AX+BZ=E \\
				 AY+BH=O \\
				 BX+AZ=O \\
				 BY+AH=E
			 \end{cases} \Rightarrow \begin{cases}
				 X=\frac{1}{2}[(A+B)^{-1}+(A-B)^{-1}] \\
				 Y=\frac{1}{2}[(A+B)^{-1}-(A-B)^{-1}] \\
				 Z=\frac{1}{2}[(A+B)^{-1}-(A-B)^{-1}] \\
				 H=\frac{1}{2}[(A+B)^{-1}+(A-B)^{-1}]
			 \end{cases}$
		 故 $\begin{pmatrix}
				 A & B \\
				 B & A
			 \end{pmatrix}^{-1} = \frac{1}{2}\begin{pmatrix}
				 (A+B)^{-1}+(A-B)^{-1} & (A+B)^{-1}-(A-B)^{-1} \\
				 (A+B)^{-1}-(A-B)^{-1} & (A+B)^{-1}+(A-B)^{-1}
			 \end{pmatrix}$


 \subsection{} %C


	 \paragraph{} %1
		 证明:设
		 $A = \begin{pmatrix}
				 a_{11} & 0      & \cdots & 0      \\
				 a_{21} & a_{22} & \cdots & 0      \\
				 \vdots & \vdots & \ddots & \vdots \\
				 a_{n1} & a_{n2} & \cdots & a_{nn}
			 \end{pmatrix}$,
		 $A^{-1} = \begin{pmatrix}
				 b_{11} & b_{12} & \cdots & b_{1n} \\
				 b_{21} & b_{22} & \cdots & b_{2n} \\
				 \vdots & \vdots & \ddots & \vdots \\
				 b_{n1} & b_{n2} & \cdots & b_{nn}
			 \end{pmatrix}$
		 由于 $|A| \neq 0$,可得 $a_{ii} \neq 0 (i=1,2,\ldots,n)$.
		 $\because AA^{-1}=E$,比较第1行的元素,可得
		 $\begin{cases}
				 a_{11}b_{11}=1 \\
				 a_{11}b_{12}=0 \\
				 a_{11}b_{13}=0 \\
				 \vdots         \\
				 a_{11}b_{1n}=0
			 \end{cases}$
		 从而可得 $b_{12}=0$, $b_{13}=0$, $\ldots$, $b_{1n}=0$.
		 同理可以比较 $AA^{-1}$ 和 $E$ 的其他行,得 $b_{ij}=0 (i<j)$.
		 可见 $A^{-1}$ 是下三角矩阵.


	 \paragraph{} %2
		 证明:假设 $AB-BA=E$,
		 则考虑主对角元素之和
		 \[\sum_{i=1}^n \left[\sum_{k=1}^n a_{ik}b_{ki} - \sum_{k=1}^n b_{ik}a_{ki}\right]\]
		 \[= \sum_{i=1}^n \sum_{k=1}^n a_{ik}b_{ki} - \sum_{i=1}^n \sum_{k=1}^n b_{ik}a_{ki}\]
		 \[= \sum_{k=1}^n \sum_{i=1}^n a_{ik}b_{ki} - \sum_{i=1}^n \sum_{k=1}^n b_{ik}a_{ki}\]
		 \[= 0 \neq n\text{,矛盾!故 } AB-BA \neq E.\]

\section{1.4}


 \subsection{} %A


	 \paragraph{} %1
		 \begin{enumerate}
			 \item %(1)
			       $\left(\begin{array}{ccc:c}
					       1 & -2 & 1 & 1 \\
					       2 & -1 & 5 & 0 \\
					       0 & 3  & 1 & 2
				       \end{array}\right) \to \left(\begin{array}{ccc:c}
					       1 &   &   & \frac{17}{3} \\
					         & 1 &   & \frac{9}{3}  \\
					         &   & 1 & -2
				       \end{array}\right)$,
			       故 $x = \begin{pmatrix}
					       \frac{17}{3} \\
					       \frac{9}{3}  \\
					       -2
				       \end{pmatrix}$.

			 \item %(2)
			       $\left(\begin{array}{ccc:c}
					       1 & -1 & 1 & 1 \\
					       3 & 0  & 4 & 5 \\
					       1 & 2  & 2 & 0
				       \end{array}\right) \to \left(\begin{array}{ccc:c}
					       1 &   & \frac{4}{3} &   \\
					         & 1 & \frac{1}{3} &   \\
					         &   &             & 1
				       \end{array}\right)$ 无解.

			 \item %(3)
			       $\left(\begin{array}{cccc}
					       3 & 4  & -5  & 7  \\
					       2 & -3 & 3   & -2 \\
					       4 & 11 & -13 & 16 \\
					       7 & -2 & 1   & 3
				       \end{array}\right) \to \left(\begin{array}{cccc}
					       1 &   & -\frac{3}{17}  & \frac{13}{17} \\
					         & 1 & -\frac{19}{17} & \frac{20}{17} \\
					         &   & 0              & 0             \\
					         &   &                & 0
				       \end{array}\right)$ $\Rightarrow \begin{cases}
					       x_1 = \frac{3}{17}x_3 - \frac{13}{17}x_4 \\
					       x_2 = \frac{19}{17}x_3 - \frac{20}{17}x_4
				       \end{cases}$
			       $x_3, x_4$ 为自由未知量,令 $\begin{pmatrix}
					       x_3 \\
					       x_4
				       \end{pmatrix}$ 取 $\begin{pmatrix}
					       1 \\
					       0
				       \end{pmatrix}, \begin{pmatrix}
					       0 \\
					       1
				       \end{pmatrix}$ $\Rightarrow \eta_1 = \begin{pmatrix}
					       \frac{3}{17}  \\
					       \frac{19}{17} \\
					       1             \\
					       0
				       \end{pmatrix}, \eta_2 = \begin{pmatrix}
					       -\frac{13}{17} \\
					       -\frac{20}{17} \\
					       0              \\
					       1
				       \end{pmatrix}$
			       则 $x = p\begin{pmatrix}
					       \frac{3}{17}  \\
					       \frac{19}{17} \\
					       1             \\
					       0
				       \end{pmatrix} + q\begin{pmatrix}
					       -\frac{13}{17} \\
					       -\frac{20}{17} \\
					       0              \\
					       1
				       \end{pmatrix}$.

			 \item %(4)
			       $\left(\begin{array}{cccc:c}
					       3 & -2 & 1  & -3 & 4  \\
					       2 & 1  & -1 & 1  & 1  \\
					       1 & 4  & -3 & 5  & -2
				       \end{array}\right) \to \left(\begin{array}{cccc:c}
					       1 &   & -\frac{1}{7} & -\frac{1}{7} & \frac{6}{7} \\
					         & 1 & -\frac{5}{7} & \frac{9}{7}  & \frac{5}{7} \\
					         &   & 0            & 0            & 0
				       \end{array}\right)$
			       令自由未知量 $x_3, x_4$ 为 0,得特解 $\eta^* = \begin{pmatrix}
					       \frac{6}{7}  \\
					       -\frac{5}{7} \\
					       0            \\
					       0
				       \end{pmatrix}$,
			       令 $\begin{pmatrix}
					       x_3 \\
					       x_4
				       \end{pmatrix}$ 为 $\begin{pmatrix}
					       1 \\
					       0
				       \end{pmatrix}, \begin{pmatrix}
					       0 \\
					       1
				       \end{pmatrix}$ 可得基础解系
			       $\eta_1 = \begin{pmatrix}
					       \frac{1}{7} \\
					       \frac{5}{7} \\
					       1           \\
					       0
				       \end{pmatrix}, \eta_2 = \begin{pmatrix}
					       \frac{1}{7}  \\
					       -\frac{9}{7} \\
					       0            \\
					       1
				       \end{pmatrix}$,
			       则 $x = p\begin{pmatrix}
					       \frac{1}{7} \\
					       \frac{5}{7} \\
					       1           \\
					       0
				       \end{pmatrix} + q\begin{pmatrix}
					       \frac{1}{7}  \\
					       -\frac{9}{7} \\
					       0            \\
					       1
				       \end{pmatrix} + \begin{pmatrix}
					       \frac{6}{7}  \\
					       -\frac{5}{7} \\
					       0            \\
					       0
				       \end{pmatrix}$.
		 \end{enumerate}


	 \paragraph{} %2
		 \begin{enumerate}
			 \item %(1)
			       $\begin{pmatrix}
					       4  & -3 \\
					       -1 & 2
				       \end{pmatrix}^{-1} = \dfrac{1}{\begin{vmatrix}
						       4  & -3 \\
						       -1 & 2
					       \end{vmatrix}} \cdot \begin{pmatrix}
					       4  & -3 \\
					       -1 & 2
				       \end{pmatrix}^{*} = \dfrac{1}{5} \begin{pmatrix}
					       2 & 3 \\
					       1 & 4
				       \end{pmatrix}$.

			 \item %(2)
			       $\begin{pmatrix}
					       1 & -1 & -1 \\
					       0 & 1  & -1 \\
					       0 & 0  & 1
				       \end{pmatrix}^{-1} = \begin{pmatrix}
					       1 & 1 & 2 \\
					       0 & 1 & 1 \\
					       0 & 0 & 1
				       \end{pmatrix}$.

			 \item %(3)
			       $\begin{pmatrix}
					       1 & 2 & 3  & 4  \\
					       2 & 3 & 1  & 2  \\
					       1 & 1 & 1  & -1 \\
					       1 & 0 & -2 & -6
				       \end{pmatrix}^{-1} = \begin{pmatrix}
					       22  & -6 & -26 & 17  \\
					       -17 & 5  & 20  & -13 \\
					       -1  & 0  & 2   & -1  \\
					       4   & -1 & -5  & 3
				       \end{pmatrix}$.

			 \item %(4)
			       $\begin{pmatrix}
					       0      & 0      & \cdots & 0       & a_n    \\
					       a_1    & 0      & \cdots & 0       & 0      \\
					       0      & a_2    & \cdots & 0       & 0      \\
					       \vdots & \vdots & \ddots & \vdots  & \vdots \\
					       0      & 0      & \cdots & a_{n-1} & 0
				       \end{pmatrix}^{-1} = \begin{pmatrix}
					       0             & \frac{1}{a_1} & 0             & \cdots & 0                 \\
					       0             & 0             & \frac{1}{a_2} & \cdots & 0                 \\
					       \vdots        & \vdots        & \vdots        & \ddots & \vdots            \\
					       0             & 0             & 0             & \cdots & \frac{1}{a_{n-1}} \\
					       \frac{1}{a_n} & 0             & 0             & \cdots & 0
				       \end{pmatrix}$.

			 \item %(5)
			       由 $\begin{pmatrix}
					       0 & A \\
					       B & 0
				       \end{pmatrix}^{-1} = \begin{pmatrix}
					       0      & B^{-1} \\
					       A^{-1} & 0
				       \end{pmatrix}$, 多次使用该公式有
			       $\begin{pmatrix}
					       0      & 0       & \cdots & 0      & b_1    \\
					       0      & 0       & \cdots & b_2    & 0      \\
					       \vdots & \vdots  & \ddots & \vdots & \vdots \\
					       0      & b_{n-1} & \cdots & 0      & 0      \\
					       b_n    & 0       & \cdots & 0      & 0
				       \end{pmatrix}^{-1} = \begin{pmatrix}
					       0             & 0             & \cdots & 0                 & \frac{1}{b_n} \\
					       0             & 0             & \cdots & \frac{1}{b_{n-1}} & 0             \\
					       \vdots        & \vdots        & \ddots & \vdots            & \vdots        \\
					       0             & \frac{1}{b_2} & \cdots & 0                 & 0             \\
					       \frac{1}{b_1} & 0             & \cdots & 0                 & 0
				       \end{pmatrix}$.
		 \end{enumerate}


	 \paragraph{} %3
		 \begin{enumerate}
			 \item %(1)
			       由 $\begin{pmatrix}
					       3 & 5 \\
					       5 & 9
				       \end{pmatrix} X = \begin{pmatrix}
					       1 & 2 \\
					       3 & 4
				       \end{pmatrix}$ $\Rightarrow X = \begin{pmatrix}
					       3 & 5 \\
					       5 & 9
				       \end{pmatrix}^{-1}
				       \begin{pmatrix}
					       1 & 2 \\
					       3 & 4
				       \end{pmatrix} = \begin{pmatrix}
					       \frac{9}{2}  & -\frac{5}{2} \\
					       -\frac{5}{2} & \frac{3}{2}
				       \end{pmatrix}
				       \begin{pmatrix}
					       1 & 2 \\
					       3 & 4
				       \end{pmatrix} = \begin{pmatrix}
					       -3 & -1 \\
					       2  & 1
				       \end{pmatrix}$.

			 \item %(2)
			       $X = \begin{pmatrix}
					       3 & 5 \\
					       5 & 9
				       \end{pmatrix}
				       \begin{pmatrix}
					       1 & 2 \\
					       3 & 4
				       \end{pmatrix}^{-1} = \begin{pmatrix}
					       3 & 5 \\
					       5 & 9
				       \end{pmatrix}
				       \begin{pmatrix}
					       -2          & 1            \\
					       \frac{3}{2} & -\frac{1}{2}
				       \end{pmatrix} = \begin{pmatrix}
					       \frac{3}{2} & \frac{1}{2} \\
					       \frac{7}{2} & \frac{1}{2}
				       \end{pmatrix}$.

			 \item %(3)
			       $X = \begin{pmatrix}
					       1 & 2  & 3  \\
					       3 & 2  & -4 \\
					       2 & -1 & 0
				       \end{pmatrix}^{-1}
				       \begin{pmatrix}
					       1 & 3  \\
					       0 & -2 \\
					       2 & 1
				       \end{pmatrix}$ 则由于
			       $\begin{pmatrix}
					       1 & 2  & 3  & 1 & 3  \\
					       3 & 2  & -4 & 0 & -2 \\
					       2 & -1 & 0  & 2 & 1
				       \end{pmatrix} \to \begin{pmatrix}
					       1 & 2             & 3             & 13           & 11 \\
					       1 & \frac{13}{4}  & \frac{3}{4}   & \frac{11}{4}      \\
					       1 & \frac{15}{41} & \frac{35}{41}
				       \end{pmatrix} \to \begin{pmatrix}
					       1 &   &   & \frac{32}{41}  & \frac{20}{41} \\
					         & 1 &   & -\frac{18}{41} & -\frac{1}{41} \\
					         &   & 1 & \frac{15}{41}  & \frac{35}{41}
				       \end{pmatrix}$,
			       故 $X = \begin{pmatrix}
					       \frac{32}{41}  & \frac{20}{41} \\
					       -\frac{18}{41} & -\frac{1}{41} \\
					       \frac{15}{41}  & \frac{35}{41}
				       \end{pmatrix}$.
		 \end{enumerate}


	 \paragraph{} %4
		 \begin{enumerate}
			 \item %(1)
			       $\begin{pmatrix}
					       x_1 \\
					       x_2 \\
					       x_3
				       \end{pmatrix} = \begin{pmatrix}
					       1 & -1 & -1 \\
					       2 & -1 & -3 \\
					       3 & 2  & -5
				       \end{pmatrix}^{-1}
				       \begin{pmatrix}
					       2 \\
					       1 \\
					       0
				       \end{pmatrix} = \begin{pmatrix}
					       \frac{11}{3} & -\frac{7}{3} & \frac{2}{3} \\
					       \frac{1}{3}  & -\frac{2}{3} & \frac{1}{3} \\
					       \frac{7}{3}  & -\frac{5}{3} & \frac{1}{3}
				       \end{pmatrix}
				       \begin{pmatrix}
					       2 \\
					       1 \\
					       0
				       \end{pmatrix} = \begin{pmatrix}
					       5 \\
					       0 \\
					       3
				       \end{pmatrix}$.

			 \item %(2)
			       $\begin{pmatrix}
					       x_1 \\
					       x_2 \\
					       x_3
				       \end{pmatrix} = \begin{pmatrix}
					       1 & -2 & 1 \\
					       1 & -1 & 4 \\
					       0 & 1  & 1
				       \end{pmatrix}^{-1}
				       \begin{pmatrix}
					       1 \\
					       0 \\
					       3
				       \end{pmatrix} = \begin{pmatrix}
					       \frac{5}{2}  & -\frac{3}{2} & \frac{7}{2}  \\
					       \frac{1}{2}  & -\frac{1}{2} & \frac{3}{2}  \\
					       -\frac{1}{2} & \frac{1}{2}  & -\frac{1}{2}
				       \end{pmatrix}
				       \begin{pmatrix}
					       1 \\
					       0 \\
					       3
				       \end{pmatrix} = \begin{pmatrix}
					       13 \\
					       5  \\
					       -2
				       \end{pmatrix}$.
		 \end{enumerate}


	 \paragraph{} %5
		 \begin{enumerate}
			 \item %(1)
			       $\begin{pmatrix}
					       1 & -2 & 1 \\
					       2 & -4 & 2
				       \end{pmatrix} \to \begin{pmatrix}
					       1 & -2 & 1 \\
					       0 & 0  & 0
				       \end{pmatrix} \to \begin{pmatrix}
					       1 & 0 & 0 \\
					       0 & 0 & 0
				       \end{pmatrix}$.

			 \item %(2)
			       $\begin{pmatrix}
					       1 & 2  & -1 \\
					       3 & 4  & 5  \\
					       6 & -3 & 2  \\
					       0 & -1 & 1
				       \end{pmatrix} \to \begin{pmatrix}
					       1 & 2   & -1 \\
					       0 & -2  & 8  \\
					       0 & -15 & 8  \\
					       0 & -1  & 1
				       \end{pmatrix} \to \begin{pmatrix}
					       1 & 2 & -1  \\
					       0 & 1 & -4  \\
					       0 & 0 & -52 \\
					       0 & 0 & 0
				       \end{pmatrix} \to \begin{pmatrix}
					       1 & 0 & 0 \\
					       0 & 1 & 0 \\
					       0 & 0 & 1 \\
					       0 & 0 & 0
				       \end{pmatrix}$.
		 \end{enumerate}


	 \paragraph{} %6
		 由 $PAQ = \begin{pmatrix}
				 1 & 0 & 0 & 0 \\
				 0 & 1 & 0 & 0 \\
				 0 & 0 & 0 & 0
			 \end{pmatrix}$, 可取 $P = E_3(3,2(2))E_3(3,1(1))E_3(2,1(-2))$
		 取 $Q = \begin{pmatrix}
				 1 & 2 & -1 & 2 \\
				 0 & 0 & 1  & 1 \\
				 0 & 0 & 1  & 0 \\
				 0 & 1 & 0  & 0
			 \end{pmatrix} = \begin{pmatrix}
				 1  & 0 & 0 \\
				 -2 & 1 & 0 \\
				 -3 & 2 & 1
			 \end{pmatrix}$.


	 \paragraph{} %7
		 证明:由题设 $P, Q$ 是可逆矩阵,故 $P, Q$ 是若干个初等矩阵的乘积。用 $P$ 左乘 $A$ 或用 $Q$ 右乘 $A$,即对 $A$ 作若干次初等行变换或初等列变换,初等变换不改变矩阵的秩,故 $r(A) = r(PA) = r(AQ) = r(PAQ)$.


	 \paragraph{} %8
		 \begin{enumerate}
			 \item %(1)
			       由于 $\begin{pmatrix}
					       0 & A & | & E_n       \\
					       B & 0 & | &     & E_m
				       \end{pmatrix} \xrightarrow{r_1 \leftrightarrow r_2} \begin{pmatrix}
					       B & 0 & | & 0   & E_m \\
					       0 & A & | & E_n & 0
				       \end{pmatrix}$
			       $\xrightarrow[B^{-1}r_1]{A^{-1}r_2} \begin{pmatrix}
					       E_m & 0   & | & 0      & B^{-1} \\
					       0   & E_n & | & A^{-1} & 0
				       \end{pmatrix}$
			       故 $\begin{pmatrix}
					       0 & A \\
					       B & 0
				       \end{pmatrix}^{-1} = \begin{pmatrix}
					       0      & B^{-1} \\
					       A^{-1} & 0
				       \end{pmatrix}$.

			 \item %(2)
			       由于 $\begin{pmatrix}
					       0 & A & | & E_n       \\
					       B & C & | &     & E_m
				       \end{pmatrix} \xrightarrow{r_1 \leftrightarrow r_2} \begin{pmatrix}
					       B & C & | & 0 & E \\
					       0 & A & | & E & 0
				       \end{pmatrix}$
			       $\xrightarrow{A^{-1}r_2} \begin{pmatrix}
					       B & C & | & 0      & E \\
					       0 & E & | & A^{-1} & 0
				       \end{pmatrix}$
			       $\xrightarrow{r_1 - Cr_2} \begin{pmatrix}
					       B & 0 & | & CA^{-1} & E \\
					       0 & E & | & A^{-1}  & 0
				       \end{pmatrix}$
			       $\xrightarrow{B^{-1}r_1} \begin{pmatrix}
					       E & 0 & | & B^{-1}CA^{-1} & B^{-1} \\
					       0 & E & | & A^{-1}        & 0
				       \end{pmatrix}$
			       故 $\begin{pmatrix}
					       0 & A \\
					       B & C
				       \end{pmatrix}^{-1} = \begin{pmatrix}
					       B^{-1}CA^{-1} & B^{-1} \\
					       A^{-1}        & 0
				       \end{pmatrix}$.
		 \end{enumerate}


 \subsection{} %B


	 \paragraph{} %1
		 由于 $\left(\begin{array}{ccc:c}
					 1       & \lambda & 1       & 1         \\
					 1       & 1       & \lambda & \lambda   \\
					 \lambda & 1       & 1       & \lambda^2
				 \end{array}\right) \to \left(\begin{array}{ccc:c}
					 1 & \lambda     & 1         & 1                 \\
					 0 & 1-\lambda   & \lambda-1 & \lambda-1         \\
					 0 & 1-\lambda^2 & 1-\lambda & \lambda^2-\lambda
				 \end{array}\right)$
		 故当 $\lambda=1$ 时有无穷解.
		 当 $\lambda \neq 1$ 时 $\to \left(\begin{array}{ccc:c}
					 1 & \lambda & 1         & 1  \\
					 0 & 1       & -1        & -1 \\
					 0 & 0       & \lambda+2 & 1
				 \end{array}\right)$
		 则当 $\lambda \neq 2$ 且 $\lambda \neq 1$ 时有唯一解
		 当 $\lambda=2$ 时无解.
		 $\lambda=1$ 时 $\to \left(\begin{array}{ccc:c}
					 1 & 1 & 1 & 1 \\
					 0 & 0 & 0 & 0 \\
					 0 & 0 & 0 & 0
				 \end{array}\right)$
		 取 $\begin{pmatrix}
				 x_2 \\
				 x_3
			 \end{pmatrix}$ 为 $\begin{pmatrix}
				 0 \\
				 0
			 \end{pmatrix}$ 可得特解 $\eta^* = \begin{pmatrix}
				 1 \\
				 0 \\
				 0
			 \end{pmatrix}$
		 取 $\begin{pmatrix}
				 x_2 \\
				 x_3
			 \end{pmatrix}$ 为 $\begin{pmatrix}
				 1 \\
				 0
			 \end{pmatrix}, \begin{pmatrix}
				 0 \\
				 1
			 \end{pmatrix}$ 可得基础解系 $\eta_1 = \begin{pmatrix}
				 -1 \\
				 1  \\
				 0
			 \end{pmatrix}, \eta_2 = \begin{pmatrix}
				 -1 \\
				 0  \\
				 1
			 \end{pmatrix}$
		 故 $x = p\begin{pmatrix}
				 -1 \\
				 1  \\
				 0
			 \end{pmatrix} + q\begin{pmatrix}
				 -1 \\
				 0  \\
				 1
			 \end{pmatrix} + \begin{pmatrix}
				 1 \\
				 0 \\
				 0
			 \end{pmatrix}$.


	 \paragraph{} %2
		 由 $XA = A + 3X$ $\Rightarrow X = A(A-3E)^{-1}$
		 $= \begin{pmatrix}
				 4 & 2 & 3 \\
				 1 & 1 & 0 \\
				 1 & 2 & 3
			 \end{pmatrix}
			 \begin{pmatrix}
				 1 & 2  & 3 \\
				 1 & -2 & 0 \\
				 1 & 2  & 0
			 \end{pmatrix}^{-1}$
		 $= \begin{pmatrix}
				 4 & 2 & 3 \\
				 1 & 1 & 0 \\
				 1 & 2 & 3
			 \end{pmatrix}
			 \begin{pmatrix}
				 0           & \frac{1}{2}  & \frac{1}{2}  \\
				 0           & -\frac{1}{4} & \frac{1}{4}  \\
				 \frac{1}{3} & 0            & -\frac{1}{3}
			 \end{pmatrix}$
		 $= \begin{pmatrix}
				 1 & \frac{3}{2} & \frac{3}{2} \\
				 0 & \frac{1}{4} & \frac{1}{4} \\
				 1 & 0           & 0
			 \end{pmatrix}$.


	 \paragraph{} %3
		 $A = PBP^{-1}$, 则 $A^5 = PBP^{-1}PBP^{-1}PBP^{-1}PBP^{-1}PBP^{-1} = PB^5P^{-1}$
		 $\because B^2 = \begin{pmatrix}
				 1 & 0 & 0  \\
				 0 & 0 & 0  \\
				 0 & 0 & -1
			 \end{pmatrix}
			 \begin{pmatrix}
				 1 & 0 & 0  \\
				 0 & 0 & 0  \\
				 0 & 0 & -1
			 \end{pmatrix} = \begin{pmatrix}
				 1 & 0 & 0 \\
				 0 & 0 & 0 \\
				 0 & 0 & 1
			 \end{pmatrix} = B$
		 $P^{-1} = \begin{pmatrix}
				 1  & 0  & 0 \\
				 2  & -1 & 0 \\
				 -4 & 1  & 1
			 \end{pmatrix}$
		 则 $A = PBP^{-1} = \begin{pmatrix}
				 1  & 0 & 0 \\
				 2  & 1 & 0 \\
				 -4 & 1 & 1
			 \end{pmatrix}
			 \begin{pmatrix}
				 1 & 0 & 0  \\
				 0 & 1 & 0  \\
				 0 & 0 & -1
			 \end{pmatrix}
			 \begin{pmatrix}
				 1  & 0  & 0 \\
				 2  & -1 & 0 \\
				 -4 & 1  & 1
			 \end{pmatrix} = \begin{pmatrix}
				 1 & 0  & 0  \\
				 2 & 0  & 0  \\
				 6 & -1 & -1
			 \end{pmatrix}$
		 $A^5 = PB^5P^{-1} = PBP^{-1} = A = \begin{pmatrix}
				 1 & 0  & 0  \\
				 2 & 0  & 0  \\
				 6 & -1 & -1
			 \end{pmatrix}$.


	 \paragraph{} %4
		 依题意,考虑 $\begin{vmatrix}
				 1 & x \\
				 y & 1
			 \end{vmatrix} = 1-xy \neq 0$ 知 $x,y \in \mathbb{R}$, $y \neq 0$.


	 \paragraph{} %5
		 \begin{enumerate}
			 \item %(1)
			       由 $\begin{pmatrix}
					       1      & a      & a^2    & \cdots & a^{n-1} & 1      \\
					       1      & a      & a^2    & \cdots & a^{n-2} & 1      \\
					       \vdots & \vdots & \vdots & \ddots & \vdots  & \vdots \\
					       1      & a      & a^2    & \cdots & a^{n-2} & 1
				       \end{pmatrix} \to \begin{pmatrix}
					       1 &   &        &   &  & 1 & -a               \\
					         & 1 &        &   &  &   & 1  & -a          \\
					         &   & \ddots &   &  &   &    & \ddots & -a \\
					         &   &        & 1 &  &   &    &        & 1
				       \end{pmatrix}$
			       知 $\begin{pmatrix}
					       1      & a      & a^2    & \cdots & a^{n-1} \\
					       1      & a      & a^2    & \cdots & a^{n-2} \\
					       \vdots & \vdots & \vdots & \ddots & \vdots  \\
					       1      & a      & a^2    & \cdots & a^{n-2}
				       \end{pmatrix}^{-1} = \begin{pmatrix}
					       1 & -a &        &        &    \\
					         & 1  & -a     &        &    \\
					         &    & \ddots & \ddots &    \\
					         &    &        & 1      & -a \\
					         &    &        &        & 1
				       \end{pmatrix}$.

			 \item %(2)
			       由 $\begin{pmatrix}
					       0 &   &        &        & 1 \\
					       1 & 0 &        &        &   \\
					         & 1 & 0      &        &   \\
					         &   & \ddots & \ddots &   \\
					         &   &        & 1      & 0
				       \end{pmatrix} \to \begin{pmatrix}
					       n-1    & n-1    & n-1    & \cdots & n-1    \\
					       1      & 0      & 0      & \cdots & 0      \\
					       1      & 1      & 0      & \cdots & 0      \\
					       \vdots & \vdots & \vdots & \ddots & \vdots \\
					       1      & 1      & 1      & \cdots & 0
				       \end{pmatrix}$
			       $\to \begin{pmatrix}
					       1      & 0      & 0      & \cdots & 0      & \frac{1}{n-1}  & \frac{1}{n-1}  & \cdots & \frac{1}{n-1}   \\
					       0      & 1      & 0      & \cdots & 0      & -\frac{1}{n-1} & \frac{1}{n-1}  & \cdots & \frac{1}{n-1}   \\
					       0      & 0      & 1      & \cdots & 0      & -\frac{1}{n-1} & -\frac{1}{n-1} & \cdots & \frac{1}{n-1}   \\
					       \vdots & \vdots & \vdots & \ddots & \vdots & \vdots         & \vdots         & \ddots & \vdots          \\
					       0      & 0      & 0      & \cdots & 1      & -\frac{1}{n-1} & -\frac{1}{n-1} & \cdots & \frac{n-2}{n-1}
				       \end{pmatrix}$
			       $\to \begin{pmatrix}
					       1 &   &        &   &  & -\frac{n-2}{n-1} & \frac{1}{n-1}    & \cdots        & \frac{1}{n-1}    \\
					         & 1 &        &   &  & \frac{1}{n-1}    & -\frac{n-2}{n-1} & \cdots        & \frac{1}{n-1}    \\
					         &   & \ddots &   &  & \vdots           & \vdots           & \ddots        & \vdots           \\
					         &   &        & 1 &  & \frac{1}{n-1}    & \cdots           & \frac{1}{n-1} & -\frac{n-2}{n-1}
				       \end{pmatrix}$
			       故 $\begin{pmatrix}
					       0 & 1 &        &   &   \\
					       1 & 0 &        &   &   \\
					         & 1 & \ddots &   &   \\
					         &   & \ddots & 0 &   \\
					         &   &        & 1 & 0
				       \end{pmatrix}^{-1} = \begin{pmatrix}
					       -\frac{n-2}{n-1} & \frac{1}{n-1}    & \cdots        & \frac{1}{n-1}    \\
					       \frac{1}{n-1}    & -\frac{n-2}{n-1} & \cdots        & \frac{1}{n-1}    \\
					       \vdots           & \vdots           & \ddots        & \vdots           \\
					       \frac{1}{n-1}    & \cdots           & \frac{1}{n-1} & -\frac{n-2}{n-1}
				       \end{pmatrix}$.
		 \end{enumerate}


	 \paragraph{} %6
		 依题意 $r(A)=2 < n=3$
		 则 $\begin{vmatrix}
				 1 & 1 & 0   \\
				 0 & a & 0   \\
				 0 & 0 & b+c
			 \end{vmatrix} = 0$ $\Rightarrow a=0$ 或 $b+c=0$
		 ① 当 $a=0$, 则由 $r(A)=2$ 知 $b+c \neq 0$
		 ② 当 $b+c=0$, 则由 $r(A)=2$ 知 $a \neq 0$
		 $A$ 的相抵标准形为 $I_2 = \begin{pmatrix}
				 1 & 0 & 0 \\
				 0 & 1 & 0 \\
				 0 & 0 & 0
			 \end{pmatrix}$.


	 \paragraph{} %7
		 证明:充分性:$\because P, Q$ 可逆,$\therefore P, Q$ 可分解为若干个基本初等矩阵的积
		 即 $A$ 可以经过若干次初等变换得到 $B$,
		 $\therefore A \sim B$
		 必要性:$\because A \sim B$, $\therefore A$ 经过若干次初等变换可以得到 $B$
		 即 $PAQ = B$ ($P, Q$ 可逆)


	 \paragraph{} %8
		 $A = \begin{pmatrix}
				 a      & 1      & \cdots & 1      \\
				 1      & a      & \cdots & 1      \\
				 \vdots & \vdots & \ddots & \vdots \\
				 1      & 1      & \cdots & a
			 \end{pmatrix}_{n \times n}$
		 则 $a=1$ 时 $\begin{pmatrix}
				 1      & 1      & \cdots & 1      \\
				 1      & 1      & \cdots & 1      \\
				 \vdots & \vdots & \ddots & \vdots \\
				 1      & 1      & \cdots & 1
			 \end{pmatrix} \to \begin{pmatrix}
				 1      & 0      & \cdots & 0      \\
				 0      & 0      & \cdots & 0      \\
				 \vdots & \vdots & \ddots & \vdots \\
				 0      & 0      & \cdots & 0
			 \end{pmatrix}$
		 此时 $I = \begin{pmatrix}
				 1 & 0       \\
				 0 & O_{n-1}
			 \end{pmatrix}$
		 当 $a=1-n$ 时, $\begin{pmatrix}
				 1-n    & 1      & \cdots & 1      \\
				 1      & 1-n    & \cdots & 1      \\
				 \vdots & \vdots & \ddots & \vdots \\
				 1      & 1      & \cdots & 1-n
			 \end{pmatrix} \to \begin{pmatrix}
				 1-n    & 1      & \cdots & 1      \\
				 1      & 1-n    & \cdots & 1      \\
				 \vdots & \vdots & \ddots & \vdots \\
				 0      & 0      & \cdots & 0
			 \end{pmatrix} \to \begin{pmatrix}
				 -n & -n & \cdots & -n     & 1      \\
				    & -n & \cdots & -n     & 1      \\
				    &    & \ddots & \vdots & \vdots \\
				 0  & 0  & \cdots & 0      & 0
			 \end{pmatrix}$
		 $\to \begin{pmatrix}
				 1 &   &        & 0 \\
				   & 1 &        & 0 \\
				   &   & \ddots & 0 \\
				 0 & 0 & \cdots & 0
			 \end{pmatrix}$ 故 $I = \begin{pmatrix}
				 E_{n-1} & 0 \\
				 0       & 0
			 \end{pmatrix}$
		 当 $a \neq 1, 1-n$ 时则 $\begin{pmatrix}
				 a-1 &     &        &   \\
				     & a-1 &        &   \\
				     &     & \ddots &   \\
				     &     &        & a
			 \end{pmatrix} \to \begin{pmatrix}
				 a &   &        &   \\
				   & a &        &   \\
				   &   & \ddots &   \\
				   &   &        & 1
			 \end{pmatrix} \to \begin{pmatrix}
				 1 &   &        &   \\
				   & 1 &        &   \\
				   &   & \ddots &   \\
				   &   &        & 1
			 \end{pmatrix} = E_n$
		 故此时 $I = E_n$. 综上,
		 \[r(A) = \begin{cases}
				 1   & a=1           \\
				 n-1 & a=1-n         \\
				 n   & a \neq 1, 1-n
			 \end{cases}\]


	 \paragraph{} %9
		 由 $\begin{pmatrix}
				 A_{n-1}  & \beta  \\
				 \alpha^T & a_{nn}
			 \end{pmatrix}$ 进行初等行变换
		 \[\begin{pmatrix}
				 A_{n-1}  & \beta  & | & E & 0 \\
				 \alpha^T & a_{nn} & | & 0 & E
			 \end{pmatrix} \to \begin{pmatrix}
				 E        & A_{n-1}^{-1}\beta & | & A_{n-1}^{-1} & 0 \\
				 \alpha^T & a_{nn}            & | & 0            & E
			 \end{pmatrix}\]
		 \[\to \begin{pmatrix}
				 E & A_{n-1}^{-1}\beta                 & | & A_{n-1}^{-1}           & 0 \\
				 0 & a_{nn}-\alpha^T A_{n-1}^{-1}\beta & | & -\alpha^T A_{n-1}^{-1} & E
			 \end{pmatrix}\]
		 \[\to \begin{pmatrix}
				 E & 0 & | & A_{n-1}^{-1}(E_{n-1}+\beta t^{-1}\alpha^T A_{n-1}^{-1}) & -A_{n-1}^{-1}\beta t^{-1} \\
				 0 & E & | & -t^{-1}\alpha^T A_{n-1}^{-1}                            & t^{-1}
			 \end{pmatrix}\]
		 其中 $t = a_{nn} - \alpha^T A_{n-1}^{-1}\beta$
		 令 $A_{n-1} = \begin{pmatrix}
				 1 & 1  & -1 \\
				 0 & 2  & 2  \\
				 1 & -1 & 0
			 \end{pmatrix}$, $\beta = \begin{pmatrix}
				 -3 \\
				 1  \\
				 1
			 \end{pmatrix}$, $\alpha^T = (2 \ 3 \ 3)$, $a_{nn}=2$
		 则 $A^{-1} = \frac{1}{5}\begin{pmatrix}
				 1  & -1  & 2   & 1   \\
				 -3 & -12 & -11 & 7   \\
				 5  & 20  & 15  & -10 \\
				 -4 & -11 & -8  & 6
			 \end{pmatrix}$.


	 \paragraph{} %10
		 证明:假设 $A$ 为满秩矩阵,则 $|A| \neq 0$, 这与 $A$ 不可逆矛盾。
		 故 $A$ 为降秩矩阵。


 \subsection{} %C


	 \paragraph{} %1
		 \begin{enumerate}
			 \item %(1)
			       考虑第1行第1列元素,则 $q_1 = b_1$
			       考虑第1行第2列元素,则 $q_1r_1 = c_1 \Rightarrow r_1 = \frac{c_1}{q_1}$
			       考虑第$i$行第$i-1$列元素 ($i=2,\dots,n$),则 $p_i = a_i$
			       考虑第$i$行第$i$列元素 ($i=2,\dots,n$),则 $p_ir_{i-1}+q_i = b_i \Rightarrow q_i = b_i - p_ir_{i-1}$
			       考虑第$i$行第$i+1$列元素 ($i=2,\dots,n-1$),则 $q_ir_i = c_i \Rightarrow r_i = \frac{c_i}{q_i}$

			 \item %(2)
			       由 $\left(\begin{array}{ccc:c}
					       1 & 2 &   & 6 \\
					       2 & 1 & 1 & 8 \\
					         & 1 & 2 & 8 \\
					       1 & 2 &   & 6
				       \end{array}\right) \to \left(\begin{array}{ccc:c}
					       1 &   &   & 2 \\
					         & 1 &   & 2 \\
					         &   & 1 & 2 \\
					         &   &   & 2
				       \end{array}\right)$
			       则 $x = \begin{pmatrix}
					       2 \\
					       2 \\
					       2 \\
					       2
				       \end{pmatrix}$.
		 \end{enumerate}


	 \paragraph{} %2
		 $f(A) = (A-E)^n = \begin{pmatrix}
				 1 & 1 &        &        \\
				   & 1 & 1      &        \\
				   &   & \ddots & \ddots \\
				   &   &        & 1
			 \end{pmatrix}^n = \left[\begin{pmatrix}
					 0 & 1 &        &        \\
					   & 0 & 1      &        \\
					   &   & \ddots & \ddots \\
					   &   &        & 0
				 \end{pmatrix} + E\right]^n$
		 记 $B = \begin{pmatrix}
				 0 & 1 &        &        \\
				   & 0 & 1      &        \\
				   &   & \ddots & \ddots \\
				   &   &        & 0
			 \end{pmatrix}$ 则 $B^2 = \begin{pmatrix}
				 0 & 0 & 1      &        &        \\
				   & 0 & 0      & 1      &        \\
				   &   & \ddots & \ddots & \ddots \\
				   &   &        & 0      & 0      \\
				   &   &        &        & 0
			 \end{pmatrix}$, $B^3 = \begin{pmatrix}
				 0 & 0 & 0      & 1      &        &        \\
				   & 0 & 0      & 0      & 1      &        \\
				   &   & \ddots & \ddots & \ddots & \ddots \\
				   &   &        & 0      & 0      & 0      \\
				   &   &        &        & 0      & 0      \\
				   &   &        &        &        & 0
			 \end{pmatrix}$
		 $B^n = O$ ($n \geq 4$), 则
		 $f(A) = (B+E)^n = B^n + C_n^1B^{n-1} + C_n^2B^{n-2} + \cdots + C_n^{n-2}B^2 + C_n^{n-1}B + E$
		 $= E + C_n^1\begin{pmatrix}
				 0 & 1 & 0 & 0 \\
				   & 0 & 1 & 0 \\
				   &   & 0 & 1 \\
				   &   &   & 0
			 \end{pmatrix} + C_n^2\begin{pmatrix}
				 0 & 0 & 1 & 0 \\
				   & 0 & 0 & 1 \\
				   &   & 0 & 0 \\
				   &   &   & 0
			 \end{pmatrix} + C_n^3\begin{pmatrix}
				 0 & 0 & 0 & 1 \\
				   & 0 & 0 & 0 \\
				   &   & 0 & 0 \\
				   &   &   & 0
			 \end{pmatrix}$
		 $= \begin{pmatrix}
				 1 & C_n^1 & C_n^2 & C_n^3 \\
				   & 1     & C_n^1 & C_n^2 \\
				   &       & 1     & C_n^1 \\
				   &       &       & 1
			 \end{pmatrix}$.