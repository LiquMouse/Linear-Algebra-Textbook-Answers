\section{6.1}

 \subsection{} %A
	 \paragraph{} %1
		 \begin{enumerate}
			 \item %(1)
			       \( \begin{pmatrix}
				       2  & -2 \\
				       -2 & 5
			       \end{pmatrix} \) 由 \( \begin{pmatrix}
				       2  & -2 \\
				       -2 & 5
			       \end{pmatrix} \rightarrow \begin{pmatrix}
				       1 &   \\
				         & 1
			       \end{pmatrix} \) 知秩为2
			 \item %(2)
			       \( \begin{pmatrix}
				       1  & -2 & -4 \\
				       -2 & -2 & 3  \\
				       -4 & 3  & 3
			       \end{pmatrix} \) 由 \( \begin{pmatrix}
				       1  & -2 & -4 \\
				       -2 & -2 & 3  \\
				       -4 & 3  & 3
			       \end{pmatrix} \rightarrow \begin{pmatrix}
				       1 &   &   \\
				         & 1 &   \\
				         &   & 1 \\
			       \end{pmatrix} \) 知秩为3.
			 \item %(3)
			       \( \begin{pmatrix}
				       1 & 2 & 3 \\
				       2 & 4 & 6 \\
				       3 & 6 & 9
			       \end{pmatrix} \) 由 \( \begin{pmatrix}
				       1 & 2 & 3 \\
				       2 & 4 & 6 \\
				       3 & 6 & 9
			       \end{pmatrix} \rightarrow \begin{pmatrix}
				       1 & 2 & 3 \\
				         &   &   \\
				         &   &
			       \end{pmatrix} \) 知秩为1
			 \item %(4)
			       \( \begin{pmatrix}
				       0 & 1 & 0 & 1 \\
				       1 & 0 & 1 & 0 \\
				       0 & 1 & 0 & 1 \\
				       1 & 0 & 1 & 0
			       \end{pmatrix} \) 由 \( \begin{pmatrix}
				       0 & 1 & 0 & 1 \\
				       1 & 0 & 1 & 0 \\
				       0 & 1 & 0 & 1 \\
				       1 & 0 & 1 & 0
			       \end{pmatrix} \rightarrow
			       \begin{pmatrix}
				       1 &   & 1 &   \\
				         & 1 &   & 1 \\
				         &   &   &   \\
				         &   &   &   \\
			       \end{pmatrix} \) 知秩为2
			 \item %(5)
			       \( \begin{pmatrix}
				       1  & -1 & -1 & -1 \\
				       -1 & 1  & -1 & -1 \\
				       -1 & -1 & 1  & -1 \\
				       -1 & -1 & -1 & 1
			       \end{pmatrix} \) 由 \(
			       \begin{pmatrix}
				       1  & -1 & -1 & -1 \\
				       -1 & 1  & -1 & -1 \\
				       -1 & -1 & 1  & -1 \\
				       -1 & -1 & -1 & 1
			       \end{pmatrix} \rightarrow
			       \begin{pmatrix}
				       1 &   &   &   \\
				         & 1 &   &   \\
				         &   & 1 &   \\
				         &   &   & 1 \\
			       \end{pmatrix} \) 知秩为4.
		 \end{enumerate}


	 \paragraph{} %2
		 \begin{enumerate}
			 \item %(1)
			       \( 3x_{2}^{2} - 4x_{1}x_{2} \)
			 \item %(2)
			       \( 7x_{1}^{2} + 5x_{2}^{2} + 8x_{1}x_{2} \)
			 \item %(3)
			       \( -x_{1}^{2} + 2x_{2}^{2} - 3x_{3}^{2} + 8x_{1}x_{2} + 12x_{1}x_{3} - 10x_{2}x_{3} \)
			 \item %(4)
			       \( -6x_{1}^{2} + 7x_{2}^{2} - 2x_{3}^{2} + 6x_{1}x_{2} \)
			 \item %(5)
			       \( x_{1}^{2} + x_{2}^{2} - 2x_{3}^{2} - 2x_{4}^{2} + 4x_{1}x_{2} - 2x_{3}x_{4} \)
			 \item %(6)
			       \( x_{1}^{2} + x_{2}^{2} + x_{3}^{2} + x_{4}^{2} + 2x_{1}x_{2} + 2x_{1}x_{3} + 2x_{1}x_{4} + 2x_{2}x_{3} + 2x_{2}x_{4} + 2x_{3}x_{4} \)
		 \end{enumerate}


 \subsection{} %B


	 \paragraph{} %1
		 证明: 必要性: 由 \( A = O \), 则必有 \( \alpha^{\mathrm{T}}A\alpha = 0 \)

		 充分性: 设 \( A = (a_{ij})_{n \times n} \), 则 \( X = (x_{1}, x_{2}, \dots, x_{n})^{\mathrm{T}} \),

		 由 \( X^{\mathrm{T}}AX = 0 \), 取 \( X \) 为 \( \varepsilon_{1} = (1, 0, 0, \dots, 0)^{\mathrm{T}} \), \( \varepsilon_{2} = (0, 1, \dots, 0)^{\mathrm{T}} \), ..., \( \varepsilon_{n} = (0, 0, \dots, 1)^{\mathrm{T}} \).

		 代入 \( X^{\mathrm{T}}AX = 0 \Rightarrow a_{11} = 0, a_{22} = 0, \dots, a_{nn} = 0 \).

		 又取 \( X \) 为 \( \varepsilon_{i} + \varepsilon_{j} = (0, 0, \dots, 1, 0, \dots, 1, \dots, 0)^{\mathrm{T}} \),

		 代入 \( X^{\mathrm{T}}AX = 0 \), 可以得到 \( a_{ij} = 0 \) (\( i \neq j \), \( i, j = 1, 2, \dots, n \)).

		 则 \( A = O \).


	 \paragraph{} %2
		 证明: 必要性: 由 \( A^{\mathrm{T}} = -A \), 则 \( (X^{\mathrm{T}}AX)^{\mathrm{T}} = X^{\mathrm{T}}A^{\mathrm{T}}X = -X^{\mathrm{T}}AX \),

		 又 \( X^{\mathrm{T}}AX \) 是一个数, 于是 \( (X^{\mathrm{T}}AX)^{\mathrm{T}} = X^{\mathrm{T}}AX \),

		 则 \( X^{\mathrm{T}}AX = 0 \).

		 充分性: 设 \( X^{\mathrm{T}}AX = 0 \), 取 \( X \) 为 \( \varepsilon_{1} = (1, 0, \dots, 0)^{\mathrm{T}} \), \( \varepsilon_{2} = (0, 1, \dots, 0)^{\mathrm{T}} \), ..., \( \varepsilon_{n} = (0, 0, \dots, 1)^{\mathrm{T}} \),

		 代入 \( X^{\mathrm{T}}AX = 0 \Rightarrow a_{11} = 0, a_{22} = 0, \dots, a_{nn} = 0. \)
		 又取 \( X = \varepsilon_{i} + \varepsilon_{j} \) (\( i \neq j \)),

		 代入 \( X^{\mathrm{T}}AX = 0 \) 有 \( X^{\mathrm{T}}AX = (\varepsilon_{i} + \varepsilon_{j})A(\varepsilon_{i} + \varepsilon_{j})^{\mathrm{T}} = a_{ij} + a_{ji} = 0 \)
		 故 \( a_{ij} = -a_{ji} \), 从而 \( A \) 为反对称矩阵.


\section{6.2}

 \subsection{} %A


	 \paragraph{} %1
		 \begin{enumerate}
			 \item %(1)
			       \( A = \begin{pmatrix}
				       2  & -2 & 0  \\
				       -2 & 1  & -2 \\
				       0  & -2 & 0
			       \end{pmatrix} \),

			       则 \( f_{A}(\lambda) = |\lambda E - A| = \begin{vmatrix}
				       \lambda - 2 & 2           & 0       \\
				       2           & \lambda - 1 & 2       \\
				       0           & 2           & \lambda
			       \end{vmatrix} = (\lambda + 2)(\lambda - 1)(\lambda - 4) \),

			       \( \lambda =-2 \) 时, \( \begin{pmatrix}
				       -4 & 2  & 0  \\
				       2  & -3 & 2  \\
				       0  & 2  & -2
			       \end{pmatrix} \rightarrow \begin{pmatrix}
				       1 & 0 & -\frac{1}{2} \\
				       0 & 1 & -1           \\
				       0 & 0 & 0
			       \end{pmatrix}, \alpha_{1} = (1, 2, 2)^{\mathrm{T}} \);

			       \( \lambda = 1 \) 时, \( \begin{pmatrix}
				       -1 & 2 & 0 \\
				       2  & 0 & 2 \\
				       0  & 2 & 1
			       \end{pmatrix} \rightarrow \begin{pmatrix}
				       1 & 0 & 1           \\
				       0 & 1 & \frac{1}{2} \\
				       0 & 0 & 0
			       \end{pmatrix} , \alpha_{2} = (2, 1, -2)^{\mathrm{T}} \);

			       \( \lambda = 4 \) 时, \( \begin{pmatrix}
				       2 & 2 & 0 \\
				       2 & 3 & 2 \\
				       0 & 2 & 4
			       \end{pmatrix} \rightarrow \begin{pmatrix}
				       1 & 0 & -2 \\
				       0 & 1 & 2  \\
				       0 & 0 & 0
			       \end{pmatrix} , \alpha_{3} = (2, -2, 1)^{\mathrm{T}} \);

			       正交单位化, 有 \( p_{1} = \left(\frac{1}{3}, \frac{2}{3}, \frac{2}{3}\right)^{\mathrm{T}} \), \( p_{2} = \left(\frac{2}{3}, \frac{1}{3}, -\frac{2}{3}\right)^{\mathrm{T}} \), \( p_{3} = \left(\frac{2}{3}, -\frac{2}{3}, \frac{1}{3}\right)^{\mathrm{T}}, \)

			       标准形 \( y_{1}^{2} - 2y_{2}^{2} + 4y_{3}^{2} \), \( X = \begin{pmatrix}
				       \frac{2}{3}  & \frac{1}{3} & \frac{2}{3}  \\
				       \frac{1}{3}  & \frac{2}{3} & -\frac{2}{3} \\
				       -\frac{2}{3} & \frac{2}{3} & \frac{1}{3}
			       \end{pmatrix}Y \).
			 \item %(2)
			       \( A = \begin{pmatrix}
				       1  & -2 & -4 \\
				       -2 & 4  & -2 \\
				       -4 & -2 & 1
			       \end{pmatrix} \),

			       则 \( f_{A}(\lambda) = |\lambda E - A| = \begin{vmatrix}
				       \lambda - 1 & 2           & 4           \\
				       2           & \lambda - 4 & 2           \\
				       4           & 2           & \lambda - 1
			       \end{vmatrix} = (\lambda + 4)(\lambda - 5)^{2} \),

			       \( 5E - A = \begin{pmatrix}
				       4 & 2 & 4 \\
				       2 & 1 & 2 \\
				       4 & 2 & 4
			       \end{pmatrix} \rightarrow \begin{pmatrix}
				       1 & \frac{1}{2} & 1 \\
				         &             &   \\
				         &             &
			       \end{pmatrix} \),

			       得 \( \alpha_{1} = (1, -2, 0)^{\mathrm{T}} \), \( \alpha_{2} = (-1, 0, 1)^{\mathrm{T}} \);

			       \( -4E - A = \begin{pmatrix}
				       -5 & 2  & 4  \\
				       2  & -8 & 2  \\
				       4  & 2  & -5
			       \end{pmatrix} \rightarrow \begin{pmatrix}
				       1 &   & -1           \\
				         & 1 & -\frac{1}{2} \\
				         &   &
			       \end{pmatrix} \),

			       得 \( \alpha_{3} = (2, 1, 2)^{\mathrm{T}} \);

			       正交单位化, 有 \[ p_{1} = \left(\frac{1}{\sqrt{5}}, 0, -\frac{1}{\sqrt{5}}\right)^{\mathrm{T}},
				       \newline  p_{2} = \left(\frac{\sqrt{5}}{6}, -\frac{2\sqrt{5}}{3}, -\frac{\sqrt{5}}{6}\right)^{\mathrm{T}}, \newline p_{3} = \left(\frac{2}{3}, \frac{1}{3}, \frac{2}{3}\right), \]

			       标准形 \( 5y_{1}^{2} + 5y_{2}^{2} - 4y_{3}^{2} \), \( X = \begin{pmatrix}
				       \frac{1}{\sqrt{5}}  & \frac{\sqrt{5}}{6}   & \frac{2}{3} \\
				       0                   & -\frac{2\sqrt{5}}{3} & \frac{1}{3} \\
				       -\frac{1}{\sqrt{5}} & -\frac{\sqrt{5}}{6}  & \frac{2}{3}
			       \end{pmatrix}Y \).
			 \item %(3)
			       \( A = \begin{pmatrix}
				       0           & \frac{1}{2} & \frac{1}{2} \\
				       \frac{1}{2} & 0           & \frac{1}{2} \\
				       \frac{1}{2} & \frac{1}{2} & 0
			       \end{pmatrix} \),

			       则 \( f_{A}(\lambda) = |\lambda E - A| = \begin{vmatrix}
				       \lambda      & -\frac{1}{2} & -\frac{1}{2} \\
				       -\frac{1}{2} & \lambda      & -\frac{1}{2} \\
				       -\frac{1}{2} & -\frac{1}{2} & \lambda
			       \end{vmatrix} = (\lambda + \frac{1}{2})^{2}(\lambda - 1) \),

			       \( -\frac{1}{2}E - A = \begin{pmatrix}
				       -\frac{1}{2} & -\frac{1}{2} & -\frac{1}{2} \\
				       -\frac{1}{2} & -\frac{1}{2} & -\frac{1}{2} \\
				       -\frac{1}{2} & -\frac{1}{2} & -\frac{1}{2}
			       \end{pmatrix} \rightarrow \begin{pmatrix}
				       1 & 1 & 1 \\
				         &   &   \\
				         &   &
			       \end{pmatrix}\),

			       得 \(\alpha_{1} = (1, -1, 0)^{\mathrm{T}} \), \( \alpha_{2} = (1, 0, -1)^{\mathrm{T}} \);

			       \( E - A = \begin{pmatrix}
				       1            & -\frac{1}{2} & -\frac{1}{2} \\
				       -\frac{1}{2} & 1            & -\frac{1}{2} \\
				       -\frac{1}{2} & -\frac{1}{2} & 1
			       \end{pmatrix} \rightarrow \begin{pmatrix}
				       1 &   & -1 \\
				         & 1 & -1 \\
				         &   &
			       \end{pmatrix}\),

			       得 \(\alpha_{3} = (1, 1, 1)^{\mathrm{T}} \)

			       则 \( p_{1} = \left(\frac{1}{\sqrt{2}}, 0, -\frac{1}{\sqrt{2}}\right)^{\mathrm{T}} \), \( p_{2} = \left(-\frac{1}{\sqrt{6}}, \frac{2}{\sqrt{6}}, -\frac{1}{\sqrt{6}}\right) \),\newline \( p_{3} = \left(\frac{1}{\sqrt{3}}, \frac{1}{\sqrt{3}}, \frac{1}{\sqrt{3}}\right) \),

			       标准形 \( -\frac{1}{2}y_{1}^{2} - \frac{1}{2}y_{2}^{2} + y_{3}^{2} \), \( X = \begin{pmatrix}
				       -\frac{1}{\sqrt{2}} & -\frac{1}{\sqrt{6}} & \frac{1}{\sqrt{3}} \\
				       0                   & \frac{2}{\sqrt{6}}  & \frac{1}{\sqrt{3}} \\
				       -\frac{1}{\sqrt{2}} & \frac{1}{\sqrt{6}}  & \frac{1}{\sqrt{3}}
			       \end{pmatrix}Y. \)

			 \item %(4)
			       \( A = \begin{pmatrix}
				       0           & 0           & 0           & \frac{1}{2} \\
				       0           & 0           & \frac{1}{2} & 0           \\
				       0           & \frac{1}{2} & 0           & 0           \\
				       \frac{1}{2} & 0           & 0           & 0
			       \end{pmatrix} \),

			       则 \( f_{A}(\lambda) = |\lambda E - A| = \begin{vmatrix}
				       \lambda      & 0            & 0            & -\frac{1}{2} \\
				       0            & \lambda      & -\frac{1}{2} & 0            \\
				       0            & -\frac{1}{2} & \lambda      & 0            \\
				       -\frac{1}{2} & 0            & 0            & \lambda
			       \end{vmatrix} = (\lambda - \frac{1}{2})^{2}(\lambda + \frac{1}{2})^{2} \),

			       \( \frac{1}{2}E - A = \begin{pmatrix}
				       \frac{1}{2}  & 0            & 0            & -\frac{1}{2} \\
				       0            & \frac{1}{2}  & -\frac{1}{2} & 0            \\
				       0            & -\frac{1}{2} & \frac{1}{2}  & 0            \\
				       -\frac{1}{2} & 0            & 0            & \frac{1}{2}
			       \end{pmatrix} \rightarrow \begin{pmatrix}
				       1 & 0 & 0  & -1 \\
				       0 & 1 & -1 & 0  \\
				       0 & 0 & 0  & 0  \\
				       0 & 0 & 0  & 0
			       \end{pmatrix}\),

			       得 \(\alpha_{1} = (1, 0, 0, 1)^{\mathrm{T}} \), \( \alpha_{2} = (0, 1, 1, 0)^{\mathrm{T}} \);

			       \( -\frac{1}{2}E - A = \begin{pmatrix}
				       -\frac{1}{2} & 0            & 0            & -\frac{1}{2} \\
				       0            & -\frac{1}{2} & -\frac{1}{2} & 0            \\
				       0            & -\frac{1}{2} & -\frac{1}{2} & 0            \\
				       -\frac{1}{2} & 0            & 0            & -\frac{1}{2}
			       \end{pmatrix} \rightarrow \begin{pmatrix}
				       1 & 0 & 0 & 1 \\
				       0 & 1 & 1 & 0 \\
				       0 & 0 & 0 & 0 \\
				       0 & 0 & 0 & 0
			       \end{pmatrix}\),

			       得 \(\alpha_{3} = (1, 0, 0, -1)^{\mathrm{T}} \), \( \alpha_{4} = (0, 1, -1, 0)^{\mathrm{T}} \);

			       标准形 \( \frac{1}{2}y_{1}^{2} + \frac{1}{2}y_{2}^{2} - \frac{1}{2}y_{3}^{2} - \frac{1}{2}y_{4}^{2} \), \( X = \begin{pmatrix}
				       \frac{1}{\sqrt{2}} & 0                  & \frac{1}{\sqrt{2}}  & 0                   \\
				       0                  & \frac{1}{\sqrt{2}} & 0                   & \frac{1}{\sqrt{2}}  \\
				       0                  & \frac{1}{\sqrt{2}} & 0                   & -\frac{1}{\sqrt{2}} \\
				       \frac{1}{\sqrt{2}} & 0                  & -\frac{1}{\sqrt{2}} & 0
			       \end{pmatrix}Y \)
		 \end{enumerate}


	 \paragraph{} %2
		 \begin{enumerate}
			 \item %(1)
			       \begin{flalign*}
				        & \quad x_{1}^{2} + 2x_{2}^{2} - x_{3}^{2} + 2x_{1}x_{2} - 2x_{3}x_{1}                                       & \\
				        & = [x_{1}^{2} + 2x_{1}(x_{2} - x_{3}) + (x_{2} - x_{3})^{2}] - (x_{2} - x_{3})^{2} + 2x_{2}^{2} - x_{3}^{2} & \\
				        & = (x_{1} + x_{2} - x_{3})^{2} + x_{2}^{2} + 2x_{2}x_{3} - 2x_{3}^{2}                                       & \\
				        & = (x_{1} + x_{2} - x_{3})^{2} + (x_{2} + x_{3})^{2} - 3x_{3}^{2}                                           &
			       \end{flalign*}



			       则 \( \begin{cases}
				       y_{1} = x_{1} + x_{2} - x_{3} \\
				       y_{2} = x_{2} + x_{3}         \\
				       y_{3} = x_{3}
			       \end{cases} \),

			       得 \( X = \begin{pmatrix}
				       1 & -1 & 2  \\
				         & 1  & -1 \\
				         &    & 1
			       \end{pmatrix}Y \), 化为 \( y_{1}^{2} + y_{2}^{2} - 3y_{3}^{2} \).
			 \item %(2)
			       令 \( \begin{cases}
				       x_{1} = y_{1} + y_{2} \\
				       x_{2} = y_{1} - y_{2} \\
				       x_{3} = y_{3}
			       \end{cases} \)

			       则 \( f = 2y_{1}^{2} - 2y_{2}^{2} + 4y_{1}y_{3} = 2(y_{1}^{2} + 2y_{1}y_{3} + y_{3}^{2}) - 2y_{2}^{2} - 2y_{3}^{2} = 2(y_{1} + y_{3})^{2} - 2y_{2}^{2} - 2y_{3}^{2}, \)

			       则 \( \begin{cases}
				       z_{1} = y_{1} + y_{3} \\
				       z_{2} = y_{2}         \\
				       z_{3} = y_{3}
			       \end{cases} \Rightarrow \begin{cases}
				       y_{1} = z_{1} - z_{3} \\
				       y_{2} = z_{2}         \\
				       y_{3} = z_{3}
			       \end{cases} \)

			       故 \( X = \begin{pmatrix}
				       1 & 1  & -1 \\
				       1 & -1 & -1 \\
				       0 & 0  & 1
			       \end{pmatrix}Z \), 化为 \( 2z_{1}^{2} - 2z_{2}^{2} - 2z_{3}^{2} \).
			 \item %(3)
			       \begin{flalign*}
				        & \quad x_{1}^{2} - x_{2}^{2} + 2x_{1}x_{2} + 4x_{3}x_{1}                                          & \\
				        & = [x_{1}^{2} + 2(x_{2} + 2x_{3})x_{1} + (x_{2} + 2x_{3})^{2}] - (x_{2} + 2x_{3})^{2} - x_{2}^{2} & \\
				        & = (x_{1} + x_{2} + 2x_{3})^{2} - x_{2}^{2} - (x_{2} + 2x_{3})^{2}                                &
			       \end{flalign*}
			       令 \( \begin{cases}
				       y_{1} = x_{1} + x_{2} + 2x_{3} \\
				       y_{2} = x_{2}                  \\
				       y_{3} = x_{2} + 2x_{3}
			       \end{cases} \)

			       得 \( X = \begin{pmatrix}
				       1 & 0            & -1          \\
				       0 & 1            & 0           \\
				       0 & -\frac{1}{2} & \frac{1}{2}
			       \end{pmatrix}Y \), 化为 \( y_{1}^{2} - y_{2}^{2} - y_{3}^{2} \).

			 \item %(4)
			       令 \( X = \begin{pmatrix}
				       1 & -1 \\
				       1 & -1 \\
				       1 & 1  \\
				       1 & 1
			       \end{pmatrix}Y \), 则 \( x_{1}x_{4} + x_{2}x_{3} = y_{1}^{2} - y_{4}^{2} + y_{2}^{2} - y_{3}^{2} \)
		 \end{enumerate}


	 \paragraph{} %3
		 \begin{tabular}{ccccc}
			     & 秩 & 正惯性指数 & 负惯性指数 & 符号差 \\
			 1.                            \\
			 (1) & 3 & 2     & 1     & 1   \\
			 (2) & 3 & 2     & 1     & 1   \\
			 (3) & 3 & 1     & 2     & -1  \\
			 (4) & 4 & 2     & 2     & 0   \\
			 2.                            \\
			 (1) & 3 & 2     & 1     & 1   \\
			 (2) & 3 & 1     & 2     & -1  \\
			 (3) & 3 & 1     & 2     & -1  \\
			 (4) & 4 & 2     & 2     & 0
		 \end{tabular}












	 \paragraph{} %4
		 依题意 \( A \sim \begin{pmatrix}
			 9 &   &   \\
			   & 9 &   \\
			   &   & 0
		 \end{pmatrix} \) 且 \( A\beta_{3} = 0 \)

		 考虑 \( PP^{\mathrm{T}} = \begin{pmatrix}
			 1 \\
			 2 \\
			 2
		 \end{pmatrix}(1\ 2\ 2) = \begin{pmatrix}
			 1 & 2 & 2 \\
			 2 & 4 & 4 \\
			 2 & 4 & 4
		 \end{pmatrix} \rightarrow \begin{pmatrix}
			 1 & 2 & 2 \\
			   &   &   \\
			   &   &
		 \end{pmatrix} \)

		 由于不同的特征值的特征向量正交. 故 \( PP^{\mathrm{T}}\alpha = 0 \) 的解即其他特征向量

		 则可取 \( \alpha_{2} = \begin{pmatrix}
			 2  \\
			 -1 \\
			 0
		 \end{pmatrix} \), \( \alpha_{3} = \begin{pmatrix}
			 2 \\
			 0 \\
			 -1
		 \end{pmatrix} \), \( Q = \begin{pmatrix}
			 1 & 2  & 2  \\
			 2 & -1 & 0  \\
			 2 & 0  & -1
		 \end{pmatrix} \), \( \Lambda = \begin{pmatrix}
			 0 &   &   \\
			   & 9 &   \\
			   &   & 9
		 \end{pmatrix} \)

		 故 \( A = Q\Lambda Q^{-1} = \begin{pmatrix}
			 1 & 2  & 2  \\
			 2 & -1 & 0  \\
			 2 & 0  & -1
		 \end{pmatrix}\begin{pmatrix}
			 0 &   &   \\
			   & 9 &   \\
			   &   & 9
		 \end{pmatrix}\begin{pmatrix}
			 \frac{1}{9} & \frac{2}{9}  & \frac{2}{9}  \\
			 \frac{2}{9} & -\frac{5}{9} & \frac{4}{9}  \\
			 \frac{2}{9} & \frac{4}{9}  & -\frac{5}{9}
		 \end{pmatrix} = \begin{pmatrix}
			 8  & -2 & -2 \\
			 -2 & 5  & -4 \\
			 -2 & -4 & 5
		 \end{pmatrix} \)


\section{6.3}

 \subsection{} %A


	 \paragraph{} %1
		 \begin{enumerate}
			 \item %(1)
			       由 \( A = \begin{pmatrix}
				       1  & -4 & 1  \\
				       -4 & 1  & -2 \\
				       1  & -2 & 2
			       \end{pmatrix} \), \( f_{A}(\lambda) = \begin{vmatrix}
				       \lambda - 1 & 4           & -1          \\
				       4           & \lambda - 1 & 2           \\
				       -1          & 2           & \lambda - 2
			       \end{vmatrix} \),

			       由于 \( \begin{vmatrix}
				       1  & -4 \\
				       -4 & 1
			       \end{vmatrix} = -7 < 0 \), 故 \( A \) 不是正定的.
			 \item %(2)
			       \( A = \begin{pmatrix}
				       7  & -2 & 0  \\
				       -2 & 8  & -2 \\
				       0  & -2 & 6
			       \end{pmatrix} \), 又 \( |7| > 0 \), \( \begin{vmatrix}
				       7  & -2 \\
				       -2 & 8
			       \end{vmatrix} = 52 > 0 \), \( \begin{vmatrix}
				       7  & -2 & 0  \\
				       -2 & 8  & -2 \\
				       0  & -2 & 6
			       \end{vmatrix} = 284 > 0 \), 故 \( A \) 是正定的.
			 \item %(3)
			       \( A = \begin{pmatrix}
				       99 & -6  & 24  \\
				       -6 & 130 & -30 \\
				       24 & -30 & 70
			       \end{pmatrix} \), 又 \( |99| > 0 \), \( \begin{vmatrix}
				       99 & -6  \\
				       -6 & 130
			       \end{vmatrix} > 0 \), \( \begin{vmatrix}
				       99 & -6  & 24  \\
				       -6 & 130 & -30 \\
				       24 & -30 & 70
			       \end{vmatrix} = 743040 > 0 \), 故 A 是正定的.
			 \item %(4)
			       \( A = \begin{pmatrix}
				       10 & 4   & 12  \\
				       4  & 2   & -14 \\
				       12 & -14 & 1
			       \end{pmatrix} \), 又 \( |A| = -3588 < 0 \), 不是正定的
		 \end{enumerate}

	 \paragraph{} %2
		 证明: 依题意, \( X^{\mathrm{T}}AX \), \( X^{\mathrm{T}}BX \) 是正定二次型, 则
		 \[ X^{\mathrm{T}}(A+B)X = X^{\mathrm{T}}AX + X^{\mathrm{T}}BX, \]
		 由于 \( X^{\mathrm{T}}AX > 0 \), \( X^{\mathrm{T}}BX > 0 \), 故 \( X^{\mathrm{T}}(A+B)X > 0 \),

		 故 \( X^{\mathrm{T}}(A+B)X \) 也是正定的,

		 从而 \( A+B \) 也是正定矩阵.

	 \paragraph{} %3
		 \begin{enumerate}
			 \item %(1)
			       \( A = \begin{pmatrix}
				       4  & -1 & 2  \\
				       -1 & 1  & -1 \\
				       2  & -1 & t
			       \end{pmatrix} \), 则 \( |A| > 1 \), \( \begin{vmatrix}
				       4  & -1 \\
				       -1 & 1
			       \end{vmatrix} = 3 > 0 \), \( \begin{vmatrix}
				       4  & -1 & 2  \\
				       -1 & 1  & -1 \\
				       2  & -1 & t
			       \end{vmatrix} > 0 \),

			       得 \( t > \frac{4}{3} \).
			 \item %(2)
			       \( A = \begin{pmatrix}
				       2  & t & -3 \\
				       t  & 3 & 1  \\
				       -3 & 1 & 4
			       \end{pmatrix} \), 由 \( \begin{vmatrix}
				       2 & t \\
				       t & 3
			       \end{vmatrix} > 0 \) 得 \( t^{2} < 6 \), 又 \( \begin{vmatrix}
				       2  & t & -3 \\
				       t  & 3 & 1  \\
				       -3 & 1 & 4
			       \end{vmatrix} > 0 \),

			       得 \( -4t^{2} - 72t - 49 > 0 \),

			       但由 \( t > -\sqrt{6} \),

			       则 \( 4t^{2} + 72t + 49 > 193 - 72\sqrt{6} > 16 > 0 \), 故它不可能正定
		 \end{enumerate}


	 \paragraph{} %4
		 证明: 由于 \( A^{\mathrm{T}} = A \), 则 \( A = A^{\mathrm{T}}A^{-1}A \), 即 \( A^{-1} \) 合同于 \( A \),

		 由于 \( A \) 是正定的, 故 \( A^{-1} \) 也正定.


	 \paragraph{} %5
		 证明: 由于 \( f(x_{1}, x_{2}, \dots, x_{n}) = X^{\mathrm{T}}AX = \sum_{i=1}^{n}\sum_{j=1}^{n}a_{ij}x_{i}x_{j} \) 正定,

		 取 \( X_{i} = \varepsilon_{i}^{\mathrm{T}} = (0, \dots, 0, 1, 0, \dots, 0)^{\mathrm{T}} \) (其中第 \( i \) 个分量 \( x_{i} = 1 \)),

		 则 \( X_{i}^{\mathrm{T}}AX_{i} = a_{ii}x_{i}^{2} = a_{ii} > 0 \).