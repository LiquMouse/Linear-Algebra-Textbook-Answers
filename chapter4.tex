\section{4.1}
 \subsection{} %A
	 \paragraph{} %1
		 \begin{enumerate}
			 \item %(1)
			       由于 \( \begin{pmatrix}
				       3  & 4 & -7 & 1 \\
				       2  & 1 & -6 & 0 \\
				       -1 & 2 & 5  & 1
			       \end{pmatrix} \rightarrow \begin{pmatrix}
				       1 &   & -\frac{17}{5} & -\frac{1}{5} \\
				         & 1 & \frac{4}{5}   & \frac{2}{5}  \\
				         &   &               &
			       \end{pmatrix} \)

			       故取 \( x_{3}, x_{4} \) 为自由未知量. 令 \( \begin{pmatrix}
				       x_{3} \\
				       x_{4}
			       \end{pmatrix} \) 为 \( \begin{pmatrix}
				       1 \\
				       0
			       \end{pmatrix} \), \( \begin{pmatrix}
				       0 \\
				       1
			       \end{pmatrix} \) 得基础解系\[ \eta_{1} = \begin{pmatrix}
					       \frac{17}{5} \\
					       -\frac{4}{5} \\
					       1            \\
					       0
				       \end{pmatrix} \quad \eta_{2} = \begin{pmatrix}
					       \frac{1}{5}  \\
					       -\frac{2}{5} \\
					       0            \\
					       1
				       \end{pmatrix} \]

			       故 \( x = c_{1}\eta_{1} + c_{2}\eta_{2} = c_{1}\begin{pmatrix}
				       \frac{17}{5} \\
				       -\frac{4}{5} \\
				       1            \\
				       0
			       \end{pmatrix} + c_{2}\begin{pmatrix}
				       \frac{1}{5}  \\
				       -\frac{2}{5} \\
				       0            \\
				       1
			       \end{pmatrix} \)
			 \item %(2)
			       由于 \( \begin{pmatrix}
				       2  & 3 & 1 & 0 \\
				       -5 & 7 & 0 & 1
			       \end{pmatrix} \) 则取 \( x_{3}, x_{4} \) 为自由未知量.

			       令 \( \begin{pmatrix}
				       x_{1} \\
				       x_{2}
			       \end{pmatrix} \) 为 \( \begin{pmatrix}
				       1 \\
				       0
			       \end{pmatrix} \), \( \begin{pmatrix}
				       0 \\
				       1
			       \end{pmatrix} \), 得 \( \eta_{1} = \begin{pmatrix}
				       1  \\
				       0  \\
				       -2 \\
				       5
			       \end{pmatrix} \quad \eta_{2} = \begin{pmatrix}
				       0  \\
				       1  \\
				       -3 \\
				       -7
			       \end{pmatrix} \)

			       故 \( x = c_{1}\begin{pmatrix}
				       1  \\
				       0  \\
				       -2 \\
				       5
			       \end{pmatrix} + c_{2}\begin{pmatrix}
				       0  \\
				       1  \\
				       -3 \\
				       -7
			       \end{pmatrix} \)
			 \item %(3)
			       由于 \( \begin{pmatrix}
				       1 & 2 & 1 & 1  \\
				       2 & 2 & 0 & -1 \\
				       5 & 6 & 1 & -1
			       \end{pmatrix} \rightarrow \begin{pmatrix}
				       1 &   & -1 & -2          \\
				         & 1 & 1  & \frac{3}{2} \\
				         &   &    &
			       \end{pmatrix} \) 则取 \( x_{3}, x_{4} \) 为自由未知量

			       令 \( \begin{pmatrix}
				       x_{3} \\
				       x_{4}
			       \end{pmatrix} \) 为 \( \begin{pmatrix}
				       1 \\
				       0
			       \end{pmatrix} \), \( \begin{pmatrix}
				       0 \\
				       2
			       \end{pmatrix} \), 得 \( \eta_{1} = \begin{pmatrix}
				       1  \\
				       -1 \\
				       1  \\
				       0
			       \end{pmatrix} \quad \eta_{2} = \begin{pmatrix}
				       4  \\
				       -3 \\
				       0  \\
				       2
			       \end{pmatrix} \)

			       故 \( x = c_{1}\begin{pmatrix}
				       1  \\
				       -1 \\
				       1  \\
				       0
			       \end{pmatrix} + c_{2}\begin{pmatrix}
				       4  \\
				       -3 \\
				       0  \\
				       2
			       \end{pmatrix} \)
			 \item %(4)
			       由于 \( \begin{pmatrix}
				       1  & -2 & 1  & 1  & 1  \\
				       1  & -2 & 2  & -1 & -1 \\
				       -1 & 2  & -1 & -2 & -3 \\
				       2  & -4 & 3  & 1  & 2
			       \end{pmatrix} \rightarrow \begin{pmatrix}
				       1 & -2 &   &   & -3 \\
				         &    & 1 &   & 2  \\
				         &    &   & 1 & 2  \\
				         &    &   &   &
			       \end{pmatrix} \) 故取 \( x_{2}, x_{5} \) 为自由未知量

			       令 \( \begin{pmatrix}
				       x_{2} \\
				       x_{5}
			       \end{pmatrix} \) 为 \( \begin{pmatrix}
				       1 \\
				       0
			       \end{pmatrix} \), \( \begin{pmatrix}
				       0 \\
				       1
			       \end{pmatrix} \) 得 \( \eta_{1} = \begin{pmatrix}
				       2 \\
				       1 \\
				       0 \\
				       0 \\
				       0
			       \end{pmatrix} \quad \eta_{2} = \begin{pmatrix}
				       3  \\
				       0  \\
				       -2 \\
				       -2 \\
				       1
			       \end{pmatrix} \)

			       故 \( x = c_{1}\begin{pmatrix}
				       2 \\
				       1 \\
				       0 \\
				       0 \\
				       0
			       \end{pmatrix} + c_{2}\begin{pmatrix}
				       3  \\
				       0  \\
				       -2 \\
				       -2 \\
				       1
			       \end{pmatrix} \)
			 \item %(5)
			       由于系数矩阵为 \( \begin{pmatrix}
				       1 & 1 & 1 & 1 \\
				         &   &   &   \\
				         &   &   &
			       \end{pmatrix} \), 则取 \( x_{2}, x_{3}, x_{4} \) 为自由未知量.

			       则令 \( \begin{pmatrix}
				       x_{2} \\
				       x_{3} \\
				       x_{4}
			       \end{pmatrix} \) 为 \( \begin{pmatrix}
				       1 \\
				       0 \\
				       0
			       \end{pmatrix} \), \( \begin{pmatrix}
				       0 \\
				       1 \\
				       0
			       \end{pmatrix} \), \( \begin{pmatrix}
				       0 \\
				       0 \\
				       1
			       \end{pmatrix} \) 得 \( \eta_{1} = \begin{pmatrix}
				       -1 \\
				       1  \\
				       0  \\
				       0
			       \end{pmatrix} \), \( \eta_{2} = \begin{pmatrix}
				       -1 \\
				       0  \\
				       1  \\
				       0
			       \end{pmatrix} \), \( \eta_{3} = \begin{pmatrix}
				       -1 \\
				       0  \\
				       0  \\
				       1
			       \end{pmatrix} \)

			       故 \( x = c_{1}\begin{pmatrix}
				       -1 \\
				       1  \\
				       0  \\
				       0
			       \end{pmatrix} + c_{2}\begin{pmatrix}
				       -1 \\
				       0  \\
				       1  \\
				       0
			       \end{pmatrix} + c_{3}\begin{pmatrix}
				       -1 \\
				       0  \\
				       0  \\
				       1
			       \end{pmatrix} \)
			 \item %(6)
			       同(5), 有 \( x = c_{1}\begin{pmatrix}
				       -1     \\
				       1      \\
				       0      \\
				       \vdots \\
				       0
			       \end{pmatrix} + c_{2}\begin{pmatrix}
				       -1     \\
				       0      \\
				       1      \\
				       \vdots \\
				       0
			       \end{pmatrix} + \cdots + c_{n-1}\begin{pmatrix}
				       -1     \\
				       0      \\
				       0      \\
				       \vdots \\
				       1
			       \end{pmatrix} \)
		 \end{enumerate}


	 \paragraph{} %2
		 \begin{enumerate}
			 \item %(1)
			       对 I, 有 \( \begin{pmatrix}
				       1 & 2 & 3 & -1 \\
				       3 & 2 & 1 & -1
			       \end{pmatrix} \rightarrow \begin{pmatrix}
				       1 & 0 & -1 & 0            \\
				       0 & 1 & 2  & -\frac{1}{2}
			       \end{pmatrix} \Rightarrow x_{\text{I}} = a_{1}\begin{pmatrix}
				       -1 \\
				       -2 \\
				       1  \\
				       0
			       \end{pmatrix} + a_{2}\begin{pmatrix}
				       0 \\
				       1 \\
				       0 \\
				       2
			       \end{pmatrix} \)

			       对 II, 有 \( \begin{pmatrix}
				       2 & 3 & 1 & 1  \\
				       2 & 2 & 2 & -1 \\
				       5 & 5 & 2 & 0
			       \end{pmatrix} \rightarrow \begin{pmatrix}
				       1 &   &   & -\frac{5}{6} \\
				         & 1 &   & \frac{7}{6}  \\
				         &   & 1 & -\frac{5}{6}
			       \end{pmatrix} \Rightarrow x_{\text{II}} = b\begin{pmatrix}
				       -5 \\
				       -7 \\
				       5  \\
				       6
			       \end{pmatrix} \)

			       故取 \( a_{1} = 5a_{0} \), \( a_{2} = 3a_{0} \) 时 \( x_{\text{I}} = a_{0}\begin{pmatrix}
				       5  \\
				       -7 \\
				       5  \\
				       6
			       \end{pmatrix} \) 为公共解.

			       故全部非空公共解为 \( x = c\begin{pmatrix}
				       5  \\
				       -7 \\
				       5  \\
				       6
			       \end{pmatrix} \)
		 \end{enumerate}


	 \paragraph{} %3
		 依题意, 系数行列式为0. 即
		 \[ \begin{vmatrix}
				 1  & 1  & a \\
				 -1 & a  & 1 \\
				 1  & -1 & 2
			 \end{vmatrix} = \begin{vmatrix}
				 1 & 1   & a   \\
				 0 & a+1 & a+1 \\
				 0 & -2  & 2-a
			 \end{vmatrix} = (a+1)\begin{vmatrix}
				 1 & 1 & a   \\
				   & 1 & 1   \\
				   &   & 4-a
			 \end{vmatrix} = (a+1)(4-a) = 0 \]

		 当 \( a = -1 \) 时 \( \begin{pmatrix}
			 1  & 1  & -1 \\
			 -1 & -1 & 1  \\
			 1  & -1 & 2
		 \end{pmatrix} \rightarrow \begin{pmatrix}
			 1 &   & \frac{1}{2}  \\
			   & 1 & -\frac{3}{2} \\
			   &   &
		 \end{pmatrix} \)

		 则通解为 \( c\begin{pmatrix}
			 -1 \\
			 3  \\
			 2
		 \end{pmatrix} \)

		 当 \( a = 4 \) 时 \( \begin{pmatrix}
			 1  & 1  & 4 \\
			 -1 & 4  & 1 \\
			 1  & -1 & 2
		 \end{pmatrix} \rightarrow \begin{pmatrix}
			 1 &   & 3 \\
			   & 1 & 1 \\
			   &   &
		 \end{pmatrix} \)

		 则通解为 \( c\begin{pmatrix}
			 -3 \\
			 -1 \\
			 1
		 \end{pmatrix} \)


	 \paragraph{} %4
		 依题意, 系数行列式为 0. 即
		 \[ \begin{vmatrix}
				 a & -2  & 3  \\
				 1 & a+2 & 3  \\
				 2 & 1   & -1
			 \end{vmatrix} = \begin{vmatrix}
				 a+6 & 0   & 0  \\
				 7   & a+5 & 0  \\
				 2   & 1   & -1
			 \end{vmatrix} = -(a+5)(a+6) = 0 \]

		 当 \( a = -5 \) 时,

		 由 \( \begin{pmatrix}
			 -5 & -3 & 3  \\
			 1  & -3 & 3  \\
			 2  & 1  & -1
		 \end{pmatrix} \rightarrow \begin{pmatrix}
			 1 & 0 & 0  \\
			 0 & 1 & -1 \\
			 0 & 0 & 0
		 \end{pmatrix} \) 知, 通解为 \( c\begin{pmatrix}
			 0 \\
			 1 \\
			 1
		 \end{pmatrix} \)


		 当 \( a = -6 \) 时,

		 由 \( \begin{pmatrix}
			 -6 & -3 & 3  \\
			 1  & -4 & 3  \\
			 2  & 1  & -1
		 \end{pmatrix} \rightarrow \begin{pmatrix}
			 1 & 0 & -\frac{1}{4} \\
			 0 & 1 & -\frac{7}{4} \\
			 0 & 0 & 0
		 \end{pmatrix} \) 知, 通解为 \( c\begin{pmatrix}
			 1 \\
			 7 \\
			 9
		 \end{pmatrix} \)


	 \paragraph{} %5
		 \( \begin{pmatrix}
			 \xi_{1}          \\
			 \xi_{1}+2\xi_{2} \\
			 \xi_{1}+2\xi_{2}+3\xi_{3}
		 \end{pmatrix} = \begin{pmatrix}
			 1         \\
			 1 & 2     \\
			 1 & 2 & 3
		 \end{pmatrix}\begin{pmatrix}
			 \xi_{1} \\
			 \xi_{2} \\
			 \xi_{3}
		 \end{pmatrix} \)

		 由于 \( \begin{vmatrix}
			 1         \\
			 1 & 2     \\
			 1 & 2 & 3
		 \end{vmatrix} = 6 \neq 0 \). 故 \( r\begin{pmatrix}
			 \xi_{1}          \\
			 \xi_{1}+2\xi_{2} \\
			 \xi_{1}+2\xi_{2}+3\xi_{3}
		 \end{pmatrix} = r\begin{pmatrix}
			 \xi_{1} \\
			 \xi_{2} \\
			 \xi_{3}
		 \end{pmatrix} \)

		 故 \( \xi_{1}, \xi_{1}+2\xi_{2}, \xi_{1}+2\xi_{2}+3\xi_{3} \) 也是 \( Ax=0 \) 的基础解系.


	 \paragraph{} %6
		 显然 \( \beta, \alpha_{1}, \alpha_{2}, \dots, \alpha_{s} \) 与 \( \beta, \beta+\alpha_{1}, \dots, \beta+\alpha_{s} \) 可以互相线性表示, 故
		 \[ r(\beta, \alpha_{1}, \alpha_{2}, \dots, \alpha_{s}) = r(\beta, \beta+\alpha_{1}, \dots, \beta+\alpha_{s}) \]
		 又 \( \beta \) 不是 \( Ax=0 \) 的解, 故 \( \beta, \alpha_{1}, \dots, \alpha_{s} \) 线性无关,

		 \( \therefore r(\beta, \alpha_{1}, \dots, \alpha_{s}) = s+1 \)

		 \( \therefore r(\beta, \beta+\alpha_{1}, \dots, \beta+\alpha_{s}) = s+1 \)

		 故 \( r(\beta+\alpha_{1}, \dots, \beta+\alpha_{s}) = s \)

		 即 \( \beta+\alpha_{1}, \beta+\alpha_{2}, \dots, \beta+\alpha_{s} \) 线性无关.


	 \paragraph{} %7
		 由于 \( r\begin{pmatrix}
			 A & O \\
			 O & B
		 \end{pmatrix} = r(A)+r(B) \)

		 又 \( r\begin{pmatrix}
			 A & O \\
			 O & B
		 \end{pmatrix} \leq r\begin{pmatrix}
			 A & O \\
			 E & B
		 \end{pmatrix} \) 且 \( \begin{pmatrix}
			 A & O \\
			 E & B
		 \end{pmatrix} \xrightarrow{c_2-c_1B} \begin{pmatrix}
			 A & -AB \\
			 E & O
		 \end{pmatrix} \xrightarrow{r_{1}-Ar_{2}} \begin{pmatrix}
			 O & -AB \\
			 E & O
		 \end{pmatrix} \)

		 故 \( r(AB)+r(E) = r\begin{pmatrix}
			 O & -AB \\
			 E & O
		 \end{pmatrix} = r\begin{pmatrix}
			 A & O \\
			 E & B
		 \end{pmatrix} \geq r\begin{pmatrix}
			 A & O \\
			 O & B
		 \end{pmatrix} = r(A)+r(B) \)

		 \( \because AB = O \), 则 \( r(AB) = 0 \), 故 \( r(A)+r(B) \leq n \).


	 \paragraph{} %8
		 证明: 当 \( r(A) = n \) 时, \( |A| \neq 0 \), 则 \( |A^{*}| = |A|^{n-1} \neq 0, \text{ 故 } r(A^{*}) = n. \)

		 当 \( r(A) = n-1 \) 时, 则 \( A \) 中至少有一个 \( n-1 \) 阶子式不为0, 故
		 \( A^{*} \) 中至少有1个不为0的元素, 故 \( r(A^{*}) \geq 1 \)

		 又 \( AA^{*} = |A|E = O \), 故 \( r(A)+r(A^{*}) \leq n \) 即 \( r(A^{*}) \leq 1 \)

		 故 \( r(A^{*}) = 1 \)

		 当 \( r(A) < n-1 \) 时, \( A \) 的每个 \( n-1 \) 阶子式均为零, 故 \( A^{*} = O \)

		 故 \( r(A^{*}) = 0 \)


	 \paragraph{} %9
		 证明: 假设 \( A \) 不可逆

		 则由 \( AB = AC \Rightarrow A(B-C) = O \)

		 则 \( (B-C)^{\mathrm{T}}A^{\mathrm{T}} = O \), 即 \( A^{\mathrm{T}} \) 是 \( (B-C)^{\mathrm{T}}x = O \) 的解.

		 由于 \( A \) 不可逆, 则 \( r(A^{\mathrm{T}}) < n \)

		 又 \( r(A^{\mathrm{T}}) = n-r(B-C) \)

		 故 \( r(B-C) > 0 \) 则 \( B \neq C \), 矛盾!

		 故 \( A \) 可逆

 \subsection{} %B


	 \paragraph{} %1
		 设所求齐次方程组为 \( Ax = O \)

		 由 \( n - r(A) = 2 \), 且 \( n = 4 \), 知 \( r(A) = 2 \)

		 又 \( A(\alpha_{1}, \alpha_{2}) = 0 \), 则 \( \begin{pmatrix}
			 \alpha_{1}^{\mathrm{T}} \\
			 \alpha_{2}^{\mathrm{T}}
		 \end{pmatrix} A^{\mathrm{T}} = O \)

		 考虑 \( Bx = \begin{pmatrix}
			 \alpha_{1}^{\mathrm{T}} \\
			 \alpha_{2}^{\mathrm{T}}
		 \end{pmatrix} x = O \), 由 \( n - r(B) = 4 - 2 = 2 \), 知 \( Bx = O \) 的基础解系是 \( A^{\mathrm{T}} \) 的列向量

		 \( B = \begin{pmatrix}
			 \alpha_{1}^{\mathrm{T}} \\
			 \alpha_{2}^{\mathrm{T}}
		 \end{pmatrix} = \begin{pmatrix}
			 2 & 1 & -5 & 0 \\
			 1 & 1 & 1  & 1
		 \end{pmatrix} \rightarrow \begin{pmatrix}
			 1 & 0 & -6 & -1 \\
			 0 & 1 & 7  & 2
		 \end{pmatrix} \)

		 则基础解系为 \( \begin{pmatrix}
			 6  \\
			 -7 \\
			 1  \\
			 0
		 \end{pmatrix}, \begin{pmatrix}
			 1  \\
			 -2 \\
			 0  \\
			 1
		 \end{pmatrix} \)

		 故 \( A \) 可取 \( \begin{pmatrix}
			 6 & -7 & 1 & 0 \\
			 1 & -2 & 0 & 1
		 \end{pmatrix} \)

		 则所求齐次方程组为 \( \begin{cases}
			 6x_{1} - 7x_{2} + x_{3} = 0 \\
			 x_{1} - 2x_{2} + x_{4} = 0
		 \end{cases} \)


	 \paragraph{} %2
		 由 \( \alpha_{4} = \alpha_{1} + 2\alpha_{2} - \alpha_{3} \) 知 \( (\alpha_{1}, \alpha_{2}, \alpha_{3}, \alpha_{4}) \begin{pmatrix}
			 1  \\
			 2  \\
			 -1 \\
			 -1
		 \end{pmatrix} = 0 \)

		 即 \( \begin{pmatrix}
			 1  \\
			 2  \\
			 -1 \\
			 -1
		 \end{pmatrix} \) 是一组解.

		 又由于 \( \alpha_{2}, \alpha_{3}, \alpha_{4} \) 线性无关, 则 \( r(A) = 3 \)

		 故基础解系的解的个数为 \( n - r(A) = 1 \)

		 从而 \( Ax = O \) 的通解为 \( c \begin{pmatrix}
			 1  \\
			 2  \\
			 -1 \\
			 -1
		 \end{pmatrix} \)


	 \paragraph{} %3
		 基础解系中向量个数为 \( n - r(A) = 1 \)

		 又由于 \( A \) 的各行元素之和均为0, 则
		 \[ A \cdot \begin{pmatrix}
				 1      \\
				 1      \\
				 \vdots \\
				 1
			 \end{pmatrix} = \alpha_{1} + \alpha_{2} + \dots + \alpha_{n} = 0, \] 故 \( \begin{pmatrix}
			 1      \\
			 1      \\
			 \vdots \\
			 1
		 \end{pmatrix} \) 是特解

		 \( Ax = 0 \) 的通解为 \( c \begin{pmatrix}
			 1      \\
			 1      \\
			 \vdots \\
			 1
		 \end{pmatrix} \)


	 \paragraph{} %4
		 \begin{enumerate}
			 \item %(1)
			       考虑
			       \[ A = \begin{vmatrix}
					       a_{1}+b & a_{2}   & \cdots & a_{n}   \\
					       a_{1}   & a_{2}+b & \cdots & a_{n}   \\
					       \vdots  & \vdots  & \ddots & \vdots  \\
					       a_{1}   & a_{2}   & \cdots & a_{n}+b
				       \end{vmatrix} \]

			       当 $b=0$ 时, 行列式所有列对应成比例, $A=0$

			       当 $b \neq 0$ 时
			       \[ A = \begin{vmatrix}
					       a_{1}+b & a_{2}   & \cdots & a_{n}   \\
					       a_{1}   & a_{2}+b & \cdots & a_{n}   \\
					       \vdots  & \vdots  & \ddots & \vdots  \\
					       a_{1}   & a_{2}   & \cdots & a_{n}+b
				       \end{vmatrix} = \begin{vmatrix}
					       1      & a_{1}   & a_{2}   & \cdots & a_{n}   \\
					       0      & a_{1}+b & a_{2}   & \cdots & a_{n}   \\
					       0      & a_{1}   & a_{2}+b & \cdots & a_{n}   \\
					       \vdots & \vdots  & \vdots  & \ddots & \vdots  \\
					       0      & a_{1}   & a_{2}   & \cdots & a_{n}+b
				       \end{vmatrix} \]
			       \[ = \begin{vmatrix}
					       1      & a_{1}  & a_{2}  & \cdots & a_{n}  \\
					       -1     & b      & 0      & \cdots & 0      \\
					       -1     & 0      & b      & \cdots & 0      \\
					       \vdots & \vdots & \vdots & \ddots & \vdots \\
					       -1     & 0      & 0      & \cdots & b
				       \end{vmatrix} = \begin{vmatrix}
					       1      & \frac{a_{1}}{b} & \frac{a_{2}}{b} & \cdots & \frac{a_{n}}{b} \\
					       -1     & 1               & 0               & \cdots & 0               \\
					       -1     & 0               & 1               & \cdots & 0               \\
					       \vdots & \vdots          & \vdots          & \ddots & \vdots          \\
					       -1     & 0               & 0               & \cdots & 1
				       \end{vmatrix} = 1 + \frac{\displaystyle \sum_{i=1}^{n} a_{i}}{b} \]

			       则当 $\sum_{i=1}^{n} a_{i} = -b$ 时, $A=0$,

			       综上, 当 $\sum_{i=1}^{n} a_{i} \neq -b$ 且 $b \neq 0$ 时, 方程组仅有零解.
			 \item %(2)
			       当 $b=0$ 时, 则 $ \begin{pmatrix}
					       a_{1}  & a_{2}  & \cdots & a_{n}  \\
					       a_{1}  & a_{2}  & \cdots & a_{n}  \\
					       \vdots & \vdots & \ddots & \vdots \\
					       a_{1}  & a_{2}  & \cdots & a_{n}
				       \end{pmatrix} \rightarrow \begin{pmatrix}
					       a_{1}  & a_{2}  & \cdots & a_{n}  \\
					       0      & 0      & \cdots & 0      \\
					       \vdots & \vdots & \ddots & \vdots \\
					       0      & 0      & \cdots & 0
				       \end{pmatrix} $

			       则不妨 $a_{1} \neq 0$, 通解为
			       \[ c_{1}\begin{pmatrix}
					       -a_{2} \\
					       a_{1}  \\
					       0      \\
					       \vdots \\
					       0
				       \end{pmatrix} + c_{2}\begin{pmatrix}
					       -a_{3} \\
					       0      \\
					       a_{1}  \\
					       \vdots \\
					       0
				       \end{pmatrix} + \cdots + c_{n-1}\begin{pmatrix}
					       -a_{n} \\
					       0      \\
					       0      \\
					       \vdots \\
					       a_{1}
				       \end{pmatrix} \]

			       当 $\sum_{i=1}^{n} a_{i} = -b$ 时, 则 $ \begin{pmatrix}
					       a_{1}+b & a_{2}   & \cdots & a_{n}   \\
					       a_{1}   & a_{2}+b & \cdots & a_{n}   \\
					       \vdots  & \vdots  & \ddots & \vdots  \\
					       a_{1}   & a_{2}   & \cdots & a_{n}+b
				       \end{pmatrix}\to
				       \begin{pmatrix}
					       a      & a_2 & \cdots & a_n        \\
					       -b     & b   &        &            \\
					       -b     &     & b      &            \\
					       \vdots &     &        & \ddots     \\
					       -b     &     &        &        & b
				       \end{pmatrix}
				       \rightarrow \begin{pmatrix}
					       1 &   &        &   & -1     \\
					         & 1 &        &   & -1     \\
					         &   & \ddots &   & \vdots \\
					         &   &        & 1 & -1     \\
					         &   &        &   & 0
				       \end{pmatrix} $

			       则通解为 $c\begin{pmatrix}
					       1      \\
					       1      \\
					       \vdots \\
					       1
				       \end{pmatrix} $
		 \end{enumerate}


 \subsection{} %C


	 \paragraph{} %1
		 \begin{enumerate}
			 \item %(1)
			       由 $ \begin{pmatrix}
					       1 & 1 & 0 & 0  \\
					       0 & 1 & 0 & -1
				       \end{pmatrix} \rightarrow \begin{pmatrix}
					       1 & 0 & 0 & 1  \\
					       0 & 1 & 0 & -1
				       \end{pmatrix} $

			       则 $x_{3}, x_{4}$ 为自由未知量, 取 $ \begin{pmatrix}
					       x_{3} \\
					       x_{4}
				       \end{pmatrix} $ 为 $ \begin{pmatrix}
					       1 \\
					       0
				       \end{pmatrix} $, $ \begin{pmatrix}
					       0 \\
					       1
				       \end{pmatrix} $

			       则 $ \eta_{1} = \begin{pmatrix}
					       0 \\
					       0 \\
					       1 \\
					       0
				       \end{pmatrix} ,\ \eta_{2} = \begin{pmatrix}
					       \alpha_{1} \\
					       -1         \\
					       0          \\
					       1
				       \end{pmatrix} $ 则 $ x = c_{1}\begin{pmatrix}
					       0 \\
					       0 \\
					       1 \\
					       0
				       \end{pmatrix} + c_{2}\begin{pmatrix}
					       -1\alpha_{1} \\
					       1            \\
					       0            \\
					       1
				       \end{pmatrix} $
			 \item %(2)
			       依题意, 由 \( \begin{pmatrix}
				       -c_{2} \\
				       c_{2}  \\
				       c_{1}  \\
				       c_{2}
			       \end{pmatrix} = \begin{pmatrix}
				       -k_{2}       \\
				       k_{1}+2k_{2} \\
				       k_{1}+2k_{2} \\
				       k_{2}
			       \end{pmatrix} \Rightarrow \begin{cases}
				       c_{1} = c_{2} \\
				       c_{1} = k_{2} \\
				       k_{1} = -k_{2}
			       \end{cases}\)

			       则 I 与 II 有公共解$c\begin{pmatrix}
					       -1 \\
					       1  \\
					       1  \\
					       1
				       \end{pmatrix}$
		 \end{enumerate}


	 \paragraph{} %2
		 记 $A = (\alpha_{1}, \alpha_{2}, \dots, \alpha_{n})$ $B = (\beta_{1}, \beta_{2}, \dots, \beta_{n})$

		 若 $(\alpha_{1}, \alpha_{2}, \dots, \alpha_{n}, \beta_{1}, \beta_{2}, \dots, \beta_{n})x = 0$ 有解,

		 则即非零公共解.

		 由于 $(A\ B)$ 的列向量组由 A, B 的列向量扩充而成

		 故 $(A\ B)$ 的列向量组可以由 A 的列极大线性无关组, B 的列极大无关组线性表示, 则
		 \[ r(A\ B) \leq r(A) + r(B) < n \]
		 从而 $(A\ B)x = 0$ 有非零解,

		 即 $Ax=0$ 与 $Bx=0$ 有非零公共解.


\section{4.2}

 \subsection{} %A


	 \paragraph{} %1
		 \begin{enumerate}
			 \item %(1)
			       由 \( \left(\begin{array}{cccc:c}
					       2 & 1  & 3 & 3 & 1 \\
					       1 & 1  & 1 & 2 & 0 \\
					       1 & -2 & 4 & 1 & 4
				       \end{array}\right) \rightarrow \left(\begin{array}{cccc:c}
					       1 & 0 & 2  & 0 & \frac{1}{2}  \\
					       0 & 1 & -1 & 0 & -\frac{3}{2} \\
					       0 & 0 & 0  & 1 & \frac{1}{2}
				       \end{array}\right) \)

			       则取自由未知量为 0, 得特解 \( \eta^{*} = \frac{1}{2}\begin{pmatrix}
				       1  \\
				       -3 \\
				       0  \\
				       1
			       \end{pmatrix} \)

			       又基础解系为 \( \eta = \begin{pmatrix}
				       -2 \\
				       1  \\
				       1  \\
				       0
			       \end{pmatrix} \) 故解为 \( x = \frac{1}{2}\begin{pmatrix}
				       1  \\
				       -3 \\
				       0  \\
				       1
			       \end{pmatrix} + c\begin{pmatrix}
				       -2 \\
				       1  \\
				       1  \\
				       0
			       \end{pmatrix} \)
			 \item %(2)
			       由 \( \left(\begin{array}{ccc:c}
					       1 & 1  & -1 & -1 \\
					       2 & -5 & 3  & 2  \\
					       7 & -7 & 2  & 1
				       \end{array}\right)
			       \to
			       \left(\begin{array}{ccc:c}
					       1 & 0 & 0 & -\frac{3}{7} \\
					       0 & 1 & 0 & -\frac{4}{7} \\
					       0 & 0 & 1 & 0
				       \end{array}\right) \)

			       故解为 \( x = \frac{1}{7}\begin{pmatrix}
				       -3 \\
				       -4 \\
				       0
			       \end{pmatrix} \)
			 \item %(3)
			       由 \( \left(\begin{array}{ccc:c}
					       1  & 0 & 3 & 2 \\
					       -1 & 3 & 0 & 1 \\
					       2  & 1 & 7 & 5
				       \end{array}\right)
			       \to \)\(\left(\begin{array}{ccc:c}
					       1 & 0 & 3 & 2 \\
					       0 & 1 & 1 & 1 \\
					       0 & 0 & 0 & 0
				       \end{array}\right) \)

			       自由未知量为 \( x_{3} \)

			       取 \( x_{3} = 0 \), 得 \( \eta^{*} = \begin{pmatrix}
				       2 \\
				       1 \\
				       0
			       \end{pmatrix} \) 取 \( x_{3} = 1 \), 得 \( \eta = \begin{pmatrix}
				       -3 \\
				       -1 \\
				       1
			       \end{pmatrix} \)

			       则 \( x = \begin{pmatrix}
				       2 \\
				       1 \\
				       0
			       \end{pmatrix} + c\begin{pmatrix}
				       -3 \\
				       -1 \\
				       1
			       \end{pmatrix} \)
			 \item %(4)
			       由 \( \left(\begin{array}{cccc:c}
					       1 & 0  & 3  & 1 & 2  \\
					       1 & -3 & 0  & 1 & -1 \\
					       2 & 1  & 7  & 2 & 5  \\
					       4 & 2  & 14 & 0 & 6
				       \end{array}\right)
			       \to
			       \left(\begin{array}{cccc:c}
					       1 & 0 & 3 & 0 & 1 \\
					       0 & 1 & 1 & 0 & 1 \\
					       0 & 0 & 0 & 1 & 1 \\
					       0 & 0 & 0 & 0 & 0
				       \end{array}\right) \)

			       自由未知量为 \( x_{3} \)

			       取 \( x_{3} = 0 \), 得 \( \eta^{*} = \begin{pmatrix}
				       1 \\
				       0 \\
				       0 \\
				       1
			       \end{pmatrix} \) 取 \( x_{3} = 1 \), 得 \( \eta = \begin{pmatrix}
				       -3 \\
				       -1 \\
				       1  \\
				       0
			       \end{pmatrix} \)

			       故 \( x = \begin{pmatrix}
				       1 \\
				       1 \\
				       0 \\
				       1
			       \end{pmatrix} + c\begin{pmatrix}
				       -3 \\
				       -1 \\
				       1  \\
				       0
			       \end{pmatrix} \)
		 \end{enumerate}


	 \paragraph{} %2
		 由 \( \begin{pmatrix}
			 1 & 3 & 2 & 1  & 1  \\
			 0 & 1 & a & -a & -1 \\
			 1 & 2 & 0 & 3  & 1
		 \end{pmatrix} \rightarrow \begin{pmatrix}
			 1 & 3 & 2    & 1    & 1  \\
			 0 & 1 & a    & -a   & -1 \\
			 0 & 0 & 2-2a & 2-2a & 0
		 \end{pmatrix} \)

		 则当 \( a = 2 \) 时, 无解.

		 当 \( a \neq 2 \) 时, 通解为 \( \frac{1}{a-2}\begin{pmatrix}
			 7a-10 \\
			 2-2a  \\
			 1     \\
			 0
		 \end{pmatrix} + c\begin{pmatrix}
			 -3 \\
			 0  \\
			 1  \\
			 1
		 \end{pmatrix} \)


	 \paragraph{} %3
		 系数行列式 \[ A = \begin{vmatrix}
				 1  & 1  & a \\
				 -1 & a  & 1 \\
				 1  & -1 & 2
			 \end{vmatrix} = (a+1)\begin{vmatrix}
				 1 & 1 & a   \\
				   & 1 & 1   \\
				   &   & 4-a
			 \end{vmatrix} = (a+1)(4-a) \]

		 则当 \( a \neq 4 \) 且 \( a \neq -1 \) 时, 方程组为唯一解.

		 当 \( a = 4 \) 时, 方程组有无穷多解, 由于
		 \[ \left(\begin{array}{ccc:c}
					 1  & 1  & 4 & 4  \\
					 -1 & 4  & 1 & 16 \\
					 1  & -1 & 2 & -4
				 \end{array}\right) \rightarrow \left(\begin{array}{ccc:c}
					 1 & 0 & 3 & 0 \\
					 0 & 1 & 1 & 4 \\
					 0 & 0 & 0 & 0
				 \end{array}\right) \]
		 故通解为 \(\begin{pmatrix}
			 0 \\
			 4 \\
			 0
		 \end{pmatrix} + c\begin{pmatrix}
			 -3 \\
			 -1 \\
			 1
		 \end{pmatrix} \)

		 当 \( a = -1 \) 时, 由于 \( \left(\begin{array}{ccc:c}
				 1  & 1  & -1 & 4  \\
				 -1 & -1 & 1  & -1 \\
				 1  & -1 & 2  & -4
			 \end{array}\right) \rightarrow \left(\begin{array}{ccc:c}
				 1 & 0 & \frac{1}{2}  & 0 \\
				 0 & 1 & -\frac{3}{2} & 0 \\
				 0 & 0 & 0            & 1
			 \end{array}\right) \)

		 故为方程组无解.


	 \paragraph{} %4
		 由 \( \left(\begin{array}{ccccc:c}
				 1 & 1 & 1 & 1 & 1  & 1 \\
				 0 & 1 & 2 & 2 & 6  & 3 \\
				 3 & 2 & 1 & 1 & -3 & a \\
				 5 & 4 & 3 & 3 & -1 & b
			 \end{array}\right)
		 \to
		 \left(\begin{array}{ccccc:c}
				 1 & 1 & 1 & 1 & 1 & 1   \\
				   & 1 & 2 & 2 & 6 & 3   \\
				   &   &   &   &   & a   \\
				   &   &   &   &   & a-2
			 \end{array}\right) \)

		 则由方程组有解得
		 \( \begin{cases}
			 a = 0 \\
			 b = 2
		 \end{cases} \)

		 则 \( \rightarrow \left(\begin{array}{ccccc:c}
				 1 & 0 & -1 & -1 & -5 & -2 \\
				 0 & 1 & 2  & 2  & 6  & 3  \\
				 0 & 0 & 0  & 0  & 0  & 0
			 \end{array}\right) \) 自由未知量为 \( x_{3}, x_{4}, x_{5} \),

		 取
		 \( \begin{pmatrix}
			 x_{3} \\
			 x_{4} \\
			 x_{5}
		 \end{pmatrix} = \begin{pmatrix}
			 1 \\
			 0 \\
			 0
		 \end{pmatrix}, \begin{pmatrix}
			 0 \\
			 1 \\
			 0
		 \end{pmatrix}, \begin{pmatrix}
			 0 \\
			 0 \\
			 1
		 \end{pmatrix}\)

		 得 \(\eta^{*} = \begin{pmatrix}
			 -2 \\
			 3  \\
			 0  \\
			 0  \\
			 0
		 \end{pmatrix} \)

		 取 \( \begin{pmatrix}
			 x_{3} \\
			 x_{4} \\
			 x_{5}
		 \end{pmatrix} = \begin{pmatrix}
			 1 \\
			 0 \\
			 0
		 \end{pmatrix}, \begin{pmatrix}
			 0 \\
			 1 \\
			 0
		 \end{pmatrix}, \begin{pmatrix}
			 0 \\
			 0 \\
			 1
		 \end{pmatrix} \)

		 得基础解系 \( \eta_{1} = \begin{pmatrix}
			 -1 \\
			 -2 \\
			 1  \\
			 0  \\
			 0
		 \end{pmatrix} \), \( \eta_{2} = \begin{pmatrix}
			 1  \\
			 -2 \\
			 0  \\
			 1  \\
			 0
		 \end{pmatrix} \), \( \eta_{3} = \begin{pmatrix}
			 5  \\
			 -6 \\
			 0  \\
			 0  \\
			 1
		 \end{pmatrix} \)

		 则 \( x = \begin{pmatrix}
			 -2 \\
			 3  \\
			 0  \\
			 0  \\
			 0
		 \end{pmatrix} + c_{1}\begin{pmatrix}
			 -1 \\
			 -2 \\
			 1  \\
			 0  \\
			 0
		 \end{pmatrix} + c_{2}\begin{pmatrix}
			 1  \\
			 -2 \\
			 0  \\
			 1  \\
			 0
		 \end{pmatrix} + c_{3}\begin{pmatrix}
			 5  \\
			 -6 \\
			 0  \\
			 0  \\
			 1
		 \end{pmatrix} \)


	 \paragraph{} %5
		 考虑 \( A = (\alpha_{1}, \alpha_{2}, \alpha_{3}) \), 研究 \( Ax = \beta \) 的解.
		 \begin{enumerate}
			 \item %(1)
			       即 \( Ax = \beta \) 有唯一解, 则 \( |A| \neq 0 \)

			       即 \( \begin{vmatrix}
				       1  & \frac{1}{2} & \frac{1}{2} \\
				       a  & -2          & 1           \\
				       10 & 5           & 4
			       \end{vmatrix} = \begin{vmatrix}
				       1 & \frac{1}{2}    & \frac{1}{2}   \\
				       0 & -2-\frac{a}{2} & 1-\frac{a}{2} \\
				       0 & 0              & -1
			       \end{vmatrix} = (2+\frac{a}{2}) \neq 0 \)

			       则 \( a \neq -4 \). 此时 \( \beta \) 由 \( \alpha_{1}, \alpha_{2}, \alpha_{3} \) 线性表示, 且表示法唯一.
			 \item %(2)
			       即 \( Ax = \beta \) 无解, 则 \( |A| = 0 \Rightarrow a = -4 \)

			       由于 \( \left(\begin{array}{ccc:c}
					       -4 & -2 & 1 & 1 \\
					       2  & 1  & 1 & b \\
					       10 & 5  & 4 & c
				       \end{array}\right)
			       \to
			       \left(\begin{array}{ccc:c}
					       1 & \frac{1}{2} & \frac{1}{2} & \frac{b}{2} \\
					       0 & 0           & 3           & 1+2b        \\
					       0 & 0           & -1          & c-5b
				       \end{array}\right)
			       \to
			       \left(\begin{array}{ccc:c}
					       1 & \frac{1}{2} & \frac{1}{2} & \frac{b}{2} \\
					       0 & 0           & -1          & c-5b        \\
					       0 & 0           & 0           & 3c-13b+1
				       \end{array}\right) \)

			       故当 \( 3c-13b+1 \neq 0 \), \( \beta \) 不能由 \( \alpha_{1}, \alpha_{2}, \alpha_{3} \) 线性表示. (此时 \( a = -4 \))
			 \item %(3)
			       即 \( Ax = \beta \) 有无穷多解, 则 \( a = -4 \) 且 \( 3c-13b+1 = 0 \)

			       则 \( \left(\begin{array}{ccc:c}
					       1 & \frac{1}{2} & \frac{1}{2} & \frac{b}{2} \\
					       0 & 0           & -1          & c-5b        \\
					       0 & 0           & 0           & 0
				       \end{array}\right) \rightarrow \left(\begin{array}{ccc:c}
					       1 & \frac{1}{2} & 0 & \frac{b+1}{6}  \\
					       0 & 0           & 1 & \frac{1+2b}{3} \\
					       0 & 0           & 0 & 0
				       \end{array}\right) \)

			       自由未知量为 \( x_{2} \).

			       令 \( x_{2} = 0 \Rightarrow \eta^{*} = \begin{pmatrix}
				       \frac{b-1}{6} \\
				       0             \\
				       \frac{1+2b}{3}
			       \end{pmatrix} \)

			       令 \( x_{2} = 1 \Rightarrow \eta = \begin{pmatrix}
				       -\frac{1}{2} \\
				       1            \\
				       0
			       \end{pmatrix} \)

			       则 \( \begin{pmatrix}
				       x_{1} \\
				       x_{2} \\
				       x_{3}
			       \end{pmatrix} = \eta^{*} + k\eta = \begin{pmatrix}
				       \frac{b-1+3k}{6} \\
				       k                \\
				       \frac{1+2b}{3}
			       \end{pmatrix} \)

			       故 \( \beta = \frac{b-1+3k}{6}\alpha_{1} + k\alpha_{2} + \frac{2b+1}{3}\alpha_{3} \)
		 \end{enumerate}


	 \paragraph{C}
		 A 为特解 + 通解,

		 B 中 \( \frac{1}{2}\eta_{1} + k_{1}\xi_{1} \) 是 \( Ax = \frac{1}{2}\beta \) 的解, \(\frac{1}{2}\eta_2+k_2(\xi_1+\xi_2)\) 是 \(Ax+\frac{1}{2}\beta\) 的解.

		 故 \( k_{1}\xi_{1} + k_{2}(\xi_{1}+\eta_{2}) + k_{2}(\eta_{1}+\eta_{2}) \) 是 \( Ax = \beta \) 的解.

		 C 应是 \( Ax = 2\beta \) 的解, D 同 B 理可证.


	 \paragraph{} %6
		 由 \( \left(\begin{array}{ccc:c}
				 1 & 2 & 1   & 1 \\
				 2 & 3 & a+2 & 3 \\
				 1 & a & -2  & 0
			 \end{array}\right)
		 \to
		 \left(\begin{array}{ccc:c}
				 1 & 2   & 1  & 1  \\
				 0 & -1  & a  & 1  \\
				 0 & a-2 & -3 & -1
			 \end{array}\right)
		 \to
		 \left(\begin{array}{ccc:c}
				 1 & 2  & 1          & 1   \\
				 0 & -1 & a          & 1   \\
				 0 & 0  & (a-3)(a+1) & a-3
			 \end{array}\right) \)

		 则 \( a = -1 \) 时无解.


	 \paragraph{} %7
		 证明: \( \lambda_{1}\eta_{1} + \lambda_{2}\eta_{2} + \cdots + \lambda_{s}\eta_{s} = (1-\lambda_{2}-\lambda_{3}-\cdots-\lambda_{s})\eta_{1} + \lambda_{2}\eta_{2} + \cdots + \lambda_{s}\eta_{s} = \eta_{1} + \sum_{i=2}^{s}\lambda_{i}(\eta_{i}-\eta_{1}) \)

		 因为 \( \eta_{i}-\eta_{1} \) 是导出组的解, 故 \( \sum_{i=2}^{s}\lambda_{i}(\eta_{i}-\eta_{1}) \) 也是导出组的解.

		 又 \( \eta_{1} \) 是原方程一个特解, 故 \( \lambda_{1}\eta_{1} + \lambda_{2}\eta_{2} + \cdots + \lambda_{s}\eta_{s} \) 也是 \( Ax = \beta \) 的解.


	 \paragraph{} %8
		 \begin{enumerate}
			 \item %(1)
			       设 \( k_{1}\xi_{1} + k_{2}\xi_{2} + \cdots + k_{n-r}\xi_{n-r} + k_{n-r+1}\eta = 0 \)

			       则 \( k_{1}A\xi_{1} + k_{2}A\xi_{2} + \cdots + k_{n-r}A\xi_{n-r} + k_{n-r+1}A\eta = 0 \)

			       即 \( k_{n-r+1}\beta = 0 \)

			       故 \( k_{n-r+1} = 0 \)

			       则 \( \xi_{1}, \xi_{2}, \dots, \xi_{n-r}, \eta \) 线性无关.
			 \item %(2)
			       设 \( k_{1}(\xi_{1}+\eta) + k_{2}(\xi_{2}+\eta) + \cdots + k_{n-r}(\xi_{n-r}+\eta) + k_{n-r+1}\eta = 0 \)

			       即 \( k_{1}\xi_{1} + k_{2}\xi_{2} + \cdots + k_{n-r}\xi_{n-r} + (k_{1}+k_{2}+\cdots+k_{n-r+1})\eta = 0 \)

			       则由(1)知 \( k_{1}+k_{2}+\cdots+k_{n-r+1} = 0 \),

			       则 \( k_{1}\xi_{1} + k_{2}\xi_{2} + \cdots + k_{n-r}\xi_{n-r} = 0 \)
			       又 \( \because \xi_{1}, \xi_{2}, \dots, \xi_{n-r} \) 线性无关

			       故 \( k_{1} = k_{2} = \cdots = k_{n-r} = 0 \)

			       \( \Rightarrow k_{n-r+1} = 0 \)

			       故 \( \xi_{1}+\eta, \xi_{2}+\eta, \dots, \xi_{n-r}+\eta, \eta \) 线性无关.
		 \end{enumerate}


 \subsection{} %B


	 \paragraph{} %1
		 由于 \( \left(\begin{array}{cccc:c}
				 1 & 1  & -2 & 3  & 0  \\
				 2 & 1  & -6 & 4  & -1 \\
				 3 & 2  & a  & 7  & -1 \\
				 1 & -1 & -b & -2 & b
			 \end{array} \right)
		 \to
		 \left(\begin{array}{cccc:c}
				 1 & 1  & -2  & 3  & 0  \\
				 0 & -1 & -2  & -2 & -1 \\
				 0 & -1 & a+6 & -2 & -1 \\
				 0 & -2 & -4  & -4 & b
			 \end{array} \right)
		 \to
		 \left(\begin{array}{cccc:c}
				 1 & 0 & -4  & 1 & -1  \\
				 0 & 1 & 2   & 2 & 1   \\
				 0 & 0 & a+8 & 0 & 0   \\
				 0 & 0 & 0   & 0 & b+2
			 \end{array} \right) \)



		 故当 \( b = -2 \) 时方程组有解.
		 \begin{enumerate}
			 \item %(1)
			       当 \( b = -2 \) 且 \( a = -8 \) 时, 自由未知量为 \( x_{1}, x_{4} \),

			       取 \( \begin{pmatrix}
				       x_{1} \\
				       x_{4}
			       \end{pmatrix} = \begin{pmatrix}
				       0 \\
				       0
			       \end{pmatrix} \), 得特解 \( \begin{pmatrix}
				       -1 \\
				       1  \\
				       0  \\
				       0
			       \end{pmatrix} \)

			       取 \( \begin{pmatrix}
				       x_{1} \\
				       x_{4}
			       \end{pmatrix} = \begin{pmatrix}
				       1 \\
				       0
			       \end{pmatrix} \), \( \begin{pmatrix}
				       0 \\
				       1
			       \end{pmatrix} \), 得基础解系
			       \[ \eta_{1} = \begin{pmatrix}
					       4  \\
					       -2 \\
					       1  \\
					       1
				       \end{pmatrix}, \  \eta_{2} = \begin{pmatrix}
					       -1 \\
					       -2 \\
					       0  \\
					       1
				       \end{pmatrix} \]

			       故此时 \( x = \begin{pmatrix}
				       -1 \\
				       1  \\
				       0  \\
				       0
			       \end{pmatrix} + c_{1}\begin{pmatrix}
				       4  \\
				       -2 \\
				       1  \\
				       0
			       \end{pmatrix} + c_{2}\begin{pmatrix}
				       -1 \\
				       -2 \\
				       0  \\
				       1
			       \end{pmatrix} \)
		 \end{enumerate}


	 \paragraph{} %2
		 \begin{enumerate}
			 \item %(1)
			       \( X \) 应为 3 行 2 列矩阵, 才满足矩阵乘法.
			 \item %(2)
			       由 \( \left(\begin{array}{ccc:c}
					       1 & 1 & 3 & 1 \\
					       2 & 1 & 4 & s \\
					       1 & 0 & 1 & 0
				       \end{array}\right)
			       \rightarrow
			       \left(\begin{array}{ccc:c}
					       1 & 0 & 1 & s-1 \\
					       0 & 1 & 2 & 2-s \\
					       0 & 0 & 0 & 1-s
				       \end{array}\right) \) 知当 \( s \leq 1 \) 时, \( \begin{pmatrix}
				       2 & 1 & 3 \\
				       2 & 1 & 4 \\
				       1 & 0 & 1
			       \end{pmatrix}\begin{pmatrix}
				       x_{1} \\
				       x_{2} \\
				       x_{3}
			       \end{pmatrix} = \begin{pmatrix}
				       1 \\
				       s \\
				       0
			       \end{pmatrix} \) 有解.

			       此时 \( x_{3} \) 为自由未知量,

			       取 \( x_{3} = 0 \), 得 \( \eta^{*} = \begin{pmatrix}
				       0 \\
				       1 \\
				       0
			       \end{pmatrix} \), 又基础解系 \( \eta = \begin{pmatrix}
				       -1 \\
				       -2 \\
				       1
			       \end{pmatrix} \)

			       故通解为 \( x = \begin{pmatrix}
				       0 \\
				       1 \\
				       0
			       \end{pmatrix} + c_{1}\begin{pmatrix}
				       -1 \\
				       -2 \\
				       1
			       \end{pmatrix} \)

			       由 \( \left(\begin{array}{ccc:c}
					       1 & 1 & 3 & 2 \\
					       2 & 1 & 4 & 1 \\
					       1 & 0 & 1 & t
				       \end{array}\right)
			       \to
			       \left(\begin{array}{ccc:c}
					       1 & 0 & 1 & -1  \\
					       0 & 1 & 2 & 3   \\
					       0 & 0 & 0 & t+1
				       \end{array}\right) \) 知当 \( t = -1 \) 时, \( \begin{pmatrix}
				       2 & 1 & 3 \\
				       2 & 1 & 4 \\
				       1 & 0 & 1
			       \end{pmatrix}\begin{pmatrix}
				       x_{1} \\
				       x_{2} \\
				       x_{3}
			       \end{pmatrix} = \begin{pmatrix}
				       2 \\
				       1 \\
				       t
			       \end{pmatrix} \) 有解.

			       此时 \( x_{3} \) 为自由未知量, 取 \( x_{3} = 0 \), 可得 \( \eta^{*} = \begin{pmatrix}
				       -1 \\
				       3  \\
				       0
			       \end{pmatrix} \), 又基础解系 \( \eta = \begin{pmatrix}
				       -1 \\
				       -2 \\
				       1
			       \end{pmatrix} \)

			       故通解为 \( x = \begin{pmatrix}
				       -1 \\
				       3  \\
				       0
			       \end{pmatrix} + c_{2}\begin{pmatrix}
				       -1 \\
				       -2 \\
				       1
			       \end{pmatrix} \)
			 \item %(3)
			       \( \begin{pmatrix}
				       0 & -1 \\
				       1 & 3  \\
				       0 & 0
			       \end{pmatrix} \)
		 \end{enumerate}


	 \paragraph{} %3
		 \begin{enumerate}
			 \item %(1)
			       由 \( \left(\begin{array}{cccc:c}
					       1 & 1  & 0  & -2 & -6 \\
					       4 & -1 & -1 & -1 & 1  \\
					       3 & -1 & -1 & 0  & 3
				       \end{array} \right) \rightarrow
			       \left(\begin{array}{cccc:c}
					       1 & 0 & 0 & -1 & -2 \\
					       0 & 1 & 0 & -1 & -4 \\
					       0 & 0 & 1 & -2 & -5
				       \end{array}\right) \)

			       取 \( x_{4} \) 为自由未知量, 令 \( x_{4} = 0 \), 得特解 \( \eta^{*} = \begin{pmatrix}
				       -2 \\
				       -4 \\
				       -5 \\
				       0
			       \end{pmatrix} \)

			       令 \( x_{4} = 1 \), 得基础解系 \( \eta = \begin{pmatrix}
				       1 \\
				       1 \\
				       2 \\
				       1
			       \end{pmatrix} \)

			       则通解为 \( \begin{pmatrix}
				       -2 \\
				       -4 \\
				       -5 \\
				       0
			       \end{pmatrix} + c\begin{pmatrix}
				       1 \\
				       1 \\
				       2 \\
				       1
			       \end{pmatrix} \)
			 \item %(2)
			       由 \[ \left(\begin{array}{cccc:c}
						       1 & m & -1 & -1 & -5   \\
						       0 & n & -1 & -2 & -11  \\
						       0 & 0 & 1  & -2 & -t+1
					       \end{array}\right)
				       \rightarrow
				       \left(\begin{array}{cccc:c}
						       1 & 0 & 0 & -3+\frac{4m}{n} & -7+\frac{24m}{n} \\
						       0 & 1 & 0 & -\frac{4}{n}    & -\frac{12}{n}+t  \\
						       0 & 0 & 1 & -2              & -t+1
					       \end{array}\right) \]

			       则 \[ \left(\begin{array}{cccc:c}
						       1 & 0 & 0 & -3+\frac{4m}{n} & -7+\frac{24m}{n} \\
						       0 & 1 & 0 & -\frac{4}{n}    & -\frac{12}{n}+t  \\
						       0 & 0 & 1 & -2              & -t+1
					       \end{array}\right) =
				       \left(\begin{array}{cccc:c}
						       1 & 0 & 0 & -1 & -2 \\
						       0 & 1 & 0 & -1 & -4 \\
						       0 & 0 & 1 & -2 & -5
					       \end{array} \right) \]

			       \( \Rightarrow m = 2 \), \( n = 4 \), \( t = 6 \).
		 \end{enumerate}


	 \paragraph{} %4
		 \( A = \alpha\beta^{T} = \begin{pmatrix}
			 1 \\
			 2 \\
			 1
		 \end{pmatrix}\begin{pmatrix}
			 1 & \frac{1}{2} & 0
		 \end{pmatrix} = \begin{pmatrix}
			 1 & \frac{1}{2} & 0 \\
			 2 & 1           & 0 \\
			 1 & \frac{1}{2} & 0
		 \end{pmatrix} \newline B = \beta^{T}\alpha = \begin{pmatrix}
			 1 & \frac{1}{2} & 0
		 \end{pmatrix}\begin{pmatrix}
			 1 \\
			 2 \\
			 1
		 \end{pmatrix} = 2 \)

		 则 \( 2B^{2}A^{2}X = A^{4}X + B^{4}X + Y \)

		 \( \Leftrightarrow 8A^{2}X - A^{4}X - 16X = Y \)

		 又 \( A^{2} = \alpha\beta^{T}\alpha\beta^{T} = 2\alpha\beta^{T} = 2A \), 故 \( A^{4} = 2^{3}A = 8A \)

		 则 \( (8A - 16E)X = Y \), 故 X 是该方程组的解.

		 又 \( 8A - 16E = \begin{pmatrix}
			 -8 & 4  & 0   \\
			 16 & -8 & 0   \\
			 8  & 4  & -16
		 \end{pmatrix} \)

		 由 \( \left(\begin{array}{ccc:c}
				 -8 & 4  & 0   & 0 \\
				 16 & -8 & 0   & 0 \\
				 8  & 4  & -16 & 8
			 \end{array}\right) \rightarrow
		 \left(\begin{array}{ccc:c}
				 1 & 0 & -1 & \frac{1}{2} \\
				 0 & 1 & -2 & 1           \\
				 0 & 0 & 0  & 0
			 \end{array}\right) \), 自由未知量为 \( x_{3} \).

		 取 \( x_{3} = 0 \), 则 \( \eta^{*} = \begin{pmatrix}
			 \frac{1}{3} \\
			 \frac{1}{3} \\
			 0
		 \end{pmatrix} \), 取 \( x_{3} = 1 \), 得基础解系 \( \eta=\begin{pmatrix}
			 1 \\
			 2 \\
			 1
		 \end{pmatrix} \)

		 则 \( X = \begin{pmatrix}
			 \frac{1}{2} \\
			 1           \\
			 0
		 \end{pmatrix} + c\begin{pmatrix}
			 1 \\
			 2 \\
			 1
		 \end{pmatrix} \)


	 \paragraph{} %5
		 \( A = (\alpha_{1}, \alpha_{2}, \alpha_{3}, \beta_{1}, \beta_{2}, \beta_{3}) = \begin{pmatrix}
			 1 & 1 & 1   & 1   & 2   & 2   \\
			 0 & 1 & -1  & 2   & 1   & 1   \\
			 2 & 3 & a+2 & a+3 & a+6 & a+4
		 \end{pmatrix} \)

		 \( \rightarrow \begin{pmatrix}
			 1 & 0 & 2   & -1  & 1   & 1   \\
			 0 & 1 & -1  & 2   & 1   & 1   \\
			 0 & 0 & a-1 & a-1 & a+1 & a-1
		 \end{pmatrix} \)

		 若 \( a = -1 \), 则 \( \beta_{1}, \beta_{3} \) 不能由 \( \alpha_{1}, \alpha_{2}, \alpha_{3} \) 线性表示, 此时向量组 I 与向量组 II 不等价;

		 若 \( a \neq -1 \), 则 \( \alpha_{1}, \alpha_{2}, \alpha_{3} \) 线性无关, \( \beta_{1}, \beta_{2}, \beta_{3} \) 线性无关, 此时 I 与 II 等价.


	 \paragraph{} %6
		 依题意, \( \eta_{n-r+1}-\eta_{1}, \eta_{n-r}-\eta_{1}, \dots, \eta_{s}-\eta_{1} \) 是 \( Ax=0 \) 的 \( r \) 个线性无关解

		 则 \( Ax=\beta \) 的通解为
		 \begin{align*}
			 x & = \eta_{1} + k_{1}(\eta_{n-r+1} - \eta_{1}) + \cdots + k_{r}(\eta_{s} - \eta_{1})                            \\
			   & = \left(1 - \sum_{i=1}^{r} k_{i}\right)\eta_{1} + k_{1}\eta_{2} + k_{2}\eta_{3} + \cdots + k_{r}\eta_{n-r+1}
		 \end{align*}

		 记 \( \lambda_{1} = 1-\sum_{i=1}^{r}k_{i} \), \( \lambda_i=k_{r+2-i} \), 则
		 \[ \sum_{i=1}^{n-r+1}\lambda_{i} = 1 \]

		 故 \( Ax=\beta \) 的通解为 \( \lambda_{1}\eta_{1} + \lambda_{2}\eta_{2} + \cdots + \lambda_{n-r+1}\eta_{n-r+1} \), 其中 \( \lambda_{i} \) 满足 \( \sum_{i=1}^{n-r+1}\lambda_{i} = 1 \).
