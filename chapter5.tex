\section{5.1}
 \subsection{} %A 
     \paragraph{} %1
         \begin{enumerate}
             \item %(1)
                   \( A \) 的特征多项式为
                   \[ f_{A}(\lambda) = |\lambda E - A| = \begin{vmatrix}
                           \lambda & -1      \\
                           -1      & \lambda
                       \end{vmatrix} = \lambda^{2} - 1 = (\lambda - 1)(\lambda + 1) \]

                   \( \therefore A \) 有两个特征值 \( \lambda_{1} = 1 \), \( \lambda_{2} = -1 \)

                   对于特征值 \( \lambda_{1} = 1 \), 解 \( (\lambda_{1}E - A)x = 0 \), 即 \( \begin{pmatrix}
                       1  & -1 \\
                       -1 & 1
                   \end{pmatrix}\begin{pmatrix}
                       x_{1} \\
                       x_{2}
                   \end{pmatrix} = 0 \)

                   可得一个特征向量为 \( (1, 1)^{\mathrm{T}} \). (全部为 \( k\begin{pmatrix}
                       1 \\
                       1
                   \end{pmatrix} \))

                   对于特征值 \( \lambda_{2} = -1 \), 解 \( (\lambda_{2}E - A)x = 0 \), 即 \( \begin{pmatrix}
                       -1 & -1 \\
                       -1 & -1
                   \end{pmatrix}\begin{pmatrix}
                       x_{1} \\
                       x_{2}
                   \end{pmatrix} = 0 \)

                   得一个特征向量为 \( (1, -1)^{\mathrm{T}} \). (全部为 \( k\begin{pmatrix}
                       1 \\
                       -1
                   \end{pmatrix} \))
             \item %(2)
                   \( f_{A}(\lambda) = |\lambda E - A| = \begin{vmatrix}
                       \lambda - 11 & -25         \\
                       4            & \lambda + 9
                   \end{vmatrix} = (\lambda - 1)^{2} \),

                   特征值为 \( 1, 1 \)

                   特征值 \( 1 \) 的特征向量由 \( \begin{pmatrix}
                       -10 & -25 \\
                       4   & 10
                   \end{pmatrix}\begin{pmatrix}
                       x_{1} \\
                       x_{2}
                   \end{pmatrix} = 0 \)

                   得 \( k(-5, 2)^{\mathrm{T}} \).
             \item %(3)
                   \( f_{A}(\lambda) = |\lambda E - A| = \begin{vmatrix}
                       \lambda + 2 & -1          \\
                       -5          & \lambda - 2
                   \end{vmatrix} = (\lambda - 3)(\lambda + 3) \),

                   特征值为 \( 3, -3 \)

                   特征值 \( -3 \) 的特征向量为 \( k_1(1, -1)^{\mathrm{T}} \)

                   特征值 \( 3 \) 的特征向量为 \( k_2(1, 5)^{\mathrm{T}} \)
             \item %(4)
                   \( f_{A}(\lambda) = |\lambda E - A| = \begin{vmatrix}
                       \lambda + 1 & -2          \\
                       -8          & \lambda + 1
                   \end{vmatrix} = (\lambda + 5)(\lambda - 3) \),

                   特征值为 \( -5, 3 \)

                   特征值为 \( -5 \) 的特征向量为 \( k_1(1, -2)^{\mathrm{T}} \)

                   特征值为 \( 3 \) 的特征向量为 \( k_2(1, 2)^{\mathrm{T}} \)
         \end{enumerate}

     \paragraph{} %2
         \begin{enumerate}
             \item %(1)
                   \( A \) 的特征多项式为 \( f_{A}(\lambda) = |\lambda E - A| = \begin{vmatrix}
                       \lambda - 3 & -6          & -6          \\
                       0           & \lambda - 2 & 0           \\
                       3           & 12          & \lambda + 6
                   \end{vmatrix} = (\lambda + 3)(\lambda - 2)\lambda \),

                   特征值 \( -3, 0, 2 \).

                   特征值 \( -3 \) 的特征向量为 \( k_1(1, 0, -1)^{\mathrm{T}} \)

                   特征值 \( 0 \) 的特征向量为 \( k_2(-2, 0, 1)^{\mathrm{T}} \)

                   特征值 \( 2 \) 的特征向量为 \( k_3(2, -5, 3)^{\mathrm{T}} \)
             \item %(2)
                   \( f_{A}(\lambda) = |\lambda E - A| = \begin{vmatrix}
                       \lambda - 3 & 2           & 0           \\
                       1           & \lambda - 3 & 1           \\
                       5           & -7          & \lambda + 1
                   \end{vmatrix} = (\lambda - 1)(\lambda - 2)^{2} \),

                   特征值 \( 1, 2, 2 \)

                   特征值 \( 1 \) 的特征向量为 \( k_1(1, 1, 1)^{\mathrm{T}} \)

                   特征值 \( 2 \) 的特征向量为 \( k_2(-2, -1, 1)^{\mathrm{T}} \)
             \item %(3)
                   \( f_{A}(\lambda) = |\lambda E - A| = \begin{vmatrix}
                       \lambda - 1 & -1          & -1          \\
                       -1          & \lambda - 1 & -1          \\
                       -1          & -1          & \lambda - 1
                   \end{vmatrix} = \lambda^{2}(\lambda - 3) \),

                   特征值 \( 0, 0, 3 \)

                   特征值 \( 0 \) 的特征向量为 \( k_1(-1, 1, 0)^{\mathrm{T}} + k_2(-1, 0, 1)^{\mathrm{T}} \)

                   特征值 \( 3 \) 的特征向量为 \( k_3(1, 1, 1)^{\mathrm{T}} \)
             \item %(4)
                   \( f_{A}(\lambda) = |\lambda E - A| = \begin{vmatrix}
                       \lambda - 4 & 2           & 1           \\
                       -5          & \lambda + 2 & 1           \\
                       2           & -1          & \lambda - 1
                   \end{vmatrix} = (\lambda - 1)^{3} \),

                   特征值 \( 1, 1, 1 \)

                   特征值 \( 1 \) 的向量为 \( k(-1, -2, 1)^{\mathrm{T}} \)
             \item %(5)
                   \( f_{A}(\lambda) = |\lambda E - A| = \begin{vmatrix}
                       \lambda - 2 & 8           & 4           \\
                       2           & \lambda - 2 & -2          \\
                       -7          & 14          & \lambda + 9
                   \end{vmatrix} = (\lambda + 1)(\lambda + 2)^{2} \),

                   特征值 \( -1, -2, -2 \)

                   特征值 \( -1 \) 的特征向量为 \( k_1(-4, 2, 7)^{\mathrm{T}} \)

                   特征值 \( -2 \) 的特征向量为 \( k_2(2, 1, 0)^{\mathrm{T}} + k_3(1, 0, 1)^{\mathrm{T}} \)
             \item %(6)
                   \( f_{A}(\lambda) = |\lambda E - A| = \begin{vmatrix}
                       \lambda - 23 & -10         & -16          \\
                       8            & \lambda + 2 & 6            \\
                       27           & 11          & \lambda + 19
                   \end{vmatrix} = (\lambda - 1)^{2}\lambda \),

                   特征值 \( 0, 1, 1 \)

                   特征值 \( 0 \) 的特征向量为 \( k_1(-14, 5, 17)^{\mathrm{T}} \)

                   特征值 \( 1 \) 的特征向量为 \( k_2(-6, 2, 7)^{\mathrm{T}} \).
         \end{enumerate}

     \paragraph{} %3
         \begin{enumerate}
             \item %(1)
                   \( f_{A}(\lambda) = |\lambda E - A| = \begin{vmatrix}
                       \lambda - 1 & -1          & 0           & 0           \\
                       0           & \lambda - 1 & -1          & 0           \\
                       0           & 0           & \lambda - 1 & 0           \\
                       0           & 0           & 0           & \lambda - 1
                   \end{vmatrix} = (\lambda - 1)^{4} \),

                   特征值 \( 1, 1, 1, 1 \)

                   特征值 \( 1 \) 的特征向量为 \( k_{1}(0, 0, 0, 1)^{\mathrm{T}} + k_{2}(1, 0, 0, 0)^{\mathrm{T}} \)
             \item %(2)
                   \( f_{A}(\lambda) = |\lambda E - A| = \begin{vmatrix}
                       \lambda - 1 & -1          & -1          & -1        \\
                       -1          & \lambda - 1 & 1           & 1         \\
                       -1          & 1           & \lambda - 1 & 1         \\
                       -1          & 1           & 1           & \lambda-1
                   \end{vmatrix} = (\lambda + 2)(\lambda - 2)^{3} \),

                   特征值 \( -2, 2, 2, 2 \)

                   特征值为 \( -2 \) 的特征向量为 \( k_{1}(-1, 1, 1, 1)^{\mathrm{T}} \)

                   特征值为 \( 2 \) 的特征向量为 \( k_{2}(1, 1, 0, 0)^{\mathrm{T}} + k_{3}(1, 0, 1, 0)^{\mathrm{T}} + k_{4}(1, 0, 0, 1)^{\mathrm{T}} \)
         \end{enumerate}


     \paragraph{} %4
         证明: 由 \( A\alpha = \lambda\alpha \), 左乘 \( A^{-1} \), 有
         \begin{gather*}
             A^{-1}(A\alpha) = A^{-1}(\lambda\alpha) \\
             \implies \lambda A^{-1}\alpha = \alpha \Rightarrow A^{-1}\alpha = \frac{1}{\lambda}\alpha
         \end{gather*}
         则 \( \frac{1}{\lambda} \) 为 \( A^{-1} \) 的特征值.


     \paragraph{} %5
         \begin{enumerate}
             \item %(1)
                   证明: \( \because A^{2}\alpha = A(A\alpha) = A(\lambda\alpha) = \lambda(A\alpha) = \lambda^{2}\alpha \)

                   设当 \( n = k \) 时, \( A^{k}\alpha = \lambda^{k}\alpha \)

                   当 \( n = k+1 \) 时, \( A^{k+1}\alpha = A(A^{k}\alpha) = A(\lambda^{k}\alpha) = \lambda^{k}A\alpha = \lambda^{k+1}\alpha \)

                   故 \( A^{n}\alpha = \lambda^{n}\alpha \). \( \because A^{n} = O \), 则 \( \lambda^{n}\alpha = 0 \)

                   又 \( \alpha \neq 0 \), \( \therefore \lambda^{n} = 0 \) 从而 \( \lambda = 0 \)
             \item %(2)
                   证明: 设 \( A\alpha = \lambda\alpha \), 同时左乘 \( A \), 有
                   \[ A^{2}\alpha = \lambda A\alpha = \lambda^{2}\alpha \]

                   又 \( A^{2} = A \), 故 \( \lambda\alpha = A\alpha = A^{2}\alpha = \lambda^{2}\alpha \)

                   \( \Rightarrow \lambda(\lambda - 1)\alpha = 0 \) 又 \( \alpha \neq 0 \)

                   \( \Rightarrow \lambda = 0 \) 或 \( \lambda = 1 \)
             \item %(3)
                   证明: 设 \( A\alpha = \lambda\alpha \), 同时左乘 \( A \), 有
                   \[ A^{2}\alpha = \lambda A\alpha = \lambda^{2}\alpha \]

                   又 \( \alpha = E\alpha = A^{2}\alpha \)

                   \( \Rightarrow (\lambda^{2} - 1)\alpha = 0 \) 又 \( \alpha \neq 0 \)

                   \( \Rightarrow \lambda = 1 \) 或 \( -1 \)
         \end{enumerate}


     \paragraph{} %6
         证明: 设 \( A\alpha = \lambda\alpha \)

         则 \( (kE + A)\alpha = k\alpha + A\alpha = k\alpha + \lambda\alpha = (k + \lambda)\alpha \)

         \( \therefore k + \lambda \) 是 \( kE + A \) 的特征值


     \paragraph{} %7
         \( f_{A}(\lambda) = |\lambda E - A| = \begin{vmatrix}
             \lambda - 1 & -1          & -1          \\
             -1          & \lambda - 1 & -1          \\
             -1          & -1          & \lambda - 1
         \end{vmatrix} = \lambda^{2}(\lambda - 3) \),

         特征值为 \( 0, 0, 3 \)

         特征值 \( 0 \) 的特征向量为 \( k_{1}(-1, 1, 0)^{\mathrm{T}} + l(-1, 0, 1)^{\mathrm{T}} \)

         特征值 \( 3 \) 的特征向量为 \( k_{2}(1, 1, 1)^{\mathrm{T}} \)

         故特征值 \( 0 \) 的特征子空间为 \( \left\{ k\alpha + l\beta \mid \alpha = (-1, 1, 0)^{\mathrm{T}}, \beta = (-1, 0, 1)^{\mathrm{T}}, k, l \in \mathbf{R} \right\} \)

         特征值 \( 3 \) 的特征子空间为 \( \left\{ k\alpha \mid \alpha = (1, 1, 1)^{\mathrm{T}}, k \in \mathbf{R} \right\} \).

     \paragraph{} %8
         依题意, \( A\alpha = \lambda\alpha \) 有一个特征向量为 \( (1, -2, 1)^{\mathrm{T}} \).

         则 \[ \begin{pmatrix}
                 k & 1 & 0 \\
                 1 & 2 & 1 \\
                 0 & 1 & k
             \end{pmatrix}\begin{pmatrix}
                 1  \\
                 -2 \\
                 1
             \end{pmatrix} = \lambda\begin{pmatrix}
                 1  \\
                 -2 \\
                 1
             \end{pmatrix} \Rightarrow \begin{pmatrix}
                 k - 2 \\
                 -2    \\
                 k - 2
             \end{pmatrix} = \begin{pmatrix}
                 \lambda_0 \\
                 -2        \\
                 \lambda_0
             \end{pmatrix} \]

         则 \( \lambda_{0} = k - 2 \), 则
         \[ |\lambda_{0}E - A| = \begin{vmatrix}
                 -2 & -1    & 0  \\
                 -1 & k - 4 & -1 \\
                 0  & -1    & -2
             \end{vmatrix} = 0 \quad \text{得 } k = 3 \]

         \( \begin{aligned} \text{则}
             f_{A}(\lambda) & = |\lambda E - A| = \begin{vmatrix}
                                                      \lambda - 3 & -1          & 0           \\
                                                      -1          & \lambda - 2 & -1          \\
                                                      0           & -1          & \lambda - 3
                                                  \end{vmatrix} = (\lambda - 3)
             \begin{vmatrix}
                 \lambda - 2 & 1           \\
                 -1          & \lambda - 3
             \end{vmatrix} +
             \begin{vmatrix}
                 -1 & -1          \\
                 0  & \lambda - 3
             \end{vmatrix}                                                             \\
                            & = (\lambda - 3)(\lambda - 1)(\lambda - 4)                   \\
         \end{aligned} \)

         故 \( A \) 的特征值为 \( 3, 1, 4 \).


     \paragraph{} %9
         \( \because A \) 可逆, 则 \( |A| \neq 0 \), \( \therefore \lambda \neq 0 \)

         则由 \( A^{-1}\alpha = \lambda\alpha \Rightarrow A\alpha = \frac{1}{\lambda}\alpha \)

         故 \( \alpha \) 也是 \( A \) 的特征向量, 则
         \[ \begin{pmatrix}
                 2 & 1 & 1 \\
                 1 & 2 & 1 \\
                 1 & 1 & 2
             \end{pmatrix}\begin{pmatrix}
                 1 \\
                 k \\
                 1
             \end{pmatrix} = \lambda\begin{pmatrix}
                 1 \\
                 k \\
                 1
             \end{pmatrix} \Leftrightarrow \begin{pmatrix}
                 3 + k  \\
                 2k + 2 \\
                 3 + k
             \end{pmatrix} = \begin{pmatrix}
                 \lambda   \\
                 \lambda k \\
                 \lambda
             \end{pmatrix} \Rightarrow \begin{cases}
                 3 + k = \lambda    \\
                 2k + 2 = \lambda k \\
                 3 + k = \lambda
             \end{cases} \]

         得 \( k = 1 \) 或 \( -2 \).


     \paragraph{} %10
         \begin{enumerate}
             \item %(1)
                   由第5题(1)的证明知由 \( A\alpha = \lambda\alpha \) 可得 \( A^{m}\alpha = \lambda^{m}\alpha \)

                   \(\begin{aligned}
                       \text{则} f(A)\alpha & = (A^{3} - 2A^{2} - A + 2E)\alpha                  \\
                                           & = A^{3}\alpha - 2A^{2}\alpha - A\alpha + 2\alpha   \\
                                           & = (\lambda^{3} - 2\lambda^{2} - \lambda + 2)\alpha \\
                                           & = f(\lambda)\alpha \quad
                   \end{aligned}\)

                   故 \(f(\lambda)\) 是 \(f(A)\) 的特征值.
             \item %(2)
                   由10.(1)的公式可得

                   1.(1) \( 0, 0 \)\quad(2) \( 0, 0 \)\quad (3) \( -40, 8 \)\quad (4) \( -168, 8 \)

         \end{enumerate}


     \paragraph{} %11
         证明: 同10.(1)有 \( \varphi(A)\alpha = (a_{m}A^{m} + a_{m-1}A^{m-1} + \cdots + a_{0}E)\alpha = (a_{m}\lambda^{m} + \cdots + a_{0})\alpha = \varphi(\lambda)\alpha \)

         即 \(\varphi(\lambda)\) 是 \(\varphi(A)\) 的特征值.


 \subsection{} %B


     \paragraph{} %1
         证明: 设 \( A = \begin{pmatrix}
             a & b \\
             b & a
         \end{pmatrix} \)

         则 \( f_{A}(\lambda) = |\lambda E - A| = \begin{vmatrix}
             \lambda - a & -b          \\
             -b          & \lambda - a
         \end{vmatrix} = (\lambda - a - b)\begin{vmatrix}
             1 & -b          \\
             1 & \lambda - a
         \end{vmatrix} = (\lambda - a - b)(\lambda - a + b) \)

         故 \( A \) 一定有两个实特征值 \( a+b \), \( a-b \).


     \paragraph{} %2
         证明: \( \because |\lambda E - A^{\mathrm{T}}| = |(\lambda E - A)^{\mathrm{T}}| = |\lambda E - A| \),

         \( \therefore A \) 与 \( A^{\mathrm{T}} \) 有相同的特征多项式

     \paragraph{4.} %4


         只考虑 \( n \) 维向量. 首先, 由于
         \[ r(\alpha\beta^{\mathrm{T}}) \leq \min\left\{ r(\alpha), r(\beta^{\mathrm{T}}) \right\} = 1 \]

         故矩阵 \( \alpha\beta^{\mathrm{T}} \) 必有 \( \lambda = 0 \) 的特征值. 记 \( A = \alpha\beta^{\mathrm{T}} \)

         则 \( \lambda = 0 \) 的 \( Ax = 0 \) 所含的向量个数
         \[ n - r(A) \geq n - 1 \]

         则 \( \lambda \) 的为0的代数重数至少有 \( n-1 \) 个,

         故 \( A \) 最多只有1个不为0的特征值. 记 \( k = \beta^{\mathrm{T}}\alpha \)

         又 \[
             (\alpha\beta^{\mathrm{T}})\alpha = \alpha\cdot k = k\alpha
         \]
         则 \( k = \beta^{\mathrm{T}}\alpha \) 是一个特征值.

         则 \( A \) 的特征多项式为 \( \lambda^{n-1}(\lambda - \beta^{\mathrm{T}}\alpha) \)

         特征值为 \( \beta^{\mathrm{T}}\alpha \), \( 0 \) (\( n-1 \) 重)

         \setcounter{paragraph}{4} %手动更改序号为5
     \paragraph{} %5
         \begin{enumerate}
             \item %(1)
                   \( f_{A}(\lambda) = |\lambda E - A| = \begin{vmatrix}
                       \lambda - 4 & 1           & 1       \\
                       12          & \lambda - 1 & -5      \\
                       -4          & 2           & \lambda
                   \end{vmatrix} = \lambda(\lambda - 2)(\lambda - 3) \)
             \item %(2)
                   特征值为 \( 0, 2, 3 \).

                   特征值 \( 0 \) 的特征向量为 \( k_{1}(1, 2, 2)^{\mathrm{T}} \)

                   特征值 \( 2 \) 的特征向量为 \( k_{2}(3, -1, 7)^{\mathrm{T}} \)

                   特征值 \( 3 \) 的特征向量为 \( k_{3}(-1, 1, -2)^{\mathrm{T}} \)
             \item %(3)
                   由A组5.(1)的结论, 有 \( A^{5} \) 的特征值为 \( \lambda^{5} \), \( A+2E \) 的特征值为 \( \lambda+2 \)

                   故 \( A^{5} \) 的特征值为 \( 0, 32, 243 \)

                   \( A+2E \) 的特征值为 \( 2, 4, 5 \)
         \end{enumerate}


     \paragraph{} %6
         证明: 由定理5.1知 \( m=2 \) 时成立.

         假设当 \( m=k \) 时成立, 则立 \( m=k+1 \) 时, 对于
         \[ k_{1}x_{1} + k_{2}x_{2} + \cdots + k_{k}x_{k} + k_{k+1}x_{k+1} = 0 \tag{1} \]

         对(1)用 \( A \) 左乘两端有
         \[ k_{1}\lambda_{1}x_{1} + k_{2}\lambda_{2}x_{2} + \cdots + k_{k}\lambda_{k}x_{k} + k_{k+1}\lambda_{k+1}x_{k+1} = 0 \tag{2} \]

         (1) 与 (2) 消去 \( x_{k+1} \), 有
         \[ k_{1}(\lambda_{k+1} - \lambda_{1})x_{1} + k_{2}(\lambda_{k+1} - \lambda_{2})x_{2} + \cdots + k_{k+1}(\lambda_{k+1} - \lambda_{k})x_{k} = 0 \]

         由归纳假设知 \( x_{1}, x_{2}, \dots, x_{k} \) 线性无关. 故
         \[ k_{1} = k_{2} = \cdots = k_{k} = 0 \]

         则 \( k_{k+1}x_{k+1} = 0 \Rightarrow k_{k+1} = 0 \)

         从而 \( x_{1}, x_{2}, \dots, x_{k}, x_{k+1} \) 线性无关.

         综上, 由数学归纳法知定理5.2成立.


     \paragraph{} %7
         \( m=2 \) 时, 设 \( \lambda_{1}, \lambda_{2} \) 是两个不同的特征值, 设
         \[ k_{1}x_{11} + k_{2}x_{12} + \cdots + k_{k_{1}}x_{1k_{1}} + l_{1}x_{21} + \cdots + l_{k_{2}}x_{2k_{2}} = 0 \tag{1} \]

         对 (1) 两边左乘 \( A \), 有
         \[ k_{1}A x_{11} + k_{2}A x_{12} + \cdots + k_{k_{1}}A x_{1k_{1}} + l_{1}A x_{21} + \cdots + l_{k_{2}}A x_{2k_{2}} = 0 \]

         由于 \( x_{1i} (i=1,2,\dots,k_{1}) \) 和 \( x_{2j} (j=1,2,\dots,k_{2}) \) 分别是 \( A \) 的属于 \( \lambda_{1}, \lambda_{2} \) 的特征向量, 则
         \[ k_{1}\lambda_{1}x_{11} + k_{2}\lambda_{1}x_{12} + \cdots + k_{k_{1}}\lambda_{1}x_{1k_{1}} + l_{1}\lambda_{2}x_{21} + \cdots + l_{k_{2}}\lambda_{2}x_{2k_{2}} = 0 \tag{2} \]

         (2) \( - \lambda_{2} \) (1), 有
         \[ k_{1}(\lambda_{1} - \lambda_{2})x_{11} + k_{2}(\lambda_{1} - \lambda_{2})x_{12} + \cdots + k_{k_{1}}(\lambda_{1} - \lambda_{2})x_{1k_{1}} = 0 \]

         由于 \( x_{11}, x_{12}, \dots, x_{1k_{1}} \) 线性无关, 且 \( \lambda_{1} - \lambda_{2} \neq 0 \), 从而
         \[ k_{1} = k_{2} = \cdots = k_{k_{1}} = 0 \]

         再代入 (1) 有
         \[ l_{1}x_{21} + l_{2}x_{22} + \cdots + l_{k_{2}}x_{2k_{2}} = 0 \]

         由于 \( x_{21}, x_{22}, \dots, x_{2k_{2}} \) 线性无关, 则 \( l_{1} = l_{2} = \cdots = l_{k_{2}} = 0 \)

         因此向量组 \( x_{11}, x_{12}, \dots, x_{1k_{1}}, x_{21}, \dots, x_{2k_{2}} \) 线性无关. \( m=2 \) 时成立.

         假设 \( m=n \) 时结论成立, 则立 \( m=n+1 \) 时, 设
         \[ k_{11}x_{11} + k_{12}x_{12} + \cdots + k_{1k_{1}}x_{1k_{1}} + k_{21}x_{21} + \cdots + k_{2k_2}x_{2k_{2}} + \cdots + k_{n1}x_{n1} + \cdots + k_{n k_n}x_{n k_n} + \cdots + k_{(n+1)k_{n+1}}x_{(n+1)k_{n+1}} = 0 \]

         同左乘 \( A \) 后有
         \[ k_{11}\lambda_{1}x_{11} + k_{12}\lambda_{1}x_{12} + \cdots + k_{1k_{1}}\lambda_{1}x_{1k_{1}} + \cdots + k_{(n+1)1}\lambda_{n+1}x_{(n+1)1} + \cdots + k_{(n+1)k_{n+1}}\lambda_{n+1}x_{(n+1)k_{n+1}} = 0 \]

         消去 \( x_{(n+1)1}, x_{(n+1)2}, \dots, x_{(n+1)k_{n+1}} \), 有
         \[ k_{11}(\lambda_{n+1} - \lambda_{1})x_{11} + k_{12}(\lambda_{n+1} - \lambda_{1})x_{12} + \cdots + k_{nn}(\lambda_{n+1} - \lambda_{n})x_{nk_{n}} = 0 \]

         由归纳假设知 \( k_{11} = k_{12} = \cdots = k_{n k_n} = 0 \), 往回代入, 有
         \[ k_{(n+1)1}x_{(n+1)1} + k_{(n+1)2}x_{(n+1)2} + \cdots + k_{(n+1)k_{n+1}}x_{(n+1)k_{n+1}} = 0 \]

         从而 \( k_{(n+1)1} = k_{(n+1)2} = \cdots = k_{(n+1)k_{n+1}} = 0 \)

         综上, 结论成立.


     \paragraph{} %8
         依题意, 设 \( f(t) = t^{2} + at + b \)

         则有 \( f(\lambda) = \lambda^{2} + a\lambda + b = 0 \)

         \(\begin{aligned}
             \text{考虑 } f(A)\alpha & = A^{2}\alpha + aA\alpha + b\alpha             \\
                                   & = \lambda^{2}\alpha + a\lambda\alpha + b\alpha \\
                                   & = (\lambda^{2} + a\lambda + b)\alpha           \\
                                   & = 0                                            \\
         \end{aligned}\)

         又 \(\alpha \neq 0\),

         则 \( f(A) = 0 \)

     \paragraph{} %9
         证明: \( f_{C}(\lambda) = |\lambda E_{m+n} - C| = \begin{vmatrix}
             \lambda E_{m} - A & D                 \\
             0                 & \lambda E_{n} - B
         \end{vmatrix} = |\lambda E_{m} - A|\cdot|\lambda E_{n} - B| = f_{A}(\lambda)f_{B}(\lambda) \)


     \paragraph{} %10
         \begin{enumerate}
             \item %(1)
                   \( f_{A}(\lambda) = \begin{vmatrix}
                       \lambda - 1 & -2          & 1           & -2          \\
                       -2          & \lambda - 4 & -2          & 1           \\
                                   &             & \lambda - 1 & -2          \\
                                   &             & 1           & \lambda + 2
                   \end{vmatrix} = \begin{vmatrix}
                       \lambda - 1 & -2          \\
                       -2          & \lambda - 4
                   \end{vmatrix} \cdot \begin{vmatrix}
                       \lambda - 1 & -2          \\
                       1           & \lambda + 2
                   \end{vmatrix} = (\lambda - 5)(\lambda + 1)\lambda^{2} \)

                   故特征值为 \( -1, 5, 0, 0 \).

                   特征值为 \( -1 \) 的特征向量是 \( k_{1}(-7, 4, -2, 2)^{\mathrm{T}} \)

                   特征值为 \( 5 \) 的特征向量是 \( k_{2}(1, 2, 0, 0)^{\mathrm{T}} \)

                   特征值为 \( 0 \) 的特征向量是 \( k_{3}(-2, 1, 0, 0)^{\mathrm{T}} \)
             \item %(2)
                   \( f_{A}(\lambda) = \begin{vmatrix}
                       \lambda - 3 & 1           &           &           \\
                       -5          & \lambda - 3 &           &           \\
                                   &             & \lambda-1 & 1         \\
                                   &             & -1        & \lambda-3
                   \end{vmatrix} = \begin{vmatrix}
                       \lambda - 3 & 1           \\
                       -5          & \lambda - 3
                   \end{vmatrix} \cdot \begin{vmatrix}
                       \lambda - 1 & 1           \\
                       -1          & \lambda - 3
                   \end{vmatrix} = (\lambda - 2)^{2}(\lambda + 2) \)

                   故特征值为 \( 2, 2, -2 \).

                   特征值为 \( -2 \) 的特征向量是 \( k_{1}(1, 5, 0)^{\mathrm{T}} \)

                   特征值为 \( 2 \) 的特征向量是 \( k_{2}(0, 0, -1, 1)^{\mathrm{T}} + k_{3}(1, 1, 0, 0)^{\mathrm{T}} \)
         \end{enumerate}


     \paragraph{} %11
         \begin{enumerate}
             \item %(1)
                   \begin{align*}
                       f_{A}(\lambda)
                        & = \begin{vmatrix}
                                \lambda - 1 & 1           & \cdots & 1           \\
                                1           & \lambda - 1 & \cdots & 1           \\
                                \vdots      & \vdots      & \ddots & \vdots      \\
                                1           & 1           & \cdots & \lambda - 1
                            \end{vmatrix} = \begin{vmatrix}
                                                1      & 1           & 1           & \cdots & 1           \\
                                                0      & \lambda - 1 & 1           & \cdots & 1           \\
                                                0      & 1           & \lambda - 1 & \cdots & 1           \\
                                                \vdots & \vdots      & \vdots      & \ddots & \vdots      \\
                                                0      & 1           & 1           & \cdots & \lambda - 1
                                            \end{vmatrix} = \begin{vmatrix}
                                                                1      & 1         & 1         & \cdots & 1         \\
                                                                -1     & \lambda-2 &           &        &           \\
                                                                -1     &           & \lambda-2 &        &           \\
                                                                \vdots &           &           & \ddots &           \\
                                                                -1     &           &           &        & \lambda-2
                                                            \end{vmatrix} \\
                        & = (\lambda - 2)
                       \begin{vmatrix}
                           1      & 1         & 1         & \cdots & 1 & 1         \\
                           -1     & \lambda-2 &           &        &               \\
                           -1     &           & \lambda-2 &        &               \\
                           \vdots &           &           & \ddots &               \\
                           0      &           &           &        & 1 & -1        \\
                           -1     &           &           &        &   & \lambda-2
                       \end{vmatrix} =\begin{vmatrix}
                                          \lambda-2  & 1         & 1         & \cdots & 1 & 1         \\
                                          -\lambda+2 & \lambda-2 &           &        &   &           \\
                                          -\lambda+2 &           & \lambda-2 &        &   &           \\
                                          \vdots     &           &           & \ddots &   &           \\
                                          0          &           &           &        & 1 & -1        \\
                                          -\lambda   &           &           &        &   & \lambda-2
                                      \end{vmatrix}               \\
                        & =
                       \begin{vmatrix}
                           \lambda+n-2 & 1         & \cdots & 1 & 1         \\
                                       & \lambda-2 &        &   &           \\
                                       &           & \ddots &   &           \\
                                       &           &        & 1 &           \\
                                       &           &        &   & \lambda-2
                       \end{vmatrix}
                       =
                       (\lambda + n - 2)(\lambda - 2)^{n-1}
                   \end{align*}

                   则 \( A \) 的特征值为 \( 2 \) 和 \( 2-n \). 又 \( n - r(2E - A) = n - 1 \), \( n - r((2-n)E - A) = 1 \).
             \item %(2)
                   当 \( \lambda \) 为 \( 2 \) 时, 特征向量是 \( k_{1}\begin{pmatrix}
                       -1     \\
                       1      \\
                       0      \\
                       0      \\
                       \vdots \\
                       0
                   \end{pmatrix} + k_{2}\begin{pmatrix}
                       -1     \\
                       0      \\
                       1      \\
                       0      \\
                       \vdots \\
                       0
                   \end{pmatrix} + \cdots + k_{n-1}\begin{pmatrix}
                       -1     \\
                       0      \\
                       0      \\
                       0      \\
                       \vdots \\
                       1
                   \end{pmatrix} \);

                   当 \( \lambda \) 为 \( 2-n \) 时, 特征向量是 \( k_{n}\begin{pmatrix}
                       -1     \\
                       0      \\
                       0      \\
                       \vdots \\
                       1
                   \end{pmatrix} \).
         \end{enumerate}


\section{5.2}

 \subsection{} %A
     \paragraph{} %1
         \begin{enumerate}
             \item %(1)
                   不能. 因 \( r(E - A) = r\begin{pmatrix}
                       0 & 1 \\
                       0 & 0
                   \end{pmatrix} = 1 \), 说明 \( (E - A)x = 0 \) 的基础解系中只有一个解向量, 即 \( \lambda = 1 \) (二重根) 只有一个特征向量, 所以 \( \begin{pmatrix}
                       1 & -1 \\
                       0 & 1
                   \end{pmatrix} \) 不能对角化.
             \item %(2)
                   可以. 如 \( \begin{pmatrix}
                       \frac{-11 + \sqrt{97}}{2} &                           \\
                                                 & \frac{-11 - \sqrt{97}}{2}
                   \end{pmatrix} \)
         \end{enumerate}

     \paragraph{} %2
         \begin{enumerate}
             \item %(1)
                   由 \( f_{A}(\lambda) = |\lambda E - A| = \begin{vmatrix}
                       \lambda - 2 & -1          & 1       \\
                       -1          & \lambda - 2 & -1      \\
                       1           & -1          & \lambda
                   \end{vmatrix} = (\lambda - 1)^{2}(\lambda - 2) \), 则特征值为 \( 1, 1, 2 \)

                   又 \( \lambda = 1 \) 时, 由 \( \begin{pmatrix}
                       -1 & -1 & 1 \\
                       -1 & -1 & 1 \\
                       1  & -1 & 1
                   \end{pmatrix} \rightarrow \begin{pmatrix}
                       1 & 1 & -1 \\
                       0 & 0 & 0  \\
                       0 & 0 & 0
                   \end{pmatrix} \) 故基础解系 \( \alpha_{1} = (1, -1, 0)^{\mathrm{T}} \), \( \alpha_{2} = (1, 0, 1)^{\mathrm{T}} \)

                   \( \lambda = 2 \) 时, 由 \( \begin{pmatrix}
                       0  & -1 & 1 \\
                       -1 & 0  & 1 \\
                       -1 & -1 & 0
                   \end{pmatrix} \rightarrow \begin{pmatrix}
                       1 &   &   \\
                         & 1 &   \\
                         &   & 1
                   \end{pmatrix} \) 故基础解系 \( \alpha_{3} = (1, 1, 1)^{\mathrm{T}} \)

                   故 \( P = (\alpha_{1}, \alpha_{2}, \alpha_{3}) = \begin{pmatrix}
                       1  & 1 & 1 \\
                       -1 & 0 & 1 \\
                       0  & 1 & 1
                   \end{pmatrix} \), \( P^{-1}AP = \operatorname{diag}(1, 1, 2) \)
             \item %(2)
                   由 \( f_{A}(\lambda) = |\lambda E - A| = \begin{vmatrix}
                       \lambda - 1 & -2          & -3          \\
                       -2          & \lambda - 1 & -3          \\
                       -3          & -3          & \lambda - 6
                   \end{vmatrix} = \lambda(\lambda + 1)(\lambda - 9) \), 则特征值为 \( 0, -1, 9 \)

                   \( \lambda = 0 \) 时, 由 \( \begin{pmatrix}
                       -1 & -2 & -3 \\
                       -2 & -1 & -3 \\
                       -3 & -3 & -6
                   \end{pmatrix} \rightarrow \begin{pmatrix}
                       1 &   & 1 \\
                         & 1 & 1 \\
                         &   &
                   \end{pmatrix} \), 故基础解系 \( \alpha_{1} = (1, 1, -1)^{\mathrm{T}} \)

                   \( \lambda = -1 \) 时, 由 \( \begin{pmatrix}
                       -2 & -2 & -3 \\
                       -2 & -2 & -3 \\
                       -3 & -3 & -7
                   \end{pmatrix} \rightarrow \begin{pmatrix}
                       1 & 1 &   \\
                         &   & 1 \\
                         &   &
                   \end{pmatrix} \), 故基础解系 \( \alpha_{2} = (-1, 1, 0)^{\mathrm{T}} \)

                   \( \lambda = 9 \) 时, 由 \( \begin{pmatrix}
                       8  & -2 & -3 \\
                       -2 & 8  & -3 \\
                       -3 & -3 & 3
                   \end{pmatrix} \rightarrow \begin{pmatrix}
                       1 &   & -\frac{1}{2} \\
                         & 1 & -\frac{1}{2} \\
                         &   &
                   \end{pmatrix} \), 故基础解系 \( \alpha_{3} = (1, 1, 2)^{\mathrm{T}} \)

                   故 \( P = \begin{pmatrix}
                       1  & -1 & 1 \\
                       1  & 1  & 1 \\
                       -1 & 0  & 2
                   \end{pmatrix} \), \( P^{-1}AP = \operatorname{diag}(0, -1, 9) \)
             \item %(3)
                   由 \( f_{A}(\lambda) = |\lambda E - A| = \begin{vmatrix}
                       \lambda & 2           & 2       \\
                       -2      & \lambda + 4 & 2       \\
                       2       & -2          & \lambda
                   \end{vmatrix} = \lambda(\lambda + 2)^{2} \), 特征值为 \( 0, -2, -2 \)

                   \( \lambda = 0 \) 时, 由 \( \begin{pmatrix}
                       0  & 2  & 2 \\
                       -2 & 4  & 2 \\
                       2  & -2 & 0
                   \end{pmatrix} \rightarrow \begin{pmatrix}
                       1 &   & 1 \\
                         & 1 & 1 \\
                         &   &
                   \end{pmatrix} \) 得基础解系 \( \alpha_{1} = (1, 1, -1)^{\mathrm{T}} \)

                   \( \lambda = 2 \) 时, 由 \( \begin{pmatrix}
                       -2 & 2  & 2  \\
                       -2 & 2  & 2  \\
                       2  & -2 & -2
                   \end{pmatrix} \rightarrow \begin{pmatrix}
                       1 & -1 & -1 \\
                         &    &    \\
                         &    &
                   \end{pmatrix} \) 得 \( \alpha_{2} = (1, 1, 0)^{\mathrm{T}} \), \( \alpha_{3} = (1, 0, 1)^{\mathrm{T}} \)

                   故 \( P = \begin{pmatrix}
                       1  & 1 & 1 \\
                       1  & 1 & 0 \\
                       -1 & 0 & 1
                   \end{pmatrix} \), \( P^{-1}AP = \operatorname{diag}(0, -2, -2) \)
             \item %(4)
                   \(f_{A}(\lambda)=
                   \begin{vmatrix}
                       \lambda-8 & 2         & 1         \\
                       2         & \lambda-5 & 2         \\
                       3         & 6         & \lambda-6
                   \end{vmatrix}
                   = (\lambda-1)(\lambda-9)^{2}\), 特征值为 1, 9, 9

                   \(\lambda = 1\) 时, 由 \(
                   \begin{pmatrix}
                       -7 & 2  & 1  \\
                       2  & -4 & 2  \\
                       3  & 6  & -5
                   \end{pmatrix}\to \begin{pmatrix}
                       1 &   & -\frac{1}{3} \\
                         & 1 & -\frac{2}{3} \\
                         &   &
                   \end{pmatrix} \), 得基础解系 \(\alpha=(1,2,3)^{\mathrm{T}}\)

                   \(\lambda=9\) 时, 由 \(\begin{pmatrix}
                       1 & 2 & 1 \\
                       2 & 4 & 2 \\
                       3 & 6 & 3
                   \end{pmatrix} \to
                   \begin{pmatrix}
                       1 & 2 & 1 \\
                         &   &   \\
                         &   &
                   \end{pmatrix} \), 得基础解系 \(\alpha_2=(-2,1,0)^{\mathrm{T}}\), \(\alpha_3=(-1,0,1)^{\mathrm{T}}\)

                   则 \(P=
                   \begin{pmatrix}
                       1 & -2 & -1 \\
                       2 & 1  & 0  \\
                       3 & 0  & 1
                   \end{pmatrix} \), \(P^{-1}AP=\operatorname{diag}(1,9,9)\)
         \end{enumerate}

     \paragraph{} %3
         \begin{enumerate}
             \item %(1)
                   \( f_{A}(\lambda) = \begin{vmatrix}
                       \lambda - 7 & 12           & -6           \\
                       -10         & \lambda + 19 & -10          \\
                       -12         & 24           & \lambda - 13
                   \end{vmatrix} = (\lambda + 1)(\lambda - 1)^{2} \), 特征值为 \( -1, 1, 1 \)

                   \( \lambda = -1 \) 时, 由 \( \begin{pmatrix}
                       -8  & 12 & -6  \\
                       -10 & 18 & -10 \\
                       -12 & 24 & -14
                   \end{pmatrix} \rightarrow \begin{pmatrix}
                       1 &   & -\frac{1}{2} \\
                         & 1 & -\frac{5}{6} \\
                         &   &
                   \end{pmatrix} \), 得 \( \alpha_{1} = (3, 5, 6)^{\mathrm{T}} \)

                   \( \lambda = 1 \) 时, 由 \( \begin{pmatrix}
                       -6  & 12 & -6  \\
                       -10 & 20 & -10 \\
                       -12 & 24 & -12
                   \end{pmatrix} \rightarrow \begin{pmatrix}
                       1 & -2 & 1 \\
                         &    &   \\
                         &    &
                   \end{pmatrix} \), 得 \( \alpha_{2} = (2, 1, 0)^{\mathrm{T}} \), \( \alpha_{3} = (-1, 0, 1)^{\mathrm{T}} \)

                   则 \( P = \begin{pmatrix}
                       3 & 2 & -1 \\
                       5 & 1 & 0  \\
                       6 & 0 & 1
                   \end{pmatrix} \), \( P^{-1}AP = \begin{pmatrix}
                       -1 &   &   \\
                          & 1 &   \\
                          &   & 1
                   \end{pmatrix} \Rightarrow A = P\begin{pmatrix}
                       -1 &   &   \\
                          & 1 &   \\
                          &   & 1
                   \end{pmatrix}P^{-1} \)

                   故 \( A^{n} = P\cdot\begin{pmatrix}
                       -1 &   &   \\
                          & 1 &   \\
                          &   & 1
                   \end{pmatrix}^{n}P^{-1} \)

                   故 \( n \) 为奇数时 \( A^{n} = P\cdot\begin{pmatrix}
                       -1 &   &   \\
                          & 1 &   \\
                          &   & 1
                   \end{pmatrix}P^{-1} = A \)

                   \( n \) 为偶数时 \( A^{n} = P\cdot\begin{pmatrix}
                       1 &   &   \\
                         & 1 &   \\
                         &   & 1
                   \end{pmatrix}P^{-1} = PP^{-1} = E \)
             \item %(2)
                   \( f_{A}(\lambda) = \begin{vmatrix}
                       \lambda - 3 & 0           & 0           \\
                       -9          & \lambda - 6 & -5          \\
                       12          & 6           & \lambda + 5
                   \end{vmatrix} = \lambda(\lambda - 1)(\lambda - 3) \), 特征值为 \( 0, 1, 3 \)

                   \( \lambda = 0 \) 时, 由 \( \begin{pmatrix}
                       -3 & 0  & 0  \\
                       -9 & -6 & -5 \\
                       12 & 6  & 5
                   \end{pmatrix} \rightarrow \begin{pmatrix}
                       1 &   &             \\
                         & 1 & \frac{5}{6} \\
                         &   &
                   \end{pmatrix} \), 基础解系 \( \alpha_{1} = (0, -5, 6)^{\mathrm{T}} \)

                   \( \lambda = 1 \) 时, 由 \( \begin{pmatrix}
                       -2 & 0  & 0  \\
                       -9 & -5 & -5 \\
                       12 & 6  & 6
                   \end{pmatrix} \rightarrow \begin{pmatrix}
                       1 &   &   \\
                         & 1 & 1 \\
                         &   &
                   \end{pmatrix} \), 基础解系 \( \alpha_{2} = (0, -1, 1)^{\mathrm{T}} \)

                   \( \lambda = 3 \) 时, 由 \( \begin{pmatrix}
                       0  & 0  & 0  \\
                       -9 & -3 & -5 \\
                       12 & 6  & 6
                   \end{pmatrix} \rightarrow \begin{pmatrix}
                       1 &   & \frac{1}{3} \\
                         & 1 & \frac{2}{3} \\
                         &   &
                   \end{pmatrix} \), \( \alpha_{3} = (-1, -2, 3)^{\mathrm{T}} \),

                   则 \( P = \begin{pmatrix}
                       0  & 0  & -1 \\
                       -5 & -1 & -2 \\
                       6  & 1  & 3
                   \end{pmatrix} \)

                   \( P^{-1}AP = \operatorname{diag}(0, 1, 3) \),

                   则 \( A = P\operatorname{diag}(0, 1, 3)P^{-1} = \begin{pmatrix}
                       0  & 0  & -1 \\
                       -5 & -1 & -2 \\
                       6  & 1  & 3
                   \end{pmatrix}\begin{pmatrix}
                       0 & 0 & 0 \\
                       0 & 1 & 0 \\
                       0 & 0 & 3
                   \end{pmatrix}\begin{pmatrix}
                       1  & 1  & 1  \\
                       -3 & -6 & -5 \\
                       -1 & 0  & 0
                   \end{pmatrix} \)

                   \(\begin{aligned}
                       \text{故 } A^{n} & = P\begin{pmatrix}
                                                0 &   &       \\
                                                  & 1 &       \\
                                                  &   & 3^{n}
                                            \end{pmatrix}P^{-1} = \begin{pmatrix}
                                                                      0  & 0  & -1 \\
                                                                      -5 & -1 & -2 \\
                                                                      6  & 1  & 3
                                                                  \end{pmatrix}\begin{pmatrix}
                                                                                   0 &   &       \\
                                                                                     & 1 &       \\
                                                                                     &   & 3^{n}
                                                                               \end{pmatrix}\begin{pmatrix}
                                                                                                1  & 1  & 1  \\
                                                                                                -3 & -6 & -5 \\
                                                                                                -1 & 0  & 0
                                                                                            \end{pmatrix} \\
                                       & = \begin{pmatrix}
                                               3^{n}           & 0  & 0  \\
                                               2\times3^{n}-27 & 6  & 5  \\
                                               2\times3^{n}+27 & -6 & -5
                                           \end{pmatrix}
                   \end{aligned} \)
             \item %(3)
                   \( f_{A}(\lambda) = \begin{vmatrix}
                       \lambda - 122 & 100           \\
                       -150          & \lambda + 123
                   \end{vmatrix} = (\lambda - 2)(\lambda + 3) \), 特征值为 \( 2, -3 \)

                   \( \lambda = 2 \) 时, 由 \( \begin{pmatrix}
                       -120 & 100 \\
                       -150 & 125
                   \end{pmatrix} \rightarrow \begin{pmatrix}
                       1 & -\frac{5}{6} \\
                       0 & 0
                   \end{pmatrix} \), 得 \( \alpha_{1} = (5, 6)^{\mathrm{T}} \)

                   \( \lambda = -3 \) 时, 由 \( \begin{pmatrix}
                       -125 & 100 \\
                       -150 & 120
                   \end{pmatrix} \rightarrow \begin{pmatrix}
                       1 & -\frac{4}{5} \\
                       0 & 0
                   \end{pmatrix} \) 得 \( \alpha_{2} = (4, 5)^{\mathrm{T}} \)

                   故 \( P = \begin{pmatrix}
                       5 & 4 \\
                       6 & 5
                   \end{pmatrix} \), \( P^{-1}AP = \begin{pmatrix}
                       2 &    \\
                         & -3
                   \end{pmatrix} \)

                   故 \( A = P\begin{pmatrix}
                       2 &    \\
                         & -3
                   \end{pmatrix}P^{-1} \)

                   则 \( A^{n} = P\begin{pmatrix}
                       2^{n} &          \\
                             & (-3)^{n}
                   \end{pmatrix}P^{-1} = \begin{pmatrix}
                       5 & 4 \\
                       6 & 5
                   \end{pmatrix}\begin{pmatrix}
                       2^{n} &          \\
                             & (-3)^{n}
                   \end{pmatrix}\begin{pmatrix}
                       5  & -4 \\
                       -6 & 5
                   \end{pmatrix} \)
                   \[ = \begin{pmatrix}
                           -24\times(-3)^{n} + 25\times2^{n} & 20\times(-3)^{n} - 20\times2^{n} \\
                           -30\times(-3)^{n} + 30\times2^{n} & 25\times(-3)^{n} - 24\times2^{n}
                       \end{pmatrix} \]
             \item %(4)
                   \( f_{A}(\lambda) = \begin{vmatrix}
                       \lambda - 10 & -4           \\
                       24           & \lambda + 10
                   \end{vmatrix} = (\lambda - 2)(\lambda + 2) \), 特征值为 \( 2, -2 \)

                   则 \( A = P\begin{pmatrix}
                       2 &    \\
                         & -2
                   \end{pmatrix}P^{-1} = 2P\begin{pmatrix}
                       1 &    \\
                         & -1
                   \end{pmatrix}P^{-1} \)

                   则当 \( n \) 为奇数时, \( A^{n} = 2^{n}P\begin{pmatrix}
                       1 &    \\
                         & -1
                   \end{pmatrix}P^{-1} = 2^{n}A \)

                   \( n \) 为偶数时, \( A^{n} = 2^{n}P\begin{pmatrix}
                       1 &   \\
                         & 1
                   \end{pmatrix}P^{-1} = 2^{n}PP^{-1} = 2^{n}E \)
         \end{enumerate}


     \paragraph{} %4
         证明: 依题意, \( P^{-1}AP = B \)

         则 \( (P^{-1}AP)^{k} = B^{k} \)

         即 \( B^{k} = P^{-1}APP^{-1}AP\cdots P^{-1}AP = P^{-1}A^{k}P \), 则 \( B^{k} \) 与 \( A^{k} \) 也相似.


     \paragraph{} %5
         证明: \( \because A \sim B \), \( C \sim D \), 则存在可逆矩阵 \( P_{1}, P_{2} \), 使
         \[ B = P_{1}^{-1}AP_{1} \quad D = P_{2}^{-1}CP_{2} \]

         \(
         \begin{aligned}
             \text{则 }\begin{pmatrix}
                          B &   \\
                            & D
                      \end{pmatrix}
              & = \begin{pmatrix}
                      P_{1}^{-1}AP_{1} &                  \\
                                       & P_{2}^{-1}CP_{2}
                  \end{pmatrix} = \begin{pmatrix}
                                      P_{1}^{-1} &            \\
                                                 & P_{2}^{-1}
                                  \end{pmatrix}\begin{pmatrix}
                                                   A &   \\
                                                     & C
                                               \end{pmatrix}\begin{pmatrix}
                                                                P_{1} &       \\
                                                                      & P_{2}
                                                            \end{pmatrix} \\
              & = \begin{pmatrix}
                      P_{1} &       \\
                            & P_{2}
                  \end{pmatrix}^{-1}\begin{pmatrix}
                                        A &   \\
                                          & C
                                    \end{pmatrix}\begin{pmatrix}
                                                     P_{1} &       \\
                                                           & P_{2}
                                                 \end{pmatrix}
         \end{aligned} \)

         故 \( \begin{pmatrix}
             A &   \\
               & C
         \end{pmatrix} \sim \begin{pmatrix}
             B &   \\
               & D
         \end{pmatrix} \)


     \paragraph{} %6
         证明:
         \begin{enumerate}
             \item %(1)
                   设 \( A \) 可对角化为 \( B \), 则 \( B = P^{-1}AP \)

                   故 \( (P^{-1}AP)^{\mathrm{T}} = B^{\mathrm{T}} = B = P^{-1}AP \)

                   即 \( P^{\mathrm{T}}A^{\mathrm{T}}(P^{\mathrm{T}})^{\mathrm{T}} = P^{-1}AP \)

                   \( \Rightarrow A^{\mathrm{T}} = (PP^{\mathrm{T}})^{-1}APP^{\mathrm{T}} \), 从而 \( A^{\mathrm{T}} \sim A \)
             \item %(2)
                   取 \( Y = PP^{\mathrm{T}} \), 由(1)知 \( A^{\mathrm{T}} = Y^{-1}AY \)

                   故 \( AY - YA^{\mathrm{T}} = 0 \)
         \end{enumerate}


     \paragraph{} %7
         \( f_{A}(\lambda) = |\lambda E - A| = \begin{vmatrix}
             \lambda - 1 &             &             &           \\
             -a          & \lambda - 1 &             &           \\
             -2          & -b          & \lambda - 2 &           \\
             -2          & -3          & -c          & \lambda-2
         \end{vmatrix} = (\lambda - 1)^{2}(\lambda - 2)^{2} \), 特征值 \( 1, 1, 2, 2 \)

         由于 \( A \) 可对角化, 故 \( (E - A)x = 0 \) 与 \( (2E - A)x = 0 \) 的基础解系均有2个解向量,

         则 \( r(E - A) = 2 \), \( r(2E - A) = 2 \).

         对 \( E - A = \begin{pmatrix}
             0  &    &    &    \\
             -a & 0  &    &    \\
             -2 & -b & -1 &    \\
             -2 & -3 & -c & -1
         \end{pmatrix} \) 与 \( 2E - A = \begin{pmatrix}
             1  &    &    &   \\
             -a & 1  &    &   \\
             -2 & -b & 0  &   \\
             -2 & -3 & -c & 0
         \end{pmatrix} \),

         由于它们秩为 2, 故 3 阶子式的行列式 \( \begin{vmatrix}
             -a &    &    \\
             -2 & -b & -1 \\
             -2 & -3 & -c
         \end{vmatrix} \) 与 \( \begin{vmatrix}
             -a & 1  &     \\
             -2 & -b &     \\
             -2 & -3 & - c
         \end{vmatrix} \) 均为 0

         \( \Rightarrow a(bc - 3) = 0 \) 且 \( c(ab - 2) = 0 \)

         当 \( a = 0 \) 时, 则 \( -2c = 0 \Rightarrow c = 0 \), 此时 \( r(E - A) = 2 \), \( r(2E - A) = 2 \), 可对角化

         当 \( a \neq 0 \) 时, 则由 \( a(bc - 3) = 0 \Rightarrow bc = 3 \) 又 \( c(ab - 2) = 0 \), 则 \( ab = 2 \)

         但由 \( \begin{pmatrix}
             1  &    &    &   \\
             -a & 1  &    &   \\
             -2 & -b & 0  &   \\
             -2 & -3 & -c & 0
         \end{pmatrix} \rightarrow
         \begin{pmatrix}
             1 &   &    &   \\
               & 1 &    &   \\
               &   & -c &   \\
               &   &    & 0 \\
         \end{pmatrix}
         \rightarrow
         \begin{pmatrix}
             1 & 0 & 1 - c \\
             0 & 1 & -c    \\
             0 & 0 & 0
         \end{pmatrix} \) 则 \( c = 0 \), 矛盾, 故 \( a = 0 \).

         综上, \( a = c = 0 \), \( b \in \mathbf{R} \) 时, \( A \) 可对角化.


     \paragraph{} %8
         由于 \( f_{A}(\lambda) = |\lambda E - A| = \begin{vmatrix}
             \lambda-1 &           &           &                    \\
             1         & \lambda-2 &           &                    \\
             1         & 2         & \lambda-3 &                    \\
             \vdots    & \vdots    & \vdots    & \ddots             \\
             1         & 2         & 3         & \cdots & \lambda-n
         \end{vmatrix} = (\lambda - 1)(\lambda - 2)\cdots(\lambda - n) \),

         故 \( A \) 有 \( n \) 个特征根, 则共有 \( n \) 个无关特征向量, 可以对角化.

         则其相似标准形为 \( \begin{pmatrix}
             1 &   &   &            \\
               & 2 &   &            \\
               &   & 3 &            \\
               &   &   & \ddots     \\
               &   &   &        & n
         \end{pmatrix} \)


 \subsection{} %B
     \paragraph{} %1
         设一组二维向量 \( \begin{pmatrix}
             x_{n} \\
             x_{n-1}
         \end{pmatrix} \), 则
         \begin{align*}
             \begin{pmatrix}
                 x_{n} \\
                 x_{n-1}
             \end{pmatrix} & = \begin{pmatrix}
                                   3x_{n-1} - 2x_{n-2} \\
                                   x_{n-1}
                               \end{pmatrix} = \begin{pmatrix}
                                                   3 & -2 \\
                                                   1 & 0
                                               \end{pmatrix} \begin{pmatrix}
                                                                 x_{n-1} \\
                                                                 x_{n-2}
                                                             \end{pmatrix}                            \\
                             & = \begin{pmatrix}
                                     3 & -2 \\
                                     1 & 0
                                 \end{pmatrix}^{2} \begin{pmatrix}
                                                       x_{n-2} \\
                                                       x_{n-3}
                                                   \end{pmatrix}                                      \\
                             & \cdots                                                                  \\
                             & = \begin{pmatrix}
                                     3 & -2 \\
                                     1 & 0
                                 \end{pmatrix}^{n-2} \begin{pmatrix}
                                                         x_{2} \\
                                                         x_{1}
                                                     \end{pmatrix}, \quad \text{记 } A = \begin{pmatrix}
                                                                                            3 & -2 \\
                                                                                            1 & 0
                                                                                        \end{pmatrix}
         \end{align*}

         又由 \( f_{A}(\lambda) = |\lambda E - A| = \begin{vmatrix}
             \lambda-3 & 2       \\
             -1        & \lambda
         \end{vmatrix} = (\lambda-2)(\lambda-1) \), 知有特征值 2,1. 则

         \( \lambda = 2 \) 时, 由 \( \begin{pmatrix}
             -1 & 2 \\
             -1 & 2
         \end{pmatrix} \rightarrow \begin{pmatrix}
             1 & -2 \\
               &
         \end{pmatrix} \) 得 \( \alpha_{1} = (2,1)^{\mathrm{T}} \)

         \( \lambda = 1 \) 时, 由 \( \begin{pmatrix}
             -2 & 2 \\
             -1 & 1
         \end{pmatrix} \rightarrow \begin{pmatrix}
             1 & -1 \\
               &
         \end{pmatrix} \) 得 \( \alpha_{2} = (1,1)^{\mathrm{T}} \)

         故 \( \begin{pmatrix}
             3 & -2 \\
             1 & 0
         \end{pmatrix} = \begin{pmatrix}
             1  & -2 \\
             -1 & 1
         \end{pmatrix}\begin{pmatrix}
             1 &   \\
               & 2
         \end{pmatrix}\begin{pmatrix}
             -1 & -2 \\
             -1 & 1
         \end{pmatrix} \)

         则 \( \begin{pmatrix}
             3 & -2 \\
             1 & 0
         \end{pmatrix}^{n-2} = \begin{pmatrix}
             1  & -2 \\
             -1 & 1
         \end{pmatrix}\begin{pmatrix}
             1 &   \\
               & 2
         \end{pmatrix}^{n-2}\begin{pmatrix}
             -1 & 2 \\
             -1 & 1
         \end{pmatrix} = \begin{pmatrix}
             2^{n-1}-1  & 2-2^{n-1}  \\
             -1+2^{n-2} & -2^{n-2}+2
         \end{pmatrix} \)

         于是 \( \begin{pmatrix}
             x_{n} \\
             x_{n-1}
         \end{pmatrix} = \begin{pmatrix}
             2^{n-1}-1  & 2-2^{n-1}  \\
             -1+2^{n-2} & -2^{n-2}+2
         \end{pmatrix}\begin{pmatrix}
             4 \\
             1
         \end{pmatrix} = \begin{pmatrix}
             3\times2^{n-1}-2 \\
             3\times2^{n-2}-2
         \end{pmatrix} \)

         故 \( x_{n} = 3\times2^{n-1}-2 \)


     \paragraph{} %2
         由 \( \begin{pmatrix}
             x_{n} \\
             x_{n-1}
         \end{pmatrix} = \begin{pmatrix}
             7x_{n-1} - 12x_{n-2} \\
             x_{n-1}
         \end{pmatrix} = \begin{pmatrix}
             7 & -12 \\
             1 & 0
         \end{pmatrix}\begin{pmatrix}
             x_{n-1} \\
             x_{n-2}
         \end{pmatrix} \), 记 \( A = \begin{pmatrix}
             7 & -12 \\
             1 & 0
         \end{pmatrix} \)

         由 \( f_{A}(\lambda) = |\lambda E - A| = \begin{vmatrix}
             \lambda-7 & 12      \\
             -1        & \lambda
         \end{vmatrix} = (\lambda-5)(\lambda-2) \), 得 \( \lambda \) 为2,5. 则

         \( 2E-A = \begin{pmatrix}
             -5 & 12 \\
             -1 & 2
         \end{pmatrix} \rightarrow \begin{pmatrix}
             -5 & 2 \\
                &
         \end{pmatrix} \) 则 \( \alpha_{1} = (2,5)^{\mathrm{T}} \)

         \( 5E-A = \begin{pmatrix}
             -2 & 12 \\
             -1 & 5
         \end{pmatrix} \rightarrow \begin{pmatrix}
             -2 & 2 \\
                &
         \end{pmatrix} \) 则 \( \alpha_{2} = (1,1)^{\mathrm{T}} \), 故 \( P = \begin{pmatrix}
             2 & 1 \\
             5 & 1
         \end{pmatrix} \), \( P^{-1} = \frac{1}{3}\begin{pmatrix}
             1  & -1 \\
             -5 & 2
         \end{pmatrix} \)

         则 \( A = P\begin{pmatrix}
             2 &   \\
               & 5
         \end{pmatrix}P^{-1} \), 则 \( A^{n-2} = P
         \begin{pmatrix}
             2 &   \\
               & 5
         \end{pmatrix} ^{n-2}P^{-1}
         \begin{pmatrix}
             -3 \\
             2
         \end{pmatrix}  \)

         故 \( \begin{pmatrix}
             x_{n} \\
             x_{n-1}
         \end{pmatrix} = \begin{pmatrix}
             2 & 1 \\
             5 & 1
         \end{pmatrix}\begin{pmatrix}
             2 &   \\
               & 5
         \end{pmatrix}^{n-2}\begin{pmatrix}
             -\frac{1}{3} & \frac{1}{3}  \\
             \frac{5}{3}  & -\frac{2}{3}
         \end{pmatrix}\begin{pmatrix}
             -3 \\
             2
         \end{pmatrix} = \begin{pmatrix}
             11\times3^{n-1}-9\times4^{n-1} \\
             11\times3^{n-2}-9\times4^{n-2}
         \end{pmatrix} \)

         因此 \( x_{n} = 11\times3^{n-1}-9\times4^{n-1} \)


     \paragraph{} %3
         由
         \begin{align*}
             \begin{pmatrix}
                 x_{n} \\
                 y_{n} \\
                 z_{n}
             \end{pmatrix} = \begin{pmatrix}
                                 5x_{n-1} - 3y_{n-1} + 2z_{n-1} \\
                                 6x_{n-1} - 4y_{n-1} + 4z_{n-1} \\
                                 4x_{n-1} - 4y_{n-1} + 5z_{n-1}
                             \end{pmatrix} & = \begin{pmatrix}
                                                   5 & -3 & 2 \\
                                                   6 & -4 & 4 \\
                                                   4 & -4 & 5
                                               \end{pmatrix}\begin{pmatrix}
                                                                x_{n-1} \\
                                                                y_{n-1} \\
                                                                z_{n-1}
                                                            \end{pmatrix}           \\
                                               & = \begin{pmatrix}
                                                       5 & -3 & 2 \\
                                                       6 & -4 & 4 \\
                                                       4 & -4 & 5
                                                   \end{pmatrix}^{2}\begin{pmatrix}
                                                                        x_{n-2} \\
                                                                        y_{n-2} \\
                                                                        z_{n-2}
                                                                    \end{pmatrix}   \\
                                               & = \begin{pmatrix}
                                                       5 & -3 & 2 \\
                                                       6 & -4 & 4 \\
                                                       4 & -4 & 5
                                                   \end{pmatrix}^{n-1}\begin{pmatrix}
                                                                          x_{1} \\
                                                                          y_{1} \\
                                                                          z_{1}
                                                                      \end{pmatrix}
         \end{align*}
         设 \( A = \begin{pmatrix}
             5 & -3 & 2 \\
             6 & -4 & 4 \\
             4 & -4 & 5
         \end{pmatrix} \)

         则 \( f_{A}(\lambda) = |\lambda E - A| = \begin{vmatrix}
             \lambda-5 & 3         & -2        \\
             -6        & \lambda+4 & -4        \\
             -4        & 4         & \lambda-5
         \end{vmatrix} = (\lambda-1)(\lambda-2)(\lambda-3) \)

         又 \( E-A = \begin{pmatrix}
             -4 & 3 & -2 \\
             -6 & 5 & -4 \\
             -4 & 4 & -4
         \end{pmatrix} \rightarrow \begin{pmatrix}
             1 &   & -1 \\
               & 1 & -2 \\
               &   &
         \end{pmatrix} \) 得 \( \alpha_{1} = \begin{pmatrix}
             1 \\
             2 \\
             1
         \end{pmatrix} \)

         \( 2E-A = \begin{pmatrix}
             -3 & 3 & -2 \\
             -6 & 6 & -4 \\
             -4 & 4 & -3
         \end{pmatrix} \rightarrow \begin{pmatrix}
             1 & -1 &   \\
               &    & 1 \\
               &    &
         \end{pmatrix} \) 得 \( \alpha_{2} = \begin{pmatrix}
             1 \\
             1 \\
             0
         \end{pmatrix} \)

         \( 3E-A = \begin{pmatrix}
             -2 & 3 & -2 \\
             -6 & 7 & -4 \\
             -4 & 4 & -2
         \end{pmatrix} \rightarrow \begin{pmatrix}
             1 &   & -\frac{1}{2} \\
               & 1 & -1           \\
               &   &
         \end{pmatrix} \) 得 \( \alpha_{3} = \begin{pmatrix}
             1 \\
             2 \\
             2
         \end{pmatrix} \), 故 \( P = \begin{pmatrix}
             1 & 1 & 1 \\
             2 & 1 & 2 \\
             1 & 0 & 2
         \end{pmatrix} \)

         则 \( A = P\begin{pmatrix}
             1 &   &   \\
               & 2 &   \\
               &   & 3
         \end{pmatrix}P^{-1} \),

         \( \text{故 } A^{n-1}  = P\begin{pmatrix}
             1 &         &         \\
               & 2^{n-1} &         \\
               &         & 3^{n-1}
         \end{pmatrix}P^{-1}
         = \begin{pmatrix}
             1 & 1 & 1 \\
             2 & 1 & 2 \\
             1 & 0 & 2
         \end{pmatrix}P\begin{pmatrix}
             1 &         &         \\
               & 2^{n-1} &         \\
               &         & 3^{n-1}
         \end{pmatrix}\begin{pmatrix}
             -2 & 2  & -1 \\
             2  & -1 & 0  \\
             1  & -1 & 1
         \end{pmatrix}\)

         则 \( \begin{pmatrix}
             x_{n} \\
             y_{n} \\
             z_{n}
         \end{pmatrix} = A^{n-1}\begin{pmatrix}
             x_{1} \\
             y_{1} \\
             z_{1}
         \end{pmatrix} = \begin{pmatrix}
             3-2\times3^{n-1} \\
             6-4\times3^{n-1} \\
             3-4\times3^{n-1}
         \end{pmatrix} \)


     \paragraph{} %4
         证明: 由 \( f_{A}(\lambda) = |\lambda E - A| = \begin{vmatrix}
             \lambda-1 & -1        & -1        & \cdots & -1        \\
                       & \lambda-1 & -1        & \cdots & -1        \\
                       &           & \lambda-1 & \cdots & -1        \\
                       &           &           & \ddots & \vdots    \\
                       &           &           &        & \lambda-1
         \end{vmatrix} = (\lambda-1)^{n} \), 则特征值为1 (n重根)

         又 \( r(E-A) = n-1 \), 故 \( A \) 只有1个线性无关的特征向量, 则 \( A \) 不能与对角阵相似.


     \paragraph{} %5
         \begin{enumerate}
             \item %(1)
                   依题意, \( tr(A) = tr(B) \) 且 \( |A| = |B| \)

                   则 \( \begin{cases}
                       1+4+a = 2+2+b \\
                       \begin{vmatrix}
                           1  & -1 & 1  \\
                           2  & 4  & -2 \\
                           -3 & -3 & a
                       \end{vmatrix} = \begin{vmatrix}
                                           2 &   &   \\
                                             & 2 &   \\
                                             &   & b \\
                                       \end{vmatrix}
                   \end{cases} \)

                   得 \( \begin{cases}
                       a = 5 \\
                       b = 6
                   \end{cases} \)
             \item %(2)
                   由(1)知 \( A \) 的特征值为2,2,6, 则

                   \( 2E-A = \begin{pmatrix}
                       1  & 1  & -1 \\
                       -2 & -2 & 2  \\
                       3  & 3  & -3
                   \end{pmatrix} \rightarrow \begin{pmatrix}
                       1 & 1 & -1 \\
                         &   &    \\
                         &   &
                   \end{pmatrix} \) 得 \( \alpha_{1} = (-1,1,0)^{\mathrm{T}} \), \( \alpha_{2} = (1,0,1)^{\mathrm{T}} \)

                   \( 6E-A = \begin{pmatrix}
                       5  & 1 & -1 \\
                       -2 & 2 & 2  \\
                       3  & 3 & 1
                   \end{pmatrix} \rightarrow \begin{pmatrix}
                       1 &   & -\frac{1}{3} \\
                         & 1 & \frac{1}{3}  \\
                         &   & 0
                   \end{pmatrix} \) 得 \( \alpha_{3} = (1,-2,3)^{\mathrm{T}} \)

                   故 \( P = \begin{pmatrix}
                       -1 & 1 & 1  \\
                       1  & 0 & -2 \\
                       0  & 1 & 3
                   \end{pmatrix} \)
         \end{enumerate}


     \paragraph{} %6
         证明: 设 \( \lambda_{1}, \lambda_{2}, \dots, \lambda_{n} \) 为 \( A \) 的 \( n \) 个不同的特征值, 则存在可逆矩阵 \( P \) 使
         \[ P^{-1}AP = \begin{pmatrix}
                 \lambda_{1} &             &                      \\
                             & \lambda_{2} &                      \\
                             &             & \ddots &             \\
                             &             &        & \lambda_{n}
             \end{pmatrix} = \Lambda_{1} \]

         记 \( P = (\beta_{1}, \beta_{2}, \dots, \beta_{n}) \), 则 \( p_{i} \) (\( i=1,2,\dots,n \)) 也是 \( B \) 的特征向量, 则记对应特征值 \( \mu_{i} \)
         \[ Bp_{i} = \mu_{i}p_{i} \]

         则有 \( P^{-1}BP = \begin{pmatrix}
             \mu_{1} &         &                  \\
                     & \mu_{2} &                  \\
                     &         & \ddots &         \\
                     &         &        & \mu_{n}
         \end{pmatrix} = \Lambda_{2} \)

         于是 \( P^{-1}ABP = (P^{-1}AP)(P^{-1}BP) = \Lambda_{1}\Lambda_{2} = \Lambda_{2}\Lambda_{1} = (P^{-1}BP)(P^{-1}AP) = P^{-1}BAP \)

         于是 \( AB = BA \)


     \paragraph{} %7
         由于 \( A \sim B \), 若 \( r(A) = n \), 则由于 \( A \sim B \), 则 \( |A| = |B| \), 且 \( A^{-1} \sim B^{-1} \).

         于是 \( P^{-1}((nA^{-1})P = |B|B^{\mathrm{T}} \), 即 \( A^{*} \) 与 \( B^{*} \) 相似.

         若 \( r(A) < n \), 则 \( A, B \) 不可逆, 必 \( \exists \delta > 0 \), 当 \( t \in (0,\delta) \), 使 \( |tE+A| \neq 0 \), \( |tE+B| \neq 0 \)

         记 \( A_{t} = tE+A \), \( B_{t} = tE+B \), 则 \( B_{t} = tE+B = tE+P^{-1}AP = P^{-1}(tE+A)P = P^{-1}A_{t}P \)

         则由 \( r(A) = n \) 的证明知 \( B_{t}^{*} \sim A_{t}^{*} \) 即 \( (tE+B)^{*} = P^{-1}(tE+A)^{*}P \)

         上式两端矩阵均为 \( t \) 的多项式, 由于当 \( t \in (0,\delta) \) 时, 对应元素相似, 则取 \( t \to 0 \), 有 \( A^{*} \) 与 \( B^{*} \) 相似.


     \paragraph{} %8
         证明: 设 \( \lambda \) 是 \( A \) 的任一特征值, \( \alpha \) 是 \( \lambda \) 的特征向量

         则 \( A\alpha = \lambda\alpha \), 左乘 \( A \), 为
         \[ A^{2}\alpha = \lambda A\alpha = \lambda^{2}\alpha, \text{ 又 } A^{2} = A \]

         则 \( (\lambda^{2}-\lambda)\alpha = 0 \) 又 \( \alpha \neq 0 \), \( \therefore \lambda^{2}-\lambda = 0 \)

         于是 \( A \) 的特征值只能是0或1

         又秩 \( r(A) = r \), 故特征值为0的特征向量有 \( n-r \) 个

         则 \( A \) 的相似标准形为 \( \begin{pmatrix}
             \overbrace{\begin{matrix}
                                1 &        &   \\
                                  & \ddots &   \\
                                  &        & 1
                            \end{matrix}}^{r \text{ 个}} &     \\
                                         & \begin{matrix}
                                               0 &        &   \\
                                                 & \ddots &   \\
                                                 &        & 0
                                           \end{matrix}
         \end{pmatrix}_{n} = \begin{pmatrix}
             E_{r} &   \\
                   & O
         \end{pmatrix} \)


     \paragraph{} %9
         设 \( A\alpha = \lambda\alpha \), 则 \( A^{2}\alpha = \lambda^{2}\alpha \Rightarrow \lambda = \pm 1 \)

         故仿8知 \( A \sim \begin{pmatrix}
             E_{r} &          \\
                   & -E_{n-r}
         \end{pmatrix} \)


     \paragraph{} %10
         反证法. 假设 \( A \) 可对角化, 设
         \[ A \sim \begin{pmatrix}
                 \lambda_{1} &             &                      \\
                             & \lambda_{2} &                      \\
                             &             & \ddots &             \\
                             &             &        & \lambda_{n}
             \end{pmatrix} = \Lambda, \text{ 即设 } A = P\Lambda P^{-1} \]

         由 \( A^{k} = O \) 得
         \[ A^{k} = (P\Lambda P^{-1})(P\Lambda P^{-1})\cdots(P\Lambda P^{-1}) = O \]
         \[ \Rightarrow P\Lambda^{k}P^{-1} = O \Rightarrow \Lambda^{k} = O \]

         则 \( \lambda_{1} = \lambda_{2} = \cdots = \lambda_{n} = 0 \), 则 \( A = O \), 与题设矛盾.


     \paragraph{} %11
         依题意, 设 \( \Lambda = \begin{pmatrix}
             a &   \\
               & b
         \end{pmatrix} \)

         则由 \( \begin{cases}
             |A| = |\Lambda| \\
             tr(A) = tr(\Lambda)
         \end{cases} \)

         得 \( \begin{cases}
             a+b = 6 \\
             ab = 8
         \end{cases} \Rightarrow \begin{cases}
             a = 2 \\
             b = 4
         \end{cases} \) 或 \( \begin{cases}
             a = 4 \\
             b = 2
         \end{cases} \)

         从而 \( \Lambda \) 为 \( \begin{pmatrix}
             2 &   \\
               & 4
         \end{pmatrix} \) 或 \( \begin{pmatrix}
             4 &   \\
               & 2
         \end{pmatrix} \)


 \subsection{} %C

     \paragraph{} %1
         \begin{enumerate}
             \item %(1)
                   设 \( X = \begin{pmatrix}
                       x_{1}(t) \\
                       x_{2}(t)
                   \end{pmatrix} \), \( A = \begin{pmatrix}
                       1 & 2 \\
                       4 & 8
                   \end{pmatrix} \), 则 \( X' = AX \)

                   又由 \( f_{A}(\lambda) = \begin{vmatrix}
                       \lambda-1 & -2        \\
                       -4        & \lambda-8
                   \end{vmatrix} = \lambda^{2}-9\lambda = \lambda(\lambda-9) \) 得特征值为 0,9.

                   又 \( \lambda=0 \) 时, 由 \( \begin{pmatrix}
                       -1 & -2 \\
                       -4 & -8
                   \end{pmatrix} \rightarrow \begin{pmatrix}
                       1 & 2 \\
                       0 & 0
                   \end{pmatrix} \) 得 \( \alpha_{1} = (-2, 1)^{\mathrm{T}} \)

                   又 \( 9E-A = \begin{pmatrix}
                       8  & -2 \\
                       -4 & 1
                   \end{pmatrix} \rightarrow \begin{pmatrix}
                       -4 & 1 \\
                       0  & 0
                   \end{pmatrix} \) 得 \( \alpha_{2} = (1, 4)^{\mathrm{T}} \), 则 \( P = \begin{pmatrix}
                       -2 & 1 \\
                       1  & 4
                   \end{pmatrix} \), \( \Lambda = \begin{pmatrix}
                       0 &   \\
                         & 9
                   \end{pmatrix} \)

                   故由 \( X = P Y \) 可得 \( Y' = \begin{pmatrix}
                       0 &   \\
                         & 9
                   \end{pmatrix} Y \Rightarrow \begin{cases}
                       \frac{\mathrm{d}my_{1}}{\mathrm{d}t} = 0 \\
                       \frac{\mathrm{d}y_{2}}{\mathrm{d}t} = 9y_{2}
                   \end{cases} \Rightarrow Y = \begin{pmatrix}
                       c_{1} \\
                       c_{2}e^{9t}
                   \end{pmatrix} \)

                   则 \( X = \begin{pmatrix}
                       -2 & 1 \\
                       1  & 4
                   \end{pmatrix}\begin{pmatrix}
                       c_{1} \\
                       c_{2}e^{9t}
                   \end{pmatrix} = \begin{pmatrix}
                       -2c_{1} + c_{2}e^{9t} \\
                       c_{1} + 4c_{2}e^{9t}
                   \end{pmatrix} \)
             \item %(2)
                   设 \( X = \begin{pmatrix}
                       x_{1}(t) \\
                       x_{2}(t)
                   \end{pmatrix} \), \( A = \begin{pmatrix}
                       2  & -1 \\
                       -1 & 2
                   \end{pmatrix} \), 则 \( X' = AX \)

                   由 \( f_{A}(\lambda) = \begin{vmatrix}
                       \lambda-2 & 1         \\
                       1         & \lambda-2
                   \end{vmatrix} = (\lambda-1)(\lambda-3) \) 得特征值 1,3

                   则 \( E-A = \begin{pmatrix}
                       -1 & 1  \\
                       1  & -1
                   \end{pmatrix} \rightarrow \begin{pmatrix}
                       1 & -1 \\
                       0 & 0
                   \end{pmatrix} \) 得 \( \alpha_{1} = (1, 1)^{\mathrm{T}} \)

                   \( 3E-A = \begin{pmatrix}
                       1 & 1 \\
                       1 & 1
                   \end{pmatrix} \rightarrow \begin{pmatrix}
                       1 & 1 \\
                       0 & 0
                   \end{pmatrix} \) 得 \( \alpha_{2} = (1, -1)^{\mathrm{T}} \), 则 \( P = \begin{pmatrix}
                       1 & 1  \\
                       1 & -1
                   \end{pmatrix} \), \( \Lambda = \begin{pmatrix}
                       1 &   \\
                         & 3
                   \end{pmatrix} \)

                   由 \( X = P Y \) 可得 \( Y' = \begin{pmatrix}
                       1 &   \\
                         & 3
                   \end{pmatrix} Y \Rightarrow \begin{cases}
                       \frac{\mathrm{d}y_{1}}{\mathrm{d}t} = y_{1} \\
                       \frac{\mathrm{d}y_{2}}{\mathrm{d}t} = 3y_{2}
                   \end{cases} \Rightarrow Y = \begin{pmatrix}
                       c_{1}e^{\mathrm{T}} \\
                       c_{2}e^{3t}
                   \end{pmatrix} \)

                   故 \( X = \begin{pmatrix}
                       1 & 1  \\
                       1 & -1
                   \end{pmatrix}\begin{pmatrix}
                       c_{1}e^{t} \\
                       c_{2}e^{3t}
                   \end{pmatrix} = \begin{pmatrix}
                       c_{1}e^{t} + c_{2}e^{3t} \\
                       c_{1}e^{t} - c_{2}e^{3t}
                   \end{pmatrix} \)
             \item %(3)
                   设 \( X = \begin{pmatrix}
                       x_{1}(t) \\
                       x_{2}(t)
                   \end{pmatrix} \), \( A = \begin{pmatrix}
                       1  & -2 \\
                       -1 & 4
                   \end{pmatrix} \), 则 \( X' = AX \)

                   由 \( f_{A}(\lambda) = \begin{vmatrix}
                       \lambda-1 & 2         \\
                       -1        & \lambda-4
                   \end{vmatrix} = (\lambda-2)(\lambda-3) \) 得特征值2,3.

                   则 \( 2E-A = \begin{pmatrix}
                       1  & 2  \\
                       -1 & -2
                   \end{pmatrix} \rightarrow \begin{pmatrix}
                       1 & 2 \\
                       0 & 0
                   \end{pmatrix} \) 得 \( \alpha_{1} = (-2, 1)^{\mathrm{T}} \)

                   \( 3E-A = \begin{pmatrix}
                       2  & 2  \\
                       -1 & -1
                   \end{pmatrix} \rightarrow \begin{pmatrix}
                       1 & 1 \\
                       0 & 0
                   \end{pmatrix} \) 得 \( \alpha_{2} = (1, -1)^{\mathrm{T}} \), 则 \( P = \begin{pmatrix}
                       -2 & 1  \\
                       1  & -1
                   \end{pmatrix} \), \( \Lambda = \begin{pmatrix}
                       2 &   \\
                         & 3
                   \end{pmatrix} \)

                   则由 \( X = P Y \Rightarrow Y' = \begin{pmatrix}
                       2 &   \\
                         & 3
                   \end{pmatrix} Y \Rightarrow \begin{cases}
                       \frac{\mathrm{d}y_{1}}{\mathrm{d}t} = y_{1} \\
                       \frac{\mathrm{d}y_{2}}{\mathrm{d}t} = y_{2}
                   \end{cases} \Rightarrow Y = \begin{pmatrix}
                       c_{1}e^{2t} \\
                       c_{2}e^{3t}
                   \end{pmatrix} \)

                   \( \Rightarrow X = \begin{pmatrix}
                       -2 & 1  \\
                       1  & -1
                   \end{pmatrix}\begin{pmatrix}
                       c_{1}e^{2t} \\
                       c_{2}e^{3t}
                   \end{pmatrix} = \begin{pmatrix}
                       -2c_{1}e^{2t} + c_{2}e^{3t} \\
                       c_{1}e^{2t} - c_{2}e^{3t}
                   \end{pmatrix} \)

                   又 \( x_{1}(0) = x_{2}(0) = 1 \) 得 \( \begin{cases}
                       c_{1} = -2 \\
                       c_{2} = -3
                   \end{cases} \)

                   故 \( X = \begin{pmatrix}
                       4e^{2t} - 3e^{3t} \\
                       -2e^{2t} + 3e^{3t}
                   \end{pmatrix} \)
             \item %(4)
                   记 \( X = \begin{pmatrix}
                       x_{1}(t) \\
                       x_{2}(t) \\
                       x_{3}(t)
                   \end{pmatrix} \), \( A = \begin{pmatrix}
                       3   & 1  & 1  \\
                       -12 & 0  & 5  \\
                       4   & -2 & -1
                   \end{pmatrix} \), 则 \( X' = AX \)

                   由 \( f_{A}(\lambda) = |\lambda E - A| = \begin{vmatrix}
                       \lambda-3 & -1      & -1        \\
                       12        & \lambda & -5        \\
                       -4        & 2       & \lambda+1
                   \end{vmatrix} = \lambda^{3} - 2\lambda^{2} + 5\lambda - 62 \), 考察函数图像, 此方程有且只有一个实根. 故该微分方程组解的形式过于复杂, 但计算方法与(3)相同. 此处怀疑是题目未设置好. 但教材也没给出答案, 故不作过多讨论.
             \item %(5)
                   记 \( X = \begin{pmatrix}
                       x_{1}(t) \\
                       x_{2}(t) \\
                       x_{3}(t)
                   \end{pmatrix} \), \( A = \begin{pmatrix}
                       1  & 1 & 1 \\
                       0  & 3 & 3 \\
                       -2 & 1 & 1
                   \end{pmatrix} \), 则 \( X' = AX \)

                   由 \( f_{A}(\lambda) = |\lambda E - A| = \begin{vmatrix}
                       \lambda-1 & -1        & -1        \\
                       0         & \lambda-3 & -3        \\
                       -2        & -1        & \lambda-1
                   \end{vmatrix} = \lambda(\lambda-2)(\lambda-3) \) 故特征值为0,2,3.

                   则 \( 0E-A = \begin{pmatrix}
                       -1 & -1 & -1 \\
                       0  & -3 & -3 \\
                       2  & 1  & -1
                   \end{pmatrix} \rightarrow \begin{pmatrix}
                       1 &   &   \\
                         & 1 & 1 \\
                         &   &
                   \end{pmatrix} \) 则 \( \alpha_{1} = (0, -1, 1)^{\mathrm{T}} \)

                   \( 2E-A = \begin{pmatrix}
                       1  & -1 & -1 \\
                       0  & -1 & -3 \\
                       -2 & -1 & 1
                   \end{pmatrix} \rightarrow \begin{pmatrix}
                       1 &   & 2 \\
                         & 1 & 3 \\
                         &   &
                   \end{pmatrix} \) 则 \( \alpha_{2} = (2, 3, -1)^{\mathrm{T}} \)

                   \( 3E-A = \begin{pmatrix}
                       2 & -1 & -1 \\
                       0 & 0  & -3 \\
                       2 & -1 & 2
                   \end{pmatrix} \rightarrow \begin{pmatrix}
                       1 & -\frac{1}{2} &   \\
                         &              & 1 \\
                         &              &
                   \end{pmatrix} \) 则 \( \alpha_{3} = (1, 2, 0)^{\mathrm{T}} \)

                   故 \( P = \begin{pmatrix}
                       0  & 2  & 1 \\
                       -1 & 3  & 2 \\
                       1  & -1 & 0
                   \end{pmatrix} \), \( \Lambda = \begin{pmatrix}
                       0 &   &   \\
                         & 2 &   \\
                         &   & 3
                   \end{pmatrix} \), 则由 \( X = P Y \)

                   有 \( Y' = \begin{pmatrix}
                       0 &   &   \\
                         & 2 &   \\
                         &   & 3
                   \end{pmatrix} \begin{pmatrix}
                       y_{1} \\
                       y_{2} \\
                       y_{3}
                   \end{pmatrix} \Rightarrow \begin{cases}
                       \frac{\mathrm{d}y_{1}}{\mathrm{d}t} = 0      \\
                       \frac{\mathrm{d}y_{2}}{\mathrm{d}t} = 2y_{2} \\
                       \frac{\mathrm{d}y_{3}}{\mathrm{d}t} = 3y_{3}
                   \end{cases} \Rightarrow Y = \begin{pmatrix}
                       c_{1}       \\
                       c_{2}e^{2t} \\
                       c_{3}e^{3t}
                   \end{pmatrix} \)

                   故 \( X = \begin{pmatrix}
                       0  & 2  & 1 \\
                       -1 & 3  & 2 \\
                       1  & -1 & 0
                   \end{pmatrix}\begin{pmatrix}
                       c_{1}       \\
                       c_{2}e^{2t} \\
                       c_{3}e^{3t}
                   \end{pmatrix} = \begin{pmatrix}
                       2c_{2}e^{2t} + c_{3}e^{3t}           \\
                       -c_{1} + 3c_{2}e^{2t} + 2c_{3}e^{3t} \\
                       c_{1} - c_{2}e^{2t}
                   \end{pmatrix} \) 又 \( \begin{cases}
                       x_{1}(0) = 3 \\
                       x_{2}(0) = 3 \\
                       x_{3}(0) = 1
                   \end{cases} \)

                   得 \( \begin{cases}
                       2c_{2} + c_{3} = 3           \\
                       -c_{1} + 3c_{2} + 2c_{3} = 3 \\
                       c_{1} - c_{2} = 1
                   \end{cases} \Rightarrow \begin{pmatrix}
                       c_{1} \\
                       c_{2} \\
                       c_{3}
                   \end{pmatrix} = \begin{pmatrix}
                       2 \\
                       1 \\
                       1
                   \end{pmatrix} \) 故

                   \( X = \begin{pmatrix}
                       2e^{2t} + e^{3t}       \\
                       -2 + 3e^{2t} + 2e^{3t} \\
                       2 - e^{2t}
                   \end{pmatrix} \)
         \end{enumerate}


     \paragraph{} %2
         \begin{enumerate}
             \item %(1)
                   学习过高等数学(微积分/工科数学分析)可直接求 \(\lambda\) 用公式.

                   这里先使用线性代数方法示范.

                   令 \( x_{1}(t) = x(t) \), \( x_{2}(t) = x'(t) \)

                   则 \( \begin{cases}
                       x_{1}'(t) = x_{2}(t) \\
                       x_{2}'(t) = -3x_{1}(t) - 4x_{2}(t)
                   \end{cases} \) 记 \( X = \begin{pmatrix}
                       x_{1}(t) \\
                       x_{2}(t)
                   \end{pmatrix} \), \( A = \begin{pmatrix}
                       0  & 1  \\
                       -3 & -4
                   \end{pmatrix} \)

                   则 \( X' = AX \), 则由 \( f_{A}(\lambda) = |\lambda E - A| = (\lambda+1)(\lambda+3) \Rightarrow \lambda \) 为 -1, -3

                   则由 \( X = P Y \Rightarrow Y' = \begin{pmatrix}
                       -1 &    \\
                          & -3
                   \end{pmatrix}\begin{pmatrix}
                       y_{1} \\
                       y_{2}
                   \end{pmatrix} \Rightarrow Y = \begin{pmatrix}
                       c_{1}e^{-t} \\
                       c_{2}e^{-3t}
                   \end{pmatrix} \)

                   又由 \( -3\lambda-A = \begin{pmatrix}
                       -3 & -1 \\
                       3  & 1
                   \end{pmatrix} \rightarrow \begin{pmatrix}
                       3 & 1 \\
                       0 & 0
                   \end{pmatrix} \) 得 \( \alpha_{1} = (-1, 3)^{\mathrm{T}} \)

                   由 \( -\lambda-A = \begin{pmatrix}
                       -1 & -1 \\
                       3  & 1
                   \end{pmatrix} \rightarrow \begin{pmatrix}
                       3 & 1 \\
                       0 & 0
                   \end{pmatrix} \) 得 \( \alpha_{2} = (-1, 1)^{\mathrm{T}} \), 则 \( P = \begin{pmatrix}
                       -1 & -1 \\
                       1  & 3
                   \end{pmatrix} \)

                   故 \( X = \begin{pmatrix}
                       -1 & -1 \\
                       3  & 1
                   \end{pmatrix}\begin{pmatrix}
                       c_{1}e^{-t} \\
                       c_{2}e^{-3t}
                   \end{pmatrix} = \begin{pmatrix}
                       -c_{1}e^{-t} - c_{2}e^{-3t} \\
                       3c_{1}e^{-t} + c_{2}e^{-3t}
                   \end{pmatrix} \)

                   故 \( x = c_{1}e^{-t} + c_{2}e^{-3t} \)
             \item %(2)
                   由 \( \lambda^{2} - 5\lambda + 6 = 0 \Rightarrow \lambda = 2, 3 \)

                   则 \( x = c_{1}e^{2t} + c_{2}e^{3t} \)
             \item %(3)
                   由 \( \lambda^{2} - 7\lambda + 6 = 0 \Rightarrow \lambda = 6, 1 \)

                   则 \( x = c_{1}e^{6t} + c_{2}e^{t} \) 又 \( \begin{cases}
                       x(0) = 1 \\
                       x'(0) = 1
                   \end{cases} \Rightarrow \begin{cases}
                       c_{1} + c_{2} = 1 \\
                       6c_{1} + c_{2} = 1
                   \end{cases} \)

                   得 \( x = e^{t} \)
             \item %(4)
                   由 \( \lambda^{2} - 6\lambda + 8 = 0 \Rightarrow \lambda = 2, 4 \) 又 \( D^{2}x - 6Dx + 8x = 8t + 10 \)

                   得 \( x = \frac{8t+10}{(D-2)(D-4)} \) 由 \( L(D) = D^{2}-6D+8 \) 且 \( L(0) \neq 0 \)

                   得 \( x = \frac{8t+10}{D^{2}-6D+8} = \left( \frac{1}{8} + \frac{3}{32}D \right)(8t+10) = t + 2 \)

                   故特解为 \( x^{*} = t + 2 \)

                   故 \( x = c_{1}e^{2t} + c_{2}e^{4t} + t + 2 \)
         \end{enumerate}


     \paragraph{} %3
         \begin{enumerate}
             \item %(1)
                   同2. 先用线性代数思想示范, 然后回归高数公式解题.

                   令 \( x_{1}(t) = x(t) \), \( x_{2}(t) = x'(t) \)

                   则记 \( X = \begin{pmatrix}
                       x_{1}(t) \\
                       x_{2}(t)
                   \end{pmatrix} \), \( X' = \begin{pmatrix}
                       x_{1}'(t) \\
                       x_{2}'(t)
                   \end{pmatrix} = \begin{pmatrix}
                                  & x_{2}(t) \\
                       -4x_{1}(t) &
                   \end{pmatrix} = AX \)

                   \( A = \begin{pmatrix}
                       0  & 1 \\
                       -4 & 0
                   \end{pmatrix} \) 则 \( f_{A}(\lambda) = |\lambda E - A| = \begin{vmatrix}
                       \lambda & -1      \\
                       4       & \lambda
                   \end{vmatrix} = \lambda^{2} + 4 \) 得入为 \( \pm 2i \)

                   \( \lambda = 2i \), \( 2iE - A = \begin{pmatrix}
                       2i & -1 \\
                       4  & 2i
                   \end{pmatrix} \rightarrow \begin{pmatrix}
                       2i & -1 \\
                       0  & 0
                   \end{pmatrix} \) 得 \( \alpha_{1} = (1, 2i)^{\mathrm{T}} \)

                   \( \lambda = -2i \), \( -2iE - A = \begin{pmatrix}
                       -2i & -1  \\
                       4   & -2i
                   \end{pmatrix} \rightarrow \begin{pmatrix}
                       2i & 1 \\
                       0  & 0
                   \end{pmatrix} \) 得 \( \alpha_{2} = (-1, 2i)^{\mathrm{T}} \) 则 \( P = \begin{pmatrix}
                       1  & -1 \\
                       2i & 2i
                   \end{pmatrix} \)

                   \( \Lambda = \begin{pmatrix}
                       2i &     \\
                          & -2i
                   \end{pmatrix} \), 则由 \( X = P Y \) 得 \( Y' = \begin{pmatrix}
                       2i &     \\
                          & -2i
                   \end{pmatrix} Y \Rightarrow Y = \begin{pmatrix}
                       c_{1}e^{2it} \\
                       c_{2}e^{-2it}
                   \end{pmatrix} \)

                   故 \( X = \begin{pmatrix}
                       1  & -1 \\
                       2i & 2i
                   \end{pmatrix}\begin{pmatrix}
                       c_{1}e^{2it} \\
                       c_{2}e^{-2it}
                   \end{pmatrix} = \begin{pmatrix}
                       c_{1}e^{2it} - c_{2}e^{-2it} \\
                       2ic_{1}e^{2it} + 2ic_{2}e^{-2it}
                   \end{pmatrix} \)

                   又由欧拉公式, 则 \( x = c_{1}e^{2it} - c_{2}e^{-2it} = (c_{1}+c_{2})\cos 2t + (c_{1}-c_{2})i\sin 2t \)

                   故 \( x = C_{1}\cos 2t + C_{2}\sin 2t \)
             \item %(2)
                   由 \( \lambda^{2} - 6\lambda + 10 = 0 \Rightarrow \lambda = 3 \pm i \)

                   故 \( x = e^{3t}(C_{1}\cos t + C_{2}\sin t) \)
             \item %(3)
                   \( \begin{pmatrix}
                       x_{1}'(t) \\
                       x_{2}'(t)
                   \end{pmatrix} = \begin{pmatrix}
                              & x_{2} \\
                       -x_{1} &
                   \end{pmatrix} = \begin{pmatrix}
                       0 & -1 \\
                       1 & 0
                   \end{pmatrix}\begin{pmatrix}
                       x_{1} \\
                       x_{2}
                   \end{pmatrix} \), 记 \( X' = AX \)

                   则 \( |\lambda E - A| = \begin{vmatrix}
                       \lambda & -1      \\
                       1       & \lambda
                   \end{vmatrix} = \lambda^{2} + 1 \) 得 \(\lambda\) 为 \( \pm i \)

                   \( iE - A = \begin{pmatrix}
                       i  & 1 \\
                       -1 & i
                   \end{pmatrix} \rightarrow \begin{pmatrix}
                       i & -1 \\
                       0 & 0
                   \end{pmatrix} \) 得 \( \alpha_{1} = (-1, i)^{\mathrm{T}} \)

                   \( -iE - A = \begin{pmatrix}
                       -i & 1  \\
                       1  & -i
                   \end{pmatrix} \rightarrow \begin{pmatrix}
                       i & -1 \\
                       0 & 0
                   \end{pmatrix} \) 得 \( \alpha_{2} = (1, i)^{\mathrm{T}} \), 则 \( P = \begin{pmatrix}
                       -1 & 1 \\
                       i  & i
                   \end{pmatrix} \), \( \Lambda = \begin{pmatrix}
                       i &    \\
                         & -i
                   \end{pmatrix} \)

                   则 \( Y' = \Lambda Y = \begin{pmatrix}
                       i &    \\
                         & -i
                   \end{pmatrix}\begin{pmatrix}
                       y_{1} \\
                       y_{2}
                   \end{pmatrix} \Rightarrow Y = \begin{pmatrix}
                       c_{1}e^{it} \\
                       c_{2}e^{-it}
                   \end{pmatrix} \)

                   则 \( X = \begin{pmatrix}
                       -1 & 1 \\
                       i  & i
                   \end{pmatrix}\begin{pmatrix}
                       c_{1}e^{it} \\
                       c_{2}e^{-it}
                   \end{pmatrix} = \begin{pmatrix}
                       -c_{1}e^{it} + c_{2}e^{-it} \\
                       ic_{1}e^{it} + ic_{2}e^{-it}
                   \end{pmatrix} = \begin{pmatrix}
                       (c_{2}-c_{1})\cos t - (c_{1}+c_{2})i\sin t \\
                       (c_{1}+c_{2})\cos t - (c_{1}-c_{2})i\sin t
                   \end{pmatrix} \)

                   故 \( x_{1} = C_{1}\cos t + C_{2}\sin t \), \( x_{2} = -C_{2}\cos t + C_{1}\sin t \)
             \item %(4)
                   记 \( X = \begin{pmatrix}
                       x_{1}(t) \\
                       x_{2}(t)
                   \end{pmatrix} \) 则 \( X' = \begin{pmatrix}
                       x_{1}'(t) \\
                       x_{2}'(t)
                   \end{pmatrix} = \begin{pmatrix}
                       x_{2} \\
                       -8x_{1} - 4x_{2}
                   \end{pmatrix} = \begin{pmatrix}
                       0  & 1  \\
                       -8 & -4
                   \end{pmatrix}\begin{pmatrix}
                       x_{1} \\
                       x_{2}
                   \end{pmatrix} \)

                   则由 \( f_{A}(\lambda) = \begin{vmatrix}
                       \lambda & -1        \\
                       8       & \lambda+4
                   \end{vmatrix} = \lambda^{2} + 4\lambda + 8 \) 得 \( \lambda \) 为 \( -2 \pm 2i \)

                   \( (-2+2i)E - A = \begin{pmatrix}
                       -2+2i & -1   \\
                       8     & 2+2i
                   \end{pmatrix} \rightarrow \begin{pmatrix}
                       -2+2i & -1 \\
                       0     & 0
                   \end{pmatrix} \) 则 \( \alpha_{1} = (1, -2+2i)^{\mathrm{T}} \)

                   \( (-2-2i)E - A = \begin{pmatrix}
                       -2-2i & -1   \\
                       8     & 2-2i
                   \end{pmatrix} \rightarrow \begin{pmatrix}
                       2+2i & 1 \\
                       0    & 0
                   \end{pmatrix} \) 则 \( \alpha_{2} = (1, -2-2i)^{\mathrm{T}} \)

                   故 \( P = \begin{pmatrix}
                       1     & 1     \\
                       -2+2i & -2-2i
                   \end{pmatrix} \), \( \Lambda = \begin{pmatrix}
                       -2+2i &       \\
                             & -2-2i
                   \end{pmatrix} \)

                   由 \( X = P Y \) 得 \( Y' = \begin{pmatrix}
                       -2+2i &       \\
                             & -2-2i
                   \end{pmatrix} Y \Rightarrow Y = \begin{pmatrix}
                       c_{1}e^{-2+2it} \\
                       c_{2}e^{-2-2it}
                   \end{pmatrix} \)

                   则 \( X = \begin{pmatrix}
                       1     & 1     \\
                       -2+2i & -2-2i
                   \end{pmatrix}\begin{pmatrix}
                       c_{1}e^{-2+2it} \\
                       c_{2}e^{-2-2it}
                   \end{pmatrix} = \begin{pmatrix}
                       c_{1}e^{-2+2it} + c_{2}e^{-2-2it} \\
                       (-2+2ic_{1})e^{-2+2it} + (-2-2ic_{2})e^{-2-2it}
                   \end{pmatrix} \)

                   即 \( \begin{cases} x_{1}(t) = e^{-2t}\left( C_{1}\cos 2x + C_{2}\sin 2x \right) \\
                       x_{2}(t) = e^{-2t}\left[ C_{2}(-2C_{2}-2C_{1})\cos 2x - 2(C_{1}+C_{2})\sin 2x \right]\end{cases} \)
         \end{enumerate}


\section{5.3}

 \subsection{} %A

     \paragraph{} %1
         \begin{enumerate}
             \item %(1)
                   \( f_{A}(\lambda) = \begin{vmatrix}
                       \lambda-1 & 2         \\
                       2         & \lambda-1
                   \end{vmatrix} = (\lambda+1)(\lambda-3) \), 得特征值为 \( -1, 3 \)

                   \( 3E-A = \begin{pmatrix}
                       2 & 2 \\
                       2 & 2
                   \end{pmatrix} \rightarrow \begin{pmatrix}
                       1 & 1 \\
                         &
                   \end{pmatrix} \) 得基础解系 \( \alpha_{1} = (1, -1)^{\mathrm{T}} \)

                   \( -E-A = \begin{pmatrix}
                       -2 & 2  \\
                       2  & -2
                   \end{pmatrix} \rightarrow \begin{pmatrix}
                       1 & -1 \\
                         &
                   \end{pmatrix} \) 得 \( \alpha_{2} = (1, 1)^{\mathrm{T}} \) 单位化 \( \begin{cases}
                       p_{1} = \left( \frac{1}{\sqrt{2}}, -\frac{1}{\sqrt{2}} \right)^{\mathrm{T}} \\
                       p_{2} = \left( \frac{1}{\sqrt{2}}, \frac{1}{\sqrt{2}} \right)^{\mathrm{T}}
                   \end{cases} \)

                   故 \( Q = \begin{pmatrix}
                       \frac{1}{\sqrt{2}}  & \frac{1}{\sqrt{2}} \\
                       -\frac{1}{\sqrt{2}} & \frac{1}{\sqrt{2}}
                   \end{pmatrix} \), \( Q^{-1}AQ = \operatorname{diag}(3, -1) \).
             \item %(2)
                   \( f_{A}(\lambda) = \begin{vmatrix}
                       \lambda & 6         & -6        \\
                       6       & \lambda+3 & 0         \\
                       -6      & 0         & \lambda-3
                   \end{vmatrix} = \lambda(\lambda-9)(\lambda+9) \), 特征值为 \( -9, 0, 9 \)

                   \( 0E-A = \begin{pmatrix}
                       0  & 6 & -6 \\
                       6  & 3 &    \\
                       -6 &   & -3
                   \end{pmatrix} \rightarrow \begin{pmatrix}
                       1 &   & -\frac{1}{2} \\
                         & 1 & -1           \\
                         &   &
                   \end{pmatrix} \) 得 \( \alpha_{1} = (-1, 1, 2)^{\mathrm{T}} \)

                   \( 9E-A = \begin{pmatrix}
                       9  & 6  & -6 \\
                       6  & 12 &    \\
                       -6 &    & 6
                   \end{pmatrix} \rightarrow \begin{pmatrix}
                       1 &   & -1          \\
                         & 1 & \frac{1}{2} \\
                         &   &
                   \end{pmatrix} \) 得 \( \alpha_{2} = (2, -1, 2)^{\mathrm{T}} \)

                   \( -9E-A = \begin{pmatrix}
                       -9 & 6  & -6  \\
                       6  & -6 &     \\
                       -6 &    & -12
                   \end{pmatrix} \rightarrow \begin{pmatrix}
                       1 &   & \frac{1}{2} \\
                         & 1 & -1          \\
                         &   &
                   \end{pmatrix} \) 得 \( \alpha_{3} = (-2, -2, 1)^{\mathrm{T}} \)

                   则 \( p_{1} = \left( -\frac{1}{3}, \frac{1}{3}, \frac{2}{3} \right)^{\mathrm{T}} \), \(p_{2} = \left( \frac{2}{3}, -\frac{1}{3}, \frac{2}{3} \right)^{\mathrm{T}} \), \( p_{3} = \left( -\frac{2}{3}, -\frac{2}{3}, \frac{1}{3} \right)^{\mathrm{T}} \)

                   故 \( Q = \begin{pmatrix}
                       -\frac{1}{3} & \frac{2}{3}  & -\frac{2}{3} \\
                       \frac{1}{3}  & -\frac{1}{3} & -\frac{2}{3} \\
                       \frac{2}{3}  & \frac{2}{3}  & \frac{1}{3}
                   \end{pmatrix} \), \( Q^{\mathrm{T}}AQ = \operatorname{diag}(0, 9, -9) \).
             \item %(3)
                   \( f_{A}(\lambda) = \begin{vmatrix}
                       \lambda-4 & -2        &           \\
                       -2        & \lambda-3 & 2         \\
                                 & 2         & \lambda-2
                   \end{vmatrix} = \lambda(\lambda-3)(\lambda-6) \), 特征值为 \( 0, 3, 6 \)

                   \( 6E-A = \begin{pmatrix}
                       2  & -2 &   \\
                       -2 & 3  & 2 \\
                          & 2  & 4
                   \end{pmatrix} \rightarrow \begin{pmatrix}
                       1 &   & 2 \\
                         & 1 & 2 \\
                         &   &
                   \end{pmatrix} \) 得 \( \alpha_{1} = (2, 2, -1)^{\mathrm{T}} \)

                   \( 3E-A = \begin{pmatrix}
                       -1 & -2 &   \\
                       -2 & 0  & 2 \\
                          & 2  & 1
                   \end{pmatrix} \rightarrow \begin{pmatrix}
                       1 &   & -1          \\
                         & 1 & \frac{1}{2} \\
                         &   &
                   \end{pmatrix} \) 得 \( \alpha_{2} = (2, -1, 2)^{\mathrm{T}} \)

                   \( 0E-A = \begin{pmatrix}
                       -4 & -2 &    \\
                       -2 & -3 & 2  \\
                          & 2  & -2
                   \end{pmatrix} \rightarrow \begin{pmatrix}
                       1 &   & \frac{1}{2} \\
                         & 1 & -1          \\
                         &   &
                   \end{pmatrix} \) 得 \( \alpha_{3} = (-1, 2, 2)^{\mathrm{T}} \)

                   则 \( p_{1} = \left( -\frac{2}{3}, -\frac{1}{3}, \frac{1}{3} \right)^{\mathrm{T}} \), \( p_{2} = \left( -\frac{2}{3}, \frac{1}{3}, -\frac{2}{3} \right)^{\mathrm{T}} \), \( p_{3} = \left( \frac{1}{3}, -\frac{2}{3}, -\frac{2}{3} \right)^{\mathrm{T}} \)

                   则 \( Q = \begin{pmatrix}
                       -\frac{2}{3} & -\frac{2}{3} & \frac{1}{3}  \\
                       -\frac{1}{3} & \frac{1}{3}  & -\frac{2}{3} \\
                       \frac{1}{3}  & -\frac{2}{3} & -\frac{2}{3}
                   \end{pmatrix} \), \( Q^{\mathrm{T}}AQ = \operatorname{diag}(6, 3, 0) \)
             \item %(4)
                   \( f_{A}(\lambda) = |\lambda E - A| = \begin{vmatrix}
                       \lambda-2 & 2         & 2         \\
                       2         & \lambda-5 & -4        \\
                       2         & -4        & \lambda-5
                   \end{vmatrix} = (\lambda-1)^{2}(\lambda-10) \), 特征值为 \( 1, 1, 10 \)

                   由 \( E-A = \begin{pmatrix}
                       -1 & 2  & 2  \\
                       2  & -4 & -4 \\
                       2  & -4 & -4
                   \end{pmatrix} \rightarrow \begin{pmatrix}
                       1 & -2 & -2 \\
                         &    &    \\
                         &    &
                   \end{pmatrix} \) 则 \( \alpha_{1} = (2, 1, 0)^{\mathrm{T}} \), \( \alpha_{2} = (2, 0, 1)^{\mathrm{T}} \)

                   \( 10E-A = \begin{pmatrix}
                       8 & 2  & 2  \\
                       2 & 5  & -4 \\
                       2 & -4 & 5
                   \end{pmatrix} \rightarrow \begin{pmatrix}
                       1 &   & \frac{1}{2} \\
                         & 1 & -1          \\
                         &   &
                   \end{pmatrix} \) 则 \( \alpha_{3} = (1, -2, -2)^{\mathrm{T}} \)

                   由 \( \alpha_{1}, \alpha_{2}, \alpha_{3} \) 正交单位化, 有 \( p_{1} = \left( \frac{2}{\sqrt{5}}, \frac{1}{\sqrt{5}}, 0 \right)^{\mathrm{T}} \), \( p_{2} = \left( \frac{2}{3\sqrt{5}}, -\frac{4}{3\sqrt{5}}, \frac{\sqrt{5}}{3} \right)^{\mathrm{T}} \), \( p_{3} = \left( -\frac{1}{3}, \frac{2}{3}, \frac{2}{3} \right)^{\mathrm{T}} \)

                   则 \( Q = \begin{pmatrix}
                       \frac{2}{\sqrt{5}} & \frac{2}{3\sqrt{5}}  & -\frac{1}{3} \\
                       \frac{1}{\sqrt{5}} & -\frac{4}{3\sqrt{5}} & \frac{2}{3}  \\
                       0                  & \frac{\sqrt{5}}{3}   & \frac{2}{3}
                   \end{pmatrix} \), \( Q^{-1}AQ = \operatorname{diag}(1, 1, 10) \)
             \item %(5)
                   \( f_{A}(\lambda) = |\lambda E - A| = \begin{vmatrix}
                       \lambda-1 & -2        & -2        \\
                       -2        & \lambda-1 & -2        \\
                       -2        & -2        & \lambda-1
                   \end{vmatrix} = (\lambda+1)^{2}(\lambda-5) \), 特征值为 \( 5, -1, -1 \)

                   \( 5E-A = \begin{pmatrix}
                       4  & -2 & -2 \\
                       -2 & 4  & -2 \\
                       -2 & -2 & 4
                   \end{pmatrix} \rightarrow \begin{pmatrix}
                       1 &   & -1 \\
                         & 1 & -1 \\
                         &   &
                   \end{pmatrix} \) 则 \( \alpha_{1} = (1, 1, 1)^{\mathrm{T}} \)

                   \( -E-A = \begin{pmatrix}
                       -2 & -2 & -2 \\
                       -2 & -2 & -2 \\
                       -2 & -2 & -2
                   \end{pmatrix} \rightarrow \begin{pmatrix}
                       1 & 1 & 1 \\
                         &   &   \\
                         &   &
                   \end{pmatrix} \) 则 \( \alpha_{2} = (1, -1, 0)^{\mathrm{T}} \), \( \alpha_{3} = (1, 0, -1)^{\mathrm{T}} \)

                   正交单位化, 有 \( p_{1} = \left( \frac{1}{\sqrt{3}}, \frac{1}{\sqrt{3}}, \frac{1}{\sqrt{3}} \right)^{\mathrm{T}} \), \( p_{2} = \left( -\frac{1}{\sqrt{6}}, \frac{2}{\sqrt{6}}, -\frac{1}{\sqrt{6}} \right)^{\mathrm{T}} \), \( p_{3} = \left( \frac{1}{\sqrt{2}}, 0, -\frac{1}{\sqrt{2}} \right)^{\mathrm{T}} \)

                   则 \( Q = \begin{pmatrix}
                       \frac{1}{\sqrt{3}} & -\frac{1}{\sqrt{6}} & \frac{1}{\sqrt{2}}  \\
                       \frac{1}{\sqrt{3}} & \frac{2}{\sqrt{6}}  & 0                   \\
                       \frac{1}{\sqrt{3}} & -\frac{1}{\sqrt{6}} & -\frac{1}{\sqrt{2}}
                   \end{pmatrix} \), \( Q^{-1}AQ = \operatorname{diag}(5, -1, -1) \)
             \item %(6)
                   \( f_{A}(\lambda) = |\lambda E - A| = \begin{vmatrix}
                       \lambda-1 &           & 1         \\
                                 & \lambda-1 &           \\
                       1         &           & \lambda-1
                   \end{vmatrix} = \lambda(\lambda-1)(\lambda-2) \), 特征值为 \( 0, 1, 2 \)

                   又由 \( 0E-A = \begin{pmatrix}
                       -1 &    & 1  \\
                          & -1 &    \\
                       1  &    & -1
                   \end{pmatrix} \rightarrow \begin{pmatrix}
                       1 &   & -1 \\
                         & 1 &    \\
                         &   &
                   \end{pmatrix} \) 得 \( \alpha_{1} = (1, 0, 1)^{\mathrm{T}} \)

                   \( E-A = \begin{pmatrix}
                       0 &   & 1 \\
                         & 0 &   \\
                       1 &   & 0
                   \end{pmatrix} \rightarrow \begin{pmatrix}
                       1 & 0 & 0 \\
                         &   & 1 \\
                         &   &
                   \end{pmatrix} \) 得 \( \alpha_{2} = (0, 1, 0)^{\mathrm{T}} \)

                   \( 2E-A = \begin{pmatrix}
                       1 &   & 1 \\
                         & 1 &   \\
                       1 &   & 1
                   \end{pmatrix} \rightarrow \begin{pmatrix}
                       1 &   & 1 \\
                         & 1 &   \\
                         &   &
                   \end{pmatrix} \) 得 \( \alpha_{3} = (1, 0, -1)^{\mathrm{T}} \)

                   单位化, 得 \( p_{1} = \left( \frac{1}{\sqrt{2}}, 0, \frac{1}{\sqrt{2}} \right)^{\mathrm{T}} \), \( p_{2} = (0, 1, 0)^{\mathrm{T}} \), \( p_{3} = \left( \frac{1}{\sqrt{2}}, 0, -\frac{1}{\sqrt{2}} \right)^{\mathrm{T}} \)

                   则 \( Q = \begin{pmatrix}
                       \frac{1}{\sqrt{2}} & 0 & \frac{1}{\sqrt{2}}  \\
                       0                  & 1 & 0                   \\
                       \frac{1}{\sqrt{2}} & 0 & -\frac{1}{\sqrt{2}}
                   \end{pmatrix} \), \( Q^{-1}AQ = \operatorname{diag}(0, 1, 2) \)
             \item %(7)
                   \( f_{A}(\lambda) = \begin{vmatrix}
                       \lambda & -1      & -1      & 1       \\
                       -1      & \lambda & 1       & -1      \\
                       -1      & 1       & \lambda & -1      \\
                       1       & -1      & -1      & \lambda
                   \end{vmatrix} = (\lambda+3)(\lambda-1)^{3} \), 特征值为 \( 1, 1, 1, -3 \)

                   \( E-A = \begin{pmatrix}
                       1  & -1 & -1 & 1  \\
                       -1 & 1  & 1  & -1 \\
                       -1 & 1  & 1  & -1 \\
                       1  & -1 & -1 & 1
                   \end{pmatrix} \rightarrow \begin{pmatrix}
                       1 & -1 & -1 & 1 \\
                         &    &    &   \\
                         &    &    &   \\
                         &    &    &
                   \end{pmatrix} \) 得 \( \alpha_{1} = (1, 1, 0, 0)^{\mathrm{T}} \), \( \alpha_{2} = (1, 0, 1, 0)^{\mathrm{T}} \), \( \alpha_{3} = (1, 0, 0, -1)^{\mathrm{T}} \)

                   \( -3E-A = \begin{pmatrix}
                       -3 & -1 & -1 & 1  \\
                       -1 & -3 & 1  & -1 \\
                       -1 & 1  & -3 & -1 \\
                       1  & -1 & -1 & -3
                   \end{pmatrix} \rightarrow \begin{pmatrix}
                       1 &   &   & -1 \\
                         & 1 &   & 1  \\
                         &   & 1 & 1  \\
                         &   &   &
                   \end{pmatrix} \) 得 \( \alpha_{4} = (1, -1, -1, 1)^{\mathrm{T}} \)

                   正交单位化为 \( p_{1} = \left( \frac{1}{\sqrt{2}}, \frac{1}{\sqrt{2}}, 0, 0 \right)^{\mathrm{T}} \), \( p_{2} = \left( \frac{1}{\sqrt{6}}, -\frac{1}{\sqrt{6}}, \frac{2}{\sqrt{6}}, 0 \right)^{\mathrm{T}} \),

                   \( p_{3} = \left( -\frac{1}{2\sqrt{3}}, \frac{1}{2\sqrt{3}}, \frac{1}{2\sqrt{3}}, \frac{\sqrt{3}}{2} \right)^{\mathrm{T}} \), \( p_{4} = \left( \frac{1}{2}, -\frac{1}{2}, -\frac{1}{2}, \frac{1}{2} \right)^{\mathrm{T}} \)

                   则 \( Q = \begin{pmatrix}
                       \frac{1}{\sqrt{2}} & \frac{1}{\sqrt{6}}  & -\frac{1}{2\sqrt{3}} & \frac{1}{2}  \\
                       \frac{1}{\sqrt{2}} & -\frac{1}{\sqrt{6}} & \frac{1}{2\sqrt{3}}  & -\frac{1}{2} \\
                       0                  & \frac{1}{\sqrt{6}}  & \frac{1}{2\sqrt{3}}  & -\frac{1}{2} \\
                       0                  & 0                   & \frac{\sqrt{3}}{2}   & \frac{1}{2}
                   \end{pmatrix} \), \( Q^{-1}AQ = \operatorname{diag}(1, 1, 1, -3) \)
             \item %(8)
                   \( f_{A}(\lambda) = \begin{vmatrix}
                       \lambda & -1      &         &         \\
                       -1      & \lambda &         &         \\
                               &         & \lambda & -1      \\
                               &         & -1      & \lambda
                   \end{vmatrix} = (\lambda-1)^{2}(\lambda+1)^{2} \), 则特征值为 \( -1, -1, 1, 1 \)

                   \( -E-A = \begin{pmatrix}
                       -1 & -1 &    &    \\
                       -1 & -1 &    &    \\
                          &    & -1 & -1 \\
                          &    & -1 & -1
                   \end{pmatrix} \rightarrow \begin{pmatrix}
                       1 & -1 &   &    \\
                         &    & 1 & -1 \\
                         &    &   &    \\
                         &    &   &
                   \end{pmatrix} \) 则 \( \begin{cases} \alpha_{1} = (-1, 1, 0, 0)^{\mathrm{T}}, \\ \alpha_{2} = (0, 0, -1, 1)^{\mathrm{T}}, \end{cases} \)

                   \( E-A = \begin{pmatrix}
                       1  & -1 &    &    \\
                       -1 & 1  &    &    \\
                          &    & 1  & -1 \\
                          &    & -1 & 1
                   \end{pmatrix} \rightarrow \begin{pmatrix}
                       1 & -1 &   &    \\
                         &    & 1 & -1 \\
                         &    &   &    \\
                         &    &   &
                   \end{pmatrix} \) 则 \( \begin{cases} \alpha_{3} = (1, 1, 0, 0)^{\mathrm{T}}, \\ \alpha_{4} = (0, 0, 1, 1)^{\mathrm{T}}, \end{cases} \)

                   故 \( p_{1} = \left( -\frac{1}{\sqrt{2}}, \frac{1}{\sqrt{2}}, 0, 0 \right)^{\mathrm{T}} \), \( p_{2} = \left( 0, 0, -\frac{1}{\sqrt{2}}, \frac{1}{\sqrt{2}} \right)^{\mathrm{T}} \),

                   \( p_{3} = \left( \frac{1}{\sqrt{2}}, \frac{1}{\sqrt{2}}, 0, 0 \right)^{\mathrm{T}} \), \( p_{4} = \left( 0, 0, \frac{1}{\sqrt{2}}, \frac{1}{\sqrt{2}} \right)^{\mathrm{T}} \)

                   则 \( Q = \begin{pmatrix}
                       -\frac{1}{\sqrt{2}} & 0                   & \frac{1}{\sqrt{2}} & 0                  \\
                       \frac{1}{\sqrt{2}}  & 0                   & \frac{1}{\sqrt{2}} & 0                  \\
                       0                   & -\frac{1}{\sqrt{2}} & 0                  & \frac{1}{\sqrt{2}} \\
                       0                   & \frac{1}{\sqrt{2}}  & 0                  & \frac{1}{\sqrt{2}}
                   \end{pmatrix} \), \( Q^{-1}AQ = \operatorname{diag}(-1, -1, 1, 1) \)
         \end{enumerate}


     \paragraph{} %2
         证明: 依题意, 存在 \( Q \), 使 \( Q^{-1}AQ = \operatorname{diag}(a_{1}, a_{2}, \dots, a_{n}) \)

         由于 \( r(A) = r \), 则 \( a_{1}, a_{2}, \dots, a_{n} \) 中有 \( r \) 个非零, 不妨设为前 \( r \) 个, 则
         \[ A = Q \cdot \operatorname{diag}(a_{1}, a_{2}, \dots, a_{r}, 0, \dots, 0) \cdot Q^{-1} \]

         令 \( B_{1} = Q^{-1}\operatorname{diag}(a_{1}, 0, \dots, 0, \dots, 0)Q \), \( B_{2} = Q^{-1}\operatorname{diag}(0, a_{2}, 0, \dots, 0)Q \), \dots, \( B_{r} = Q^{-1}\operatorname{diag}(0, \dots, a_{r}, 0)Q \)

         则 \( Q(B_{1} + B_{2} + \cdots + B_{r})Q = \operatorname{diag}(a_{1}, a_{2}, \dots, a_{r}, 0, \dots, 0) = Q^{-1}AQ \)

         故 \( A = B_{1} + B_{2} + \cdots + B_{r} \) 又每个 \( B_{i} \) (\( i=1,2,\dots,r \)) 都为对称阵且秩为1

         证毕.


     \paragraph{} %3
         \( A = \alpha^{\mathrm{T}}\alpha = \begin{pmatrix}
             1  \\
             -2 \\
             3
         \end{pmatrix}(1\ -2\ 3) = \begin{pmatrix}
             1  & -2 & 3  \\
             -2 & 4  & -6 \\
             3  & -6 & 9
         \end{pmatrix} \)

         则由 \( f_{A}(\lambda) = |\lambda E - A| = \begin{vmatrix}
             \lambda-1 & 2         & -3        \\
             2         & \lambda-4 & 6         \\
             -3        & 6         & \lambda-9
         \end{vmatrix} = \lambda^{2}(\lambda-14) \) 故特征值为 \( 0, 14, 0 \)

         又 \( 0E-A = \begin{pmatrix}
             -1 & 2  & -3 \\
             2  & -4 & 6  \\
             -3 & 6  & -9
         \end{pmatrix} \rightarrow \begin{pmatrix}
             1 & -2 & 3 \\
               &    &   \\
               &    &
         \end{pmatrix} \) 得 \( \alpha_{1} = (2, 1, 0)^{\mathrm{T}} \), \( \alpha_{2} = (-3, 0, 1)^{\mathrm{T}} \)

         \( 14E-A = \begin{pmatrix}
             13 & 2  & -3 \\
             2  & 10 & 6  \\
             -3 & 6  & 5
         \end{pmatrix} \rightarrow \begin{pmatrix}
             1 &   & -\frac{1}{3} \\
               & 1 & \frac{2}{3}  \\
               &   &
         \end{pmatrix} \) 得 \( \alpha_{3} = (1, -2, 3)^{\mathrm{T}} \)

         单位正交化 \( \rightarrow p_{1} = \left( \frac{2}{\sqrt{5}}, \frac{1}{\sqrt{5}}, 0 \right)^{\mathrm{T}} \), \( p_{2} = \left( -\frac{3\sqrt{70}}{70}, \frac{3\sqrt{70}}{35}, \frac{\sqrt{70}}{14} \right)^{\mathrm{T}} \), \( p_{3} = \left( \frac{1}{\sqrt{14}}, -\frac{2}{\sqrt{14}}, \frac{3}{\sqrt{14}} \right)^{\mathrm{T}} \)

         故 \( Q = \begin{pmatrix}
             \frac{2}{\sqrt{5}} & -\frac{3\sqrt{70}}{70} & \frac{1}{\sqrt{14}}  \\
             \frac{1}{\sqrt{5}} & \frac{3\sqrt{70}}{35}  & -\frac{2}{\sqrt{14}} \\
             0                  & \frac{\sqrt{70}}{14}   & \frac{3}{\sqrt{14}}
         \end{pmatrix} \)


     \paragraph{} %4
         证明: 由于 \( Ax = \lambda x \), 则
         \[ (A\bar{x})^{\mathrm{T}}x = \bar{x}^{\mathrm{T}}A^{\mathrm{T}}x = -\bar{x}^{\mathrm{T}}(Ax) = -\bar{x}^{\mathrm{T}}(\lambda x) = -\lambda(\bar{x}^{\mathrm{T}}x) \]
         \[ (A\bar{x})^{\mathrm{T}}x = (\overline{A}\overline{x})^{\mathrm{T}}x = (\overline{Ax})^{\mathrm{T}}x = (\overline{\lambda x})^{\mathrm{T}}x = \bar{\lambda}(\bar{x}^{\mathrm{T}}x) \]

         则 \( (\lambda+\bar{\lambda})\bar{x}^{\mathrm{T}}x = 0 \), 又 \( x \neq 0 \), 于是 \( \bar{x}^{\mathrm{T}}x \neq 0 \)

         故 \( \lambda + \bar{\lambda} = 0 \) 则 \( \lambda \) 是纯虚数.
